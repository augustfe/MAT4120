\subsection{}

\begin{exercise}
  What is a doubly stochastic matrix?
  Is the set of doubly stochastic matrices convex?
\end{exercise}

\begin{solution}
  A doubly stochastic matrix is a matrix with non-negative entries
  where each row and column sums to one.
  The set is convex, as it is the convex hull of the set of
  permutation matrices.
\end{solution}

\begin{exercise}
  What does it mean that a vector \(x\) is majorized by a vector \(y\)?
\end{exercise}

\begin{solution}
  A vector is majorized if
  \begin{equation}
    \sum_{i = 1}^{k} x_i \leq \sum_{i = 1}^{k} y_i
  \end{equation}
  for all \(k = 1, \ldots, n - 1\), and \(\sum_{i = 1}^{n} x_i =
  \sum_{i = 1}^{n} y_i\).
\end{solution}

\begin{exercise}
  State the Gale Ryser theorem, in particular its connection to
  majorization and given row/column sums.
\end{exercise}

\begin{solution}
  Let \(R = (r_1, \ldots, r_m)\) and \(S = (s_1, \ldots, s_n)\) be
  non-negative integral vectors with the same sum.
  The Gale Ryser theorem then tells us that is an \(m \times n\)
  \((0, 1)\)-matrix \(A\) with
  \begin{equation}
    \begin{split}
      \sum_{j = 1}^{n} a_{ij} &= r_i \qquad (i \leq m) \\
      \sum_{i = 1}^{m} a_{ij} &= s_j \qquad (j \leq n)
    \end{split}
  \end{equation}
  if and only if \(S \preceq R^*\).

  Here, \(R^*\) is the conjugate of the integral vector \(R\), defined by
  \begin{equation}
    r_k^* = \abs*{\{i : r_i \geq k\}} \qquad (k \leq n).
  \end{equation}

  Therefore, there exists a \((0, 1)\)-matrix with row sums \(R\) and
  columns sums \(S\) if and only if \(S\) majorizes the conjugate of \(R\).
\end{solution}

\begin{exercise}
  Let \(R\) and \(S\) be vectors with positive integers.
  When can we find an integral matrix \(A\) with row sums \(R\) and
  column sums \(S\), and so that \(0 \leq A \leq C\) (with \(C\) integral)?
  Can this be related to graphs?
\end{exercise}

\begin{solution}
  Such a matrix exists if and only if
  \begin{equation}
    \sum_{\substack{i \in I \\ j \in J}}^{} c_{ij}
    \geq \sum_{j \in J}^{} s_j
    - \sum_{i \notin I}^{} r_i,
  \end{equation}
  for all \(I \subseteq \{1, 2, \ldots, m\} \) and \(J \subseteq \{1,
  2, \ldots, n\} \).

  This coincides with an integral flow of a bipartite graph with
  nodes \(u_1, \ldots, u_m\) and \(v_1, \ldots, v_n\), with \(b(u_i)
  = r_i\) and \(b(v_j) = s_j\).
  An integral flow \(x\) with \(\diver_x = b\) and \(0 \leq x \leq
  c\) then corresponds to the matrix \(C\) with the properties desired.
\end{solution}

\begin{exercise}
  What are the vertices/extreme points for the set of doubly
  stochastic matrices?
\end{exercise}

\begin{solution}
  The vertices are the permutation matrix.
  They clearly cannot be written as convex combinations of other
  doubly stochastic matrices.
  Additionally, for a doubly stochastic matrix \(A\), we can trace
  out a cycle going through non-integer entries.
  We can then construct a matrix with a positive one in even places,
  and negative ones in the odd places.
  As each entry lies in the open interval \((0, 1)\), there is an
  \(\varepsilon > 0\) such that \(A \pm \varepsilon V\) is a doubly
  stochastic matrix.
  Therefore, no other doubly stochastic matrix is an extreme point.
\end{solution}