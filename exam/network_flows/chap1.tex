\subsection{}

\begin{exercise}
  What is a flow in a graph?
\end{exercise}

\begin{solution}
  A flow is a function \( x : E \to \R\), or equivalently \(x \in \R^E\).
  \(x\) therefore assigns a value \(x(e)\) to each edge \(e \in E\).
  We typically require it to be non-negative.
\end{solution}

\begin{exercise}
  What is the divergence of a flow?
\end{exercise}

\begin{solution}
  The divergence of a flow \(x\) is the function \(\diver_x : V \to
  \R\), given by
  \begin{equation}
    \diver_x(v) = \sum_{e \in \delta^+(v)}^{} x(e) - \sum_{e \in
    \delta^-(v)}^{} x(e).
  \end{equation}
  It is therefore the difference between the outflow and inflow for
  all nodes connected to \(v\).
\end{solution}

\begin{exercise}
  What is a circulation in a graph?
\end{exercise}

\begin{solution}
  A circulation in a graph is a flow such that \(\diver_x(v) = 0\)
  for all \(v \in V\).
\end{solution}

\begin{exercise}
  What does Hoffman's circulation theorem say?
  The proof is rather technical, but you should understand the role
  of the auxiliary graph therein.
\end{exercise}

\begin{solution}
  Hoffman's circulation theorem states that for functions \(l, u : E
  \to \R\) satisfying \(l \leq u\), then there exists a circulation
  \(x\) such that
  \begin{equation}
    l \leq x \leq u
  \end{equation}
  if and only if
  \begin{equation}
    \sum_{e \in \delta^-(S)}^{} l(e)
    \leq
    \sum_{e \in \delta^+(S)}^{} u(e)
  \end{equation}
  for all \(S \subseteq V\).

  Intuitively, the requirement is that for each subset of nodes, the
  capacity of what can flow out from the subset must be more than
  what needs to flow in.\todo{Look closer at auxiliary graphs.}
\end{solution}

\begin{exercise}
  What is an \(s t\)-flow, and the value of a flow?
\end{exercise}

\begin{solution}
  An \(s t\)-flow is a flow along a path from \(s\) to \(t\), where
  the divergence of each internal node in the flow is zero.
  The value of the flow, how much flows from \(s\) to \(t\) is then
  equal to the flow out of \(s\) by the flow balance equations.
\end{solution}

\begin{exercise}
  What is an \(s t\)-cut?
\end{exercise}

\begin{solution}
  An \(s t\)-cut is an edge subset of the form \(K = \delta^+(S)\)
  for a vertex subset \(S \subseteq V\) with \(s \in S\) and \(t \notin S\).
\end{solution}

\begin{exercise}
  What is the maximum \(s t\)-flow problem, and what is the minimum
  \(s t\)-cut problem?
  What is the connection between the two?
\end{exercise}

\begin{solution}
  The maximum \(s t\)-flow problem is the problem of finding the \(s
  t\)-flow which maximizes the value of the flow.
  Similarly, the minimum \(s t\)-cut problem the is the problem of
  finding the cut \(K\) which minimizes the capacities in the cut.

  Through duality theory, the optimal value for both problems is the same.
\end{solution}

\begin{exercise}
  How is Hoffman's theorem used in the proof?
  Algorithm (Ford Fulkerson and \(x\)-augmenting paths.)
\end{exercise}

\begin{solution}
  The proof uses the theorem by adding the edge \((t, s)\) with
  \(l(e) = u(e) = M\), the minimum cut capacity.
  We therefore just need to show that there is a circulation which
  satisfies the given constraints, as then \(\diver_x(s) = 0\), so
  \(\sum_{e \in \delta^+(s)}^{} x(e) = M\).
  We set \(l(e) = 0\) and \(u(e) = c(e)\) for each other edge \(e \in E\).

  Consider then a subset \(S\) with \(s \in S\) and \( t \notin S\).
  We then have that
  \begin{equation}
    \sum_{e \in \delta^-(S)}^{} l(e) = M + 0 = M,
  \end{equation}
  and
  \begin{equation}
    \sum_{e \in \delta^+(S)}^{} u(e) =
    \sum_{e \in \delta^+(S)}^{} c(e) =
    \capc_c(\delta^+(S)).
  \end{equation}
  The second condition in Hoffman's circulation theorem then reduces to
  \begin{equation}
    M \leq \capc_c(\delta^+(S))
  \end{equation}
  for all \(S \subseteq V\), which we know to be true as \(M\) is minimal.

  The Ford-Fulkerson algorithm starts out with a zero flow \(x = 0\).
  It then repeatedly looks for an \(x\)-augmenting path \(P\) in
  \(D_x\), and if it exists then increase the forward edges in the
  path by the determined amount \(\varepsilon\), while decreasing the
  value of the backward edges by \(\varepsilon\).
\end{solution}