\subsection{}

\begin{exercise}
  State the general combinatorial optimisation problem.
  What is the node-edge incidence matrix of graph?
\end{exercise}

\begin{solution}
  In a combinatorial optimisation problem, we consider a class
  \(\mathcal{F}\) which is a subset of some finite ground set \(E\).
  Associated with the problem is a weight function \( w : E \to \R
  \), defined for an \(F \in \mathcal{F}\) as \( w(F) = \sum_{e \in F} w(e) \).
  The problem is then simply given by
  \begin{equation}
    \max \{ w(F) : F \in \mathcal{F} \}.
  \end{equation}

  For a graph \(G = (V, E)\), the node-edge incidence matrix is the
  matrix \(A \in \{0, 1\}^{V \times E}\), such that the column
  corresponding to an edge \(e = (u, v)\) has two non-zero entries,
  corresponding to the nodes \(u\) and \(v\).
\end{solution}

\begin{exercise}
  What is the forest polytope \((F(G))\) of a graph?
\end{exercise}

\begin{solution}
  The forest polytope is the convex hull of all incidence vectors of
  all forests in \(G\), where an incidence vector is the vector
  \(\chi^F\) with \(\chi_e^F = 1\) if \(e \in F\) and zero otherwise.
\end{solution}

\begin{exercise}
  Can you describe \(F(G)\) as a polyhedron in \((Q)\)?
  What are the constraints?
  (\(x(E[S]) \geq \abs{S} - 1\)).
  Say something about the proof (%
    look at the vertices, characterise them as optimal solutions of
    an LP problem, show by complementary slackness that the greedy
    algorithm finds optimal solutions.
    This is an incidence vector%
  ).
\end{exercise}

\begin{solution}
  There is a lemma which states that \(x \in \{ 0, 1 \}^E \) is an
  incidence vector of a forest if and only if
  \begin{equation}
    x(E[S]) \leq \abs{S} - 1
    \qquad \text{for all} \ S \subseteq V.
  \end{equation}
  Considering the \(Q\) be the polytope defined by
  \begin{equation}
    \begin{split}
      x_e \geq 0 \qquad &\text{for all} \ e \in E \\
      x(E[S]) \leq \abs{S} - 1 \qquad &\text{for all} \ S \subseteq V.
    \end{split}
  \end{equation}
  The integral vectors in \(Q\) are precisely the incidence vectors
  by the lemma, so \( F(G) \subseteq Q \).

  Let now \(\bar{x}\) be a vertex of \(Q\), meaning we can find it as
  a unique optimal solution \(\bar{x} = \argmax_{x \in Q} c^T\), for
  some suitable \(c\).
  The dual of this LP problem is given by
  \begin{equation}
    \begin{split}
      \text{minimize} \quad& \sum_{S \subseteq V} y_S(\abs{S} - 1) \\
      \text{subject to} \\
      & \sum_{\{S : e_r \in E[S] \} }^{} y_S \geq c_e \qquad
      \text{for all} \ e \in E \\
      & y_S \geq 0 \qquad \text{for all} \ S \subseteq V.
    \end{split}
  \end{equation}

  Applying the greedy algorithm to the primal problem, we get a
  solution \( F = \{ e_1, \ldots, e_s \} \), and assume it is in that order.
  At the \(i\)-th iteration, we therefore find \( e_i = (u, v) \),
  joining together two components in the current solution into the
  new component \( V_i \).

  For the dual problem, we need to define the dual solution \(y\).
  We let \(y_S = 0\) for all \(S \notin \{V_1, \ldots, V_s\}\).
  We define the values for the remaining solutions recursively,
  starting from the back with \( y_{V_s} = c(e_r) \).
  By the complementary slackness condition for an edge \(e\), as
  \(x_e' > 0\), we have that \(\sum_{\{S : e_r \in E[S] \} }^{} y_S = c_e\).
  By our definition of \(y_{V_s}\), we are then required to set \(y_S
  = 0\) for all sets not including the endpoints of \(e_s\).

  We define the remaining components of \(y\) with the index set
  \begin{equation}
    I(j) = \{ i : j + 1 \leq i \leq s \ \text{and both end nodes of}
    \ e_j \ \text{are in} \ V_i\},
  \end{equation}
  for \(j \in \{1, \ldots, r - 1\} \).
  We then set
  \begin{equation}
    y_{V_j} = c(e_j) - \sum_{i \in I(j)}^{} y_{V_i}
  \end{equation}
  for \(j = r - 1, r - 2, \ldots, 1 \).

  We then check that the solution \(y\) is dual feasible, \(x'\) is
  primal feasible and that the complementary slackness condition holds.
  Both solutions are then optimal in their respective problems, and
  so by uniqueness we have \(\bar{x} = x'\), showing that every
  vertex of \(Q\) is integral, so \(Q = F(G)\).
\end{solution}

\begin{exercise}
  What is a formulation for a set \(S\)?
\end{exercise}

\begin{solution}
  With \( S \subseteq \{0, 1\}^n \), a formulation for \(S\) is a
  polyhedron \(P \subseteq \R^n\) which satisfies \(P \cap \{0, 1\}^n = S \).
\end{solution}

\begin{exercise}
  What is a Hamiltonian tour in a graph?
  Can you state (in)equalities which characterise incidence vectors  % chktex 36
  of Hamiltonian cycles?
  (Subtour/degree constraints)
\end{exercise}

\begin{solution}
  A Hamiltonian tour is a tour which visits each vertex in a graph exactly once.
  A Hamiltonian cycle similarly requires that the tour starts and
  stops in the same node.
  We can formulate this with inequalities, by requiring that each
  strict subset of the graph contains at most \( \abs{S} - 1\) edges,
  such that no interior cycle exists.
  Additionally, each vertex must have degree two.
\end{solution}

\begin{exercise}
  What is the travelling salesman problem?
\end{exercise}

\begin{solution}
  TSP is the problem of finding a Hamiltonian cycle which minimizes
  some distance metric, in the simplest case the physical distance a
  salesman must travel to pass through as number of cities, before
  ending up back at home.
\end{solution}

\begin{exercise}
  What is a separation oracle?
  Why do we use these?
\end{exercise}

\begin{solution}
  A separation oracle is an algorithm, which from a trial solution
  \(\bar{x}\) to some problem defined on a polyhedron
  \begin{equation}
    P = \{ x \in \R^n : Ax \leq b \},
  \end{equation}
  either confirms that \(\bar{x} \in P\), or returns a constraint
  which \(\bar{x}\) violates.

  We use these as we can very quickly encounter problems with a large
  number of constraints, wherein we do not need to actually consider
  all in order to get a solution.
  Some of the inequalities might be able to be reduced to each other.
  Figuring this out however can be a lot of work, so just getting the
  violated ones can be a nice time save.
\end{solution}