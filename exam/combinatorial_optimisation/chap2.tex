\subsection{}

\begin{exercise}
  What is an integral polyhedron?
  What is a rational polyhedron?
\end{exercise}

\begin{solution}
  From a polyhedron \( P \subset \R^n\) we define the integer hull \(P_I\) by
  \begin{equation}
    P_I = \conv(P \cap \Z^n).
  \end{equation}
  A polyhedron \(P\) is then integral if \(P = P_I\).
  A polyhedron is rational if it is defined by linear inequalities
  with only rational numbers.
\end{solution}

\begin{exercise}
  What is a totally unimodular matrix?
  Is a matrix with \(-1\) in all entries TU? % chktex 13
\end{exercise}

\begin{solution}
  A totally unimodular matrix has the property that each square
  submatrix has determinant \(-1\), \(0\) or \(1\).
  A matrix with all negative ones is TU, as each submatrix contains
  only linearly dependent rows, and thus all have determinant \(0\),
  except the trivial submatrices which only have the element \(-1\).
\end{solution}

\begin{exercise}
  If \(A\) is TU and \(b\) is integral, what can you say about the
  polyhedron \(P = \{x \in \R^n : A x \leq b\} \)?
\end{exercise}

\begin{solution}
  We then have that the polyhedron is integral.
  The proof is based around Cramer's rule, from which we can show
  that if a matrix \(C\) is integral with \(\det{C}\) is either
  \(-1\) or \(1\), then \(C^{-1}\) is also integral.
\end{solution}

\begin{exercise}
  When is the node-edge incidence matrix of a graph TU? % chktex 13
\end{exercise}

\begin{solution}
  The node-edge incidence matrix \(A_G\) of a graph \(G\) is TU if
  and only if \(G\) is bipartite.
  The proof here is based on the Ghouila--Houri criterion, which % chktex 8
  states that a \(-1, 0, 1\)-matrix \(A\) is TU if and only if for
  each subset of rows \(J \subseteq \{1, \ldots, n\}\) there are
  partitions \(J_1, J-2\) of \(J\) such that
  \begin{equation}
    \abs*{\sum_{j \in J_1}^{} a_{i j} - \sum_{j \in J_2}^{} a_{i j}}
    \quad\text{for} \ i = 1, \ldots, m.
  \end{equation}
\end{solution}

\begin{exercise}
  Explain the following concepts in a graph:
  \begin{itemize}
    \item Matching,
    \item Node cover,
    \item Node packing,
    \item Edge cover.
  \end{itemize}
\end{exercise}

\begin{solution}
  A matching is an edge subset wherein no node is incident to more
  than one edge.
  Each node in the subset is therefore either ``matched'' with
  another node, or solitary.

  A node cover is a node subset such that each edge has an endnode in the cover.
  A node packing on the other hand is a node subset wherein no nodes
  are adjacent.
  An edge cover is finally an edge subset such that each node is
  connected to one of the edges in the subset.

  In particular, in a bipartite graph we have a minmax relation where
  the maximum cardinality of a node packing equals the minimum
  cardinality of an edge cover.
\end{solution}