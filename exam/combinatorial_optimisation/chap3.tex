\subsection{}

\begin{exercise}
  Explain Christofides algorithm for finding a Hamiltonian tour in a
  weighted graph.
  What does it mean that this is a factor \(2\)-approximation algorithm?
\end{exercise}

\begin{solution}
  Christofides algorithm is a heuristic algorithm for the metric TSP,
  wherein the weight function satisfies the triangle inequality, i.e.\
  \begin{equation}
    w( u, v) \leq w(u, z) + w(z, v)
  \end{equation}
  for all \( u, v, z \in V \).
  It is a \(2\)-approximation algorithm in the sense that the
  solution we obtain is at most a factor of 2 away, i.e.\
  \begin{equation}
    w(x) \leq \alpha v(Q).
  \end{equation}

  The algorithm consists of first finding a spanning tree \(T\) of \(G\).
  Then, we double every edge of \(T\) in order to obtain an Eulerian
  graph, and then find an Eulerian tour \(\mathcal{T}\) in this graph.
  We then return the Hamiltonian tour that visits the vertices of
  \(G\) in the order of their first appearance in \(\mathcal{T}\).
\end{solution}

\begin{exercise}
  What is an enumeration tree?
  What do we mean by branch and bound?
\end{exercise}