\subsection{}

\begin{exercise}
  What is the convex/affine hull of a set?
  Is the convex hull a convex set?
\end{exercise}

\begin{solution}
  The convex hull $C$ of a set $A$ is the set of all convex
  combinations of points in $A$.
  It is also the smallest convex set in which $A$ is a subset.

  The affine hull of a set $A$ is the set of all affine combinations
  of elements in $A$, where an affine combination is given by
  \begin{equation}
    \sum_{i = 0}^{k} \mu_i x_i
  \end{equation}
  for $\{ x_i \}_{i = 0}^k \in A$ subject to $\sum_{i = 0}^{k} \mu_i
  = 1$ for some $k > 0$.
  It is also the smallest affine set containing $A$.
\end{solution}

\begin{exercise}
  What is a polytope?
  Explain why a polytope is compact.
\end{exercise}

\begin{solution}
  A polytope is the convex hull of a finite number of points, that is
  $P = \conv\{x_1, \ldots, x_t \} \subset \R^n$.
  Polytopes are compact as there exists a continuous function $f$
  from the standard simplex $S_t$ to $\R^n$, combined with the fact
  that the standard simplices are compact.
  A point in $S_t$ can be written as $(\lambda_1, \ldots, \lambda_t)$
  with $\sum_{i = 1}^{t} \lambda_i = 1$ and $\lambda \geq 0$.
  $f$ then maps that point to $\sum_{i = 1}^{t} \lambda_i x_i$.
\end{solution}

\begin{exercise}
  What is affine independence, and what is the connection to linear
  independence?
\end{exercise}

\begin{solution}
  Affine independence is closely related to linear independence.
  A set of vectors $x_1, \ldots, x_t$ are affinely independent, if
  $\sum_{i = 0}^{t} \lambda_i x_i = 0$ and $\sum_{i = 0}^{t}
  \lambda_i = 0$ implies that $\lambda_1 = \cdots = \lambda_t = 0$.
  Note that linear independence differs, as then the first condition
  alone implies that all the coefficients are zero.
  Additionally, if the vectors are affinely independent, then the
  vectors $x_i - x_1$ for $i = 2, \ldots, t$ are linearly independent.
\end{solution}

\begin{exercise}
  What is the (affine) dimension of a set.
\end{exercise}

\begin{solution}
  The affine dimension of a set is the maximal number of affinely
  independent vectors, minus one.
  It therefore coincides with the usual notion of dimensionality,
  i.e., the maximal number of linearly independent vectors.
\end{solution}

\begin{exercise}
  What is a simplex?
\end{exercise}

\begin{solution}
  A simplex is the convex hull of a set of affinely independent vectors.
  The standard simplex is for instance given by $S_t = \conv\{e_1,
  \ldots, e_t\}$, or equivalently $S_t = \{ x \in \R^n : x \geq 0,
  \sum_{i = 1}^{t} x_i = 1 \}$.
\end{solution}

\begin{exercise}
  What is the relative topology?
  Look at the unit disk in $\R^2$ embedded in $\R^3$.
  What is the relative interior?
  The interior?
  The relative boundary?
\end{exercise}

\begin{solution}
  Relative topology is a concept which is useful when classifying
  sets in $\R^n$ with dimensionality less than $n$, by considering
  the affine hull of that set.
  For instance, when considering the unit disk in $\R^2$ embedded in
  $\R^3$, that is,
  \begin{equation}
    S = \{ (x_1, x_2, 0) : x_1^2 + x_2^2 \leq 1 \},
  \end{equation}
  we find that the interior of $S$ is empty, as any slight change in
  the $x_3$ coordinate places us outside the set.
  Explicitly, any ball around a point $x$ in $S$ with radius $r > 0$
  contains points outside of $S$.
  This is obviously not entirely useful.

  Because of this, we consider the relative topology, where the
  relative interior is defined as the intersection of the ball
  described previously, along with the affine hull of $S$.
  In this case, $\aff(S)$ is given by the $x_1$-$x_2$ plane.
  This leads to the intuitive definition of an interior, where for
  instance $(0, 0, 0)$ is ``inside'' of $S$, as the intersection
  restricts us from considering movements in $x_3$.
  That is,
  \begin{equation}
    \rint(S) = \{ (x_1, x_2, 0) : x_1^2 + x_2^2 < 1 \}.
  \end{equation}

  The relative boundary is defined as usual, given by $\rbd(S) =
  \cl(S) \setminus \rint(S)$, here given by
  \begin{equation}
    \rint(S) = \{ (x_1, x_2, 0) : x_1^2 + x_2^2 = 1 \},
  \end{equation}
  as $S = \cl(S)$.
  Note that as the interior of $S$ is empty, the usual definition of
  a boundary gives us that $\bd(S) = S$, each point is at the boundary.
\end{solution}

\begin{exercise}
  Can all points in a convex set $S$ be expressed in terms of
  affinely independent points in $S$?
  If so, how many do you need?
  And do the convex combinations of these generate all of $S$?
\end{exercise}

\begin{solution}
  Yes, by Caratheodory's theorem, for $S \subseteq \R^n$ we need at
  most $n + 1$ affinely independent points in $S = \conv(S)$.
  Convex combinations of these selected points do not in general
  generate all of $S$, and need to be selected for each $x \in S$.
\end{solution}
