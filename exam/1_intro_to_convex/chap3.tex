\subsection{}

\begin{exercise}
  What is a supporting/separating hyperplane?
\end{exercise}

\begin{solution}
  Firstly, a hyperplane is a set of the form
  \begin{equation}
    H_{\alpha, a} = \{ x \in \R^n : a^T x = \alpha \}.
  \end{equation}
  Accompanying this, we have the halfspaces
  \begin{equation}
    \begin{split}
      H_{\alpha, a}^+ &= \{ x \in \R^n : a^T x \geq \alpha \}, \\
      H_{\alpha, a}^- &= \{ x \in \R^n : a^T x \leq \alpha \}.
    \end{split}
  \end{equation}
  If a set $S$ is contained entirely in one of these halfspaces, and
  the intersection is non-zero, we say that that halfspace is a
  supporting hyperplane.
  For instance, with a circle, the tangent at a point defines a
  hyperplane, wherein the circle is entirely on one side.

  Given two sets $S$ and $T$, $H$ is a separating hyperplane if $S
  \subseteq H^\pm$ and $T \subseteq H^\mp$, that is they are entirely
  separated by the hyperplane.
\end{solution}

\begin{exercise}
  Suppose that $C$ is closed and convex, and $x$ is not in $C$.
  Can $x$ and $C$ be separated with a hyperplane?
\end{exercise}

\begin{solution}
  As $x \notin C$, and $C$ is closed, there is a ball about $x$
  entirely outside of $C$, with some radius $r > 0$.
  Clearly then, any separating hyperplane is also strongly separating.
  Let $H$ be the hyperplane supporting $C$ at $p_C(x)$ with normal
  vector $x - p_C(x)$.
  As $x \notin C$, we have that $x \neq p_C(x)$.
  $H$ then separates $x$ and $C$.
\end{solution}

\begin{exercise}
  What does Farkas' Lemma say?
  Can you sum up its proof?
  (Separating hyperplane theorem.)
\end{exercise}

\begin{solution}
  Farkas' Lemma states that for $A \in \R^{m,n}$ and $b \in \R^n$,
  there exists an $x \geq 0$ satisfying $Ax = b$ if and only if for
  each $y \in \R^m$ with $y^T A \geq 0$ it also holds that $y^T b \geq 0$.

  The proof considers the convex cone generated by the columns of
  $A$, i.e., $C = \cone\{ a^1, \ldots, a^n \}$.
  A solution $x \geq 0$ to $Ax = b$ then corresponds to $b$ lying in the cone.
  The proof then assumes that such a solution exists.

  Further, if $y^T A \geq 0$, then
  \begin{equation}
    y^T b = y^T (Ax) = (y^T A) x \geq 0,
  \end{equation}
  as both $x$ and $y^T A$ are assumed non-negative.

  Note now that $C$ is closed and convex.
  If $Ax = b$ has no non-negative solution, then $b \notin C$, such
  that $b$ and $C$ can be strongly separated by the separating
  hyperplane theorem.
  There is therefore a non-zero vector $y \in \R^m$ and $\alpha > 0$
  such that $y^T x \geq \alpha$ for all $x \in C$ and $y^T b < \alpha$.
  As $C$ is a cone, $0 \in C$, such that $\alpha \leq 0$.

  Furthermore, we must have that $y^T x \geq 0$.
  Suppose for contradiction that we had some $x_0 \in C$ such that
  $y^T x_0 < 0$.
  We would then have
  \begin{equation}
    \alpha \leq y^T x_0 < 0.
  \end{equation}
  As $C$ is a cone, we also have that $\lambda x_0 \in C$ for all $\lambda > 0$.
  We could then choose $\lambda$ sufficiently large such that
  $\lambda y^T x_0 < \alpha$, a contradiction.

  Therefore, $y^T a^j \geq 0$, as each $a^j \in C$, so $y^T A \geq 0$.
  Furthermore, we in this case have that $y^T b < \alpha \leq 0$,
  proving the other direction.
\end{solution}
