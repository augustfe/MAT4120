\subsection{}

\begin{exercise}
  Write down the general form of an optimisation problem with
  equality and inequality constraints.
  What do we mean by regular points and active constraints?
\end{exercise}

\begin{solution}
  The general form of an optimisation problem with equality and
  inequality constraints is given by
  \begin{equation}
    \begin{split}
      \text{minimize} \quad& f(x) \\
      \text{subject to} \quad& \\
      & h_i(x) = 0 \quad \text{for } i = 1, \ldots, m \\
      & g_j(x) = 0 \quad \text{for } j = 1, \ldots, r.
    \end{split}
  \end{equation}
  Equivalently, we write \( h(x) = 0 \in \R^m \) and \( g(x) \leq 0 \in \R^r \).
  Active constraints are those which are active in an optimal
  solution, such that we only need to consider equalities, being
  \(h\) and the inequalities which are active (\(g_j(x) = 0\)).

  A feasible point \(x^*\) is called regular if the gradients of
  \(h_i\) and the active \(g_j\) are linearly independent at \(x^*\).
\end{solution}

\begin{exercise}
  What do we mean by the KKT conditions?
  What is the connection with Lagrange's multiplier method?
\end{exercise}

\begin{solution}
  The KKT conditions are optimality conditions for optimisation
  problems with equality and inequality constraints.
  Along with the conditions, we consider the Lagrangian
  \begin{equation}
    L(x, \lambda, \mu)
    = f(x)
    + \sum_{i = 1}^{m} \lambda_i h_i(x)
    + \sum_{j = 1}^{r} \mu_j g_j(x).
  \end{equation}
  A triple \((x^*, \lambda^*, \mu^*)\) is said to satisfy the KKT
  conditions if we have primal feasibility, i.e.\
  \begin{equation}
    g(x^*) \leq 0
    \quad\text{and}\quad
    h(x^*) = 0,
  \end{equation}
  dual feasibility,
  \begin{equation}
    \mu_j \geq 0
    \quad \text{for } j = 1, \ldots, r,
  \end{equation}
  stationarity,
  \begin{equation}
    \nabla_x L(x^*, \lambda^*, \mu^*)
    = \nabla f(x)
    + \sum_{i = 1}^{m} \lambda_i \nabla h_i(x^*)
    + \sum_{j = 1}^{r} \mu_j \nabla g_j(x^*)
  \end{equation}
  and complimentary slackness, \(\mu^*_i g_i(x^*) = 0\).
  If \(x^*\) is regular, then the multiplies are unique, as the
  active inequality constraints are linearly independent.
\end{solution}
