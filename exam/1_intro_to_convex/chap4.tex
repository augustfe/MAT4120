
\subsection{}

\begin{exercise}
  What is a face of a convex set?
  What are the faces of different dimensions for the unit cube of $\R^3$?
\end{exercise}

\begin{solution}
  A face $F$ of a convex set $C$ is a set wherein $x = \lambda x_1 +
  (1 - \lambda) x_2 \in F$ implies that both $x_1$ and $x_2$ lie in
  $F$, for each such possible convex combination in $C$.

  The faces of dimension $0$ for the unit cube of $\R^3$ is each vertex.
  Of dimension $1$, we have the edges, while we for dimension $2$
  have the surfaces typically called faces.
  For dimension $3$, we have the full cube.
  Not sure if defined, but the face of dimension $-1$ might logically
  be concluded to be the empty set, as it doesn't contain any
  affinely independent points, and fulfils the requirements for a face.
\end{solution}

\begin{exercise}
  What is an extreme point?
  What are those for the unit cube and the unit sphere?
\end{exercise}

\begin{solution}
  An extreme point of a set $C$ is a point $x \in C$ where $x_1, x_2
  \in C$ such that $x = \tfrac{1}{2} (x_1 + x_2)$ implies that $x_1 = x = x_2$.

  For the unit cube, these are the vertices, while on the unit sphere
  this is every point on the boundary, satisfying $\norm{x} = 1$.
\end{solution}

\begin{exercise}
  Given a polytope $\conv(x_1, \ldots, x_t)$, what can you say about
  the extreme points?
\end{exercise}

\begin{solution}
  We have that the only candidates for extreme points are $x_1,
  \ldots, x_t$, however we do not necessarily have that all $x_1,
  \ldots, x_t$ are extreme.
\end{solution}

\begin{exercise}
  Let $S$ be a subset of vectors with all components being either 0 or 1.
  Explain why the extreme points of $\conv(S)$ are precisely $S$.
\end{exercise}

\begin{solution}
  We have that $S$ is a finite set of vectors, so the only candidates
  for the extreme points are precisely those vectors.
  In order for one of the points to not be an extreme point, we
  require that we write it as a combination of other points.
  Considering a specific component, say a $0$, we would either need
  for both corresponding components to be zero, or have opposite sign.
  Clearly, as there are no negative components involved, they must be equal.
  Similarly, for a component $1$, we must have either numbers above
  and below, or equivalence.
  As none are above, they must also be equal.
  The vectors are therefore identical, so the points are extreme.
\end{solution}

\begin{exercise}
  What is the recession cone of a closed convex set?
\end{exercise}

\begin{solution}
  The recession cone of a closed convex set $C$ contains all
  direction vectors, such that each halfline starting at a point in
  $C$ is entirely contained in $C$.
  Explicitly, the set of halflines starting at a point $x \in C$, given by
  \begin{equation}
    \rec(C, x) = \{ z \in \R^n : x + \lambda z \in C \ \forall
    \lambda \geq 0 \},
  \end{equation}
  is actually independent of the chosen point $x$.\todo{Read up on
  why it is independent.}
\end{solution}

\begin{exercise}
  Suppose $C$ is a convex compact set.
  What does Minkowski's theorem say about $C$?
  ($C = \conv(\ext(C))$.)
\end{exercise}

\begin{solution}
  Minkowski's theorem tells us that a convex compact set $C$ is the
  convex hull of its extreme points.
  This build upon the inner description of closed convex sets, which
  states that a non-empty and line-free closed convex set $C$ is the
  convex hull of its extreme points \emph{and extreme half-lines}.
  As $C$ in our case is compact, it is necessarily bounded, and
  therefore contains no half-lines.
\end{solution}

\begin{exercise}
  What is a vertex of a polyhedron?
  What is the connection between vertices and extreme points?
\end{exercise}

\begin{solution}
  We consider a non-empty polyhedron $P = \{ x \in \R^n : Ax \leq b
  \}$ for some $A \in \R^{m, n}$ and $b \in \R^m$.
  Each vertex in a polyhedron is then the solution of $n$ linearly
  independent equations from the linear system.

  As each such vertex lies in the polyhedron, $x_0 \in P$, it is also
  an extreme point.
  Assume for contradiction that we can write $x_0 = \tfrac{1}{2} (x_1
  + x_2)$ with $x_1 \neq x_2 \in P$.
  Then,
  \begin{equation}
    A_0 x_0 = A_0 (\tfrac{1}{2} (x_1 + x_2)) = \tfrac{1}{2} A_0 x_1 +
    \tfrac{1}{2} A_0 x_2 < b,
  \end{equation}
  where we have a strict inequality as $x_0$ is a unique solution, as
  the equations are linearly independent, contradicting $A_0 x_0 = b$.
\end{solution}

\begin{exercise}
  Why is the intersection of two polytopes again a polytope?
\end{exercise}

\begin{solution}
  Consider two polytopes
  \begin{equation}
    \begin{split}
      P_1 &= \{ x : A_1 x \leq b_1 \}, \\
      P_2 &= \{ x : A_2 x \leq b_2 \}.
    \end{split}
  \end{equation}
  The intersection is then given by
  \begin{equation}
    P_1 \cap P_2
    = \{ x :
      A_1 x \leq b_1
      \text{ and }
      A_2 x \leq b_2
    \},
  \end{equation}
  which we can achieve as a single polygon with
  \begin{equation}
    P_1 \cap P_2
    = \left\{
      x :
      \begin{bmatrix}
        A_1 \\ A_2
      \end{bmatrix}
      x \leq
      \begin{bmatrix}
        b_1 \\ b_2
      \end{bmatrix}
    \right\}.
  \end{equation}
\end{solution}

\begin{exercise}
  What does the main theorem for polyhedra say?
\end{exercise}

\begin{solution}
  The main theorem for polyhedra states that for each polyhedra \(P
  \subseteq \R^n\) there are finite sets \(V\) and \(Z\) such that
  \begin{equation}
    P = \conv(V) + \cone(Z).
  \end{equation}
  Additionally, for finite sets \(V\) and \(Z\), \(\conv(V) +
  \cone(Z)\) is a polyhedron.
  For a pointed polyhedra, $V$ is the set of vertices, while $Z$ is
  the set of direction vectors for each extreme half-line.
\end{solution}
