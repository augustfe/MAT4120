\section{An introduction to convexity}

\subsection{}

\begin{exercise}
  What is a convex set?
  What is an affine set?
  Examples:\@ the unit cube, hyperplanes.
\end{exercise}

\begin{solution}
  A convex set is intuitively a set where for each pair of two
  points, all the points one the line between those two points are
  also in the set.
  This for instance means that the unit circle is convex, while a
  typical drawing of a star is not, as connecting two ``points''
  places you outside the star.
  Explicitly, we have that a set $C$ is convex if and only if
  \begin{equation}
    \lambda x + (1 - \lambda) y \in C
  \end{equation}
  for all $x, y \in C$ and $\lambda \in [0, 1]$.

  An affine set is practically a linear subspace centred around some
  point other than zero.
  It is an affine set if and only if it is the solution set to some
  linear equation.
  That is, for a non-empty set $S \subseteq \R^n$, there is a matrix
  $A  \in \R^{m,n}$ and a vector $b \in \R^m$ such that
  \begin{equation}
    S = \{ x \in \R^n : Ax = b \}.
  \end{equation}
  It can equivalently be written as $C = L + x_0 = \{ x + x_0 : x \in
  L \}$ for some linear subspace $L$ of $\R^n$.

  To see the equivalence, let $x_0$ be a solution of the linear
  equation, such that $Ax_0 = b$.
  Then, $L$ is the kernel of $A$, as then for $z \in S$ we have
  \begin{equation}
    Az = A(x + x_0) = Ax + Ax_0 = b \implies Ax = 0.
  \end{equation}
\end{solution}

\begin{exercise}
  Is the sum of convex sets also convex?
\end{exercise}

\begin{solution}
  Yes.
  Consider two convex sets $A,B \subseteq \R^n$.
  We then have that the sum of the sets is given by
  \begin{equation}
    A + B = \{ a + b : a \in A, b \in B \}.
  \end{equation}
  As $A$ and $B$ are convex, we have for $a_1, a_2 \in A$, $b_1, b_2
  \in B$ and $\lambda \in [0, 1]$ that
  \begin{equation}
    \lambda a_1 + (1 - \lambda) a_2 + \lambda b_2 + (1 - \lambda) b_2
    = \lambda (a_1 + b_1) + (1 - \lambda) (a_2 + b_2)
    \in A + B.
  \end{equation}
\end{solution}

\begin{exercise}
  What is a cone?
  Example:\@ First octant in $\R^3$.
\end{exercise}

\begin{solution}
  A cone $C$ is a set with the property that a non-negative
  combination of two points in the cone, is also in the cone.
  That is, for $\lambda_1, \lambda_2 \geq 0$ and $x, y \in C$, we have
  \begin{equation}
    \lambda_1 x + \lambda_2 y \in C.
  \end{equation}
  The first octant in $\R^3$, often denoted $\R^3_+$, is an example
  of such a space.
\end{solution}

\begin{exercise}
  What is a polyhedron?
\end{exercise}

\begin{solution}
  A polyhedron is the solution set to any linear equation $Ax \leq b$.
\end{solution}

\subsection{}

\begin{exercise}
  What is the convex/affine hull of a set?
  Is the convex hull a convex set?
\end{exercise}

\begin{solution}
  The convex hull $C$ of a set $A$ is the set of all convex
  combinations of points in $A$.
  It is also the smallest convex set in which $A$ is a subset.

  The affine hull of a set $A$ is the set of all affine combinations
  of elements in $A$, where an affine combination is given by
  \begin{equation}
    \sum_{i = 0}^{k} \mu_i x_i
  \end{equation}
  for $\{ x_i \}_{i = 0}^k \in A$ subject to $\sum_{i = 0}^{k} \mu_i
  = 1$ for some $k > 0$.
  It is also the smallest affine set containing $A$.
\end{solution}

\begin{exercise}
  What is a polytope?
  Explain why a polytope is compact.
\end{exercise}

\begin{solution}
  A polytope is the convex hull of a finite number of points, that is
  $P = \conv\{x_1, \ldots, x_t \} \subset \R^n$.
  Polytopes are compact as there exists a continuous function $f$
  from the standard simplex $S_t$ to $\R^n$, combined with the fact
  that the standard simplices are compact.
  A point in $S_t$ can be written as $(\lambda_1, \ldots, \lambda_t)$
  with $\sum_{i = 1}^{t} \lambda_i = 1$ and $\lambda \geq 0$.
  $f$ then maps that point to $\sum_{i = 1}^{t} \lambda_i x_i$.
\end{solution}

\begin{exercise}
  What is affine independence, and what is the connection to linear
  independence?
\end{exercise}

\begin{solution}
  Affine independence is closely related to linear independence.
  A set of vectors $x_1, \ldots, x_t$ are affinely independent, if
  $\sum_{i = 0}^{t} \lambda_i x_i = 0$ and $\sum_{i = 0}^{t}
  \lambda_i = 0$ implies that $\lambda_1 = \cdots = \lambda_t = 0$.
  Note that linear independence differs, as then the first condition
  alone implies that all the coefficients are zero.
  Additionally, if the vectors are affinely independent, then the
  vectors $x_i - x_1$ for $i = 2, \ldots, t$ are linearly independent.
\end{solution}

\begin{exercise}
  What is the (affine) dimension of a set.
\end{exercise}

\begin{solution}
  The affine dimension of a set is the maximal number of affinely
  independent vectors, minus one.
  It therefore coincides with the usual notion of dimensionality,
  i.e., the maximal number of linearly independent vectors.
\end{solution}

\begin{exercise}
  What is a simplex?
\end{exercise}

\begin{solution}
  A simplex is the convex hull of a set of affinely independent vectors.
  The standard simplex is for instance given by $S_t = \conv\{e_1,
  \ldots, e_t\}$, or equivalently $S_t = \{ x \in \R^n : x \geq 0,
  \sum_{i = 1}^{t} x_i = 1 \}$.
\end{solution}

\begin{exercise}
  What is the relative topology?
  Look at the unit disk in $\R^2$ embedded in $\R^3$.
  What is the relative interior?
  The interior?
  The relative boundary?
\end{exercise}

\begin{solution}
  Relative topology is a concept which is useful when classifying
  sets in $\R^n$ with dimensionality less than $n$, by considering
  the affine hull of that set.
  For instance, when considering the unit disk in $\R^2$ embedded in
  $\R^3$, that is,
  \begin{equation}
    S = \{ (x_1, x_2, 0) : x_1^2 + x_2^2 \leq 1 \},
  \end{equation}
  we find that the interior of $S$ is empty, as any slight change in
  the $x_3$ coordinate places us outside the set.
  Explicitly, any ball around a point $x$ in $S$ with radius $r > 0$
  contains points outside of $S$.
  This is obviously not entirely useful.

  Because of this, we consider the relative topology, where the
  relative interior is defined as the intersection of the ball
  described previously, along with the affine hull of $S$.
  In this case, $\aff(S)$ is given by the $x_1$-$x_2$ plane.
  This leads to the intuitive definition of an interior, where for
  instance $(0, 0, 0)$ is ``inside'' of $S$, as the intersection
  restricts us from considering movements in $x_3$.
  That is,
  \begin{equation}
    \rint(S) = \{ (x_1, x_2, 0) : x_1^2 + x_2^2 < 1 \}.
  \end{equation}

  The relative boundary is defined as usual, given by $\rbd(S) =
  \cl(S) \setminus \rint(S)$, here given by
  \begin{equation}
    \rint(S) = \{ (x_1, x_2, 0) : x_1^2 + x_2^2 = 1 \},
  \end{equation}
  as $S = \cl(S)$.
  Note that as the interior of $S$ is empty, the usual definition of
  a boundary gives us that $\bd(S) = S$, each point is at the boundary.
\end{solution}

\begin{exercise}
  Can all points in a convex set $S$ be expressed in terms of
  affinely independent points in $S$?
  If so, how many do you need?
  And do the convex combinations of these generate all of $S$?
\end{exercise}

\begin{solution}
  Yes, by Caratheodory's theorem, for $S \subseteq \R^n$ we need at
  most $n + 1$ affinely independent points in $S = \conv(S)$.
  Convex combinations of these selected points do not in general
  generate all of $S$, and need to be selected for each $x \in S$.
\end{solution}

\subsection{}

\begin{exercise}
  What is a supporting/separating hyperplane?
\end{exercise}

\begin{solution}
  Firstly, a hyperplane is a set of the form
  \begin{equation}
    H_{\alpha, a} = \{ x \in \R^n : a^T x = \alpha \}.
  \end{equation}
  Accompanying this, we have the halfspaces
  \begin{equation}
    \begin{split}
      H_{\alpha, a}^+ &= \{ x \in \R^n : a^T x \geq \alpha \}, \\
      H_{\alpha, a}^- &= \{ x \in \R^n : a^T x \leq \alpha \}.
    \end{split}
  \end{equation}
  If a set $S$ is contained entirely in one of these halfspaces, and
  the intersection is non-zero, we say that that halfspace is a
  supporting hyperplane.
  For instance, with a circle, the tangent at a point defines a
  hyperplane, wherein the circle is entirely on one side.

  Given two sets $S$ and $T$, $H$ is a separating hyperplane if $S
  \subseteq H^\pm$ and $T \subseteq H^\mp$, that is they are entirely
  separated by the hyperplane.
\end{solution}

\begin{exercise}
  Suppose that $C$ is closed and convex, and $x$ is not in $C$.
  Can $x$ and $C$ be separated with a hyperplane?
\end{exercise}

\begin{solution}
  As $x \notin C$, and $C$ is closed, there is a ball about $x$
  entirely outside of $C$, with some radius $r > 0$.
  Clearly then, any separating hyperplane is also strongly separating.
  Let $H$ be the hyperplane supporting $C$ at $p_C(x)$ with normal
  vector $x - p_C(x)$.
  As $x \notin C$, we have that $x \neq p_C(x)$.
  $H$ then separates $x$ and $C$.
\end{solution}

\begin{exercise}
  What does Farkas' Lemma say?
  Can you sum up its proof?
  (Separating hyperplane theorem.)
\end{exercise}

\begin{solution}
  Farkas' Lemma states that for $A \in \R^{m,n}$ and $b \in \R^n$,
  there exists an $x \geq 0$ satisfying $Ax = b$ if and only if for
  each $y \in \R^m$ with $y^T A \geq 0$ it also holds that $y^T b \geq 0$.

  The proof considers the convex cone generated by the columns of
  $A$, i.e., $C = \cone\{ a^1, \ldots, a^n \}$.
  A solution $x \geq 0$ to $Ax = b$ then corresponds to $b$ lying in the cone.
  The proof then assumes that such a solution exists.

  Further, if $y^T A \geq 0$, then
  \begin{equation}
    y^T b = y^T (Ax) = (y^T A) x \geq 0,
  \end{equation}
  as both $x$ and $y^T A$ are assumed non-negative.

  Note now that $C$ is closed and convex.
  If $Ax = b$ has no non-negative solution, then $b \notin C$, such
  that $b$ and $C$ can be strongly separated by the separating
  hyperplane theorem.
  There is therefore a non-zero vector $y \in \R^m$ and $\alpha > 0$
  such that $y^T x \geq \alpha$ for all $x \in C$ and $y^T b < \alpha$.
  As $C$ is a cone, $0 \in C$, such that $\alpha \leq 0$.

  Furthermore, we must have that $y^T x \geq 0$.
  Suppose for contradiction that we had some $x_0 \in C$ such that
  $y^T x_0 < 0$.
  We would then have
  \begin{equation}
    \alpha \leq y^T x_0 < 0.
  \end{equation}
  As $C$ is a cone, we also have that $\lambda x_0 \in C$ for all $\lambda > 0$.
  We could then choose $\lambda$ sufficiently large such that
  $\lambda y^T x_0 < \alpha$, a contradiction.

  Therefore, $y^T a^j \geq 0$, as each $a^j \in C$, so $y^T A \geq 0$.
  Furthermore, we in this case have that $y^T b < \alpha \leq 0$,
  proving the other direction.
\end{solution}

\subsection{}

\begin{exercise}
  What is a face of a convex set?
  What are the faces of different dimensions for the unit cube of $\R^3$?
\end{exercise}

\begin{solution}
  A face $F$ of a convex set $C$ is a set wherein $x = \lambda x_1 +
  (1 - \lambda) x_2 \in F$ implies that both $x_1$ and $x_2$ lie in
  $F$, for each such possible convex combination in $C$.

  The faces of dimension $0$ for the unit cube of $\R^3$ is each vertex.
  Of dimension $1$, we have the edges, while we for dimension $2$
  have the surfaces typically called faces.
  For dimension $3$, we have the full cube.
  Not sure if defined, but the face of dimension $-1$ might logically
  be concluded to be the empty set, as it doesn't contain any
  affinely independent points, and fulfils the requirements for a face.
\end{solution}

\begin{exercise}
  What is an extreme point?
  What are those for the unit cube and the unit sphere?
\end{exercise}

\begin{solution}
  An extreme point of a set $C$ is a point $x \in C$ where $x_1, x_2
  \in C$ such that $x = \tfrac{1}{2} (x_1 + x_2)$ implies that $x_1 = x = x_2$.

  For the unit cube, these are the vertices, while on the unit sphere
  this is every point on the boundary, satisfying $\norm{x} = 1$.
\end{solution}

\begin{exercise}
  Given a polytope $\conv(x_1, \ldots, x_t)$, what can you say about
  the extreme points?
\end{exercise}

\begin{solution}
  We have that the only candidates for extreme points are $x_1,
  \ldots, x_t$, however we do not necessarily have that all $x_1,
  \ldots, x_t$ are extreme.
\end{solution}

\begin{exercise}
  Let $S$ be a subset of vectors with all components being either 0 or 1.
  Explain why the extreme points of $\conv(S)$ are precisely $S$.
\end{exercise}

\begin{solution}
  We have that $S$ is a finite set of vectors, so the only candidates
  for the extreme points are precisely those vectors.
  In order for one of the points to not be an extreme point, we
  require that we write it as a combination of other points.
  Considering a specific component, say a $0$, we would either need
  for both corresponding components to be zero, or have opposite sign.
  Clearly, as there are no negative components involved, they must be equal.
  Similarly, for a component $1$, we must have either numbers above
  and below, or equivalence.
  As none are above, they must also be equal.
  The vectors are therefore identical, so the points are extreme.
\end{solution}

\begin{exercise}
  What is the recession cone of a closed convex set?
\end{exercise}

\begin{solution}
  The recession cone of a closed convex set $C$ contains all
  direction vectors, such that each halfline starting at a point in
  $C$ is entirely contained in $C$.
  Explicitly, the set of halflines starting at a point $x \in C$, given by
  \begin{equation}
    \rec(C, x) = \{ z \in \R^n : x + \lambda z \in C \ \forall
    \lambda \geq 0 \},
  \end{equation}
  is actually independent of the chosen point $x$.\todo{Read up on
  why it is independent.}
\end{solution}

\begin{exercise}
  Suppose $C$ is a convex compact set.
  What does Minkowski's theorem say about $C$?
  ($C = \conv(\ext(C))$.)
\end{exercise}

\begin{solution}
  Minkowski's theorem tells us that a convex compact set $C$ is the
  convex hull of its extreme points.
  This build upon the inner description of closed convex sets, which
  states that a non-empty and line-free closed convex set $C$ is the
  convex hull of its extreme points \emph{and extreme half-lines}.
  As $C$ in our case is compact, it is necessarily bounded, and
  therefore contains no half-lines.
\end{solution}

\begin{exercise}
  What is a vertex of a polyhedron?
  What is the connection between vertices and extreme points?
\end{exercise}

\begin{solution}
  We consider a non-empty polyhedron $P = \{ x \in \R^n : Ax \leq b
  \}$ for some $A \in \R^{m, n}$ and $b \in \R^m$.
  Each vertex in a polyhedron is then the solution of $n$ linearly
  independent equations from the linear system.

  As each such vertex lies in the polyhedron, $x_0 \in P$, it is also
  an extreme point.
  Assume for contradiction that we can write $x_0 = \tfrac{1}{2} (x_1
  + x_2)$ with $x_1 \neq x_2 \in P$.
  Then,
  \begin{equation}
    A_0 x_0 = A_0 (\tfrac{1}{2} (x_1 + x_2)) = \tfrac{1}{2} A_0 x_1 +
    \tfrac{1}{2} A_0 x_2 < b,
  \end{equation}
  where we have a strict inequality as $x_0$ is a unique solution, as
  the equations are linearly independent, contradicting $A_0 x_0 = b$.
\end{solution}

\begin{exercise}
  Why is the intersection of two polytopes again a polytope?
\end{exercise}

\begin{solution}
  Consider two polytopes
  \begin{equation}
    \begin{split}
      P_1 &= \{ x : A_1 x \leq b_1 \}, \\
      P_2 &= \{ x : A_2 x \leq b_2 \}.
    \end{split}
  \end{equation}
  The intersection is then given by
  \begin{equation}
    P_1 \cap P_2
    = \{ x :
      A_1 x \leq b_1
      \text{ and }
      A_2 x \leq b_2
    \},
  \end{equation}
  which we can achieve as a single polygon with
  \begin{equation}
    P_1 \cap P_2
    = \left\{
      x :
      \begin{bmatrix}
        A_1 \\ A_2
      \end{bmatrix}
      x \leq
      \begin{bmatrix}
        b_1 \\ b_2
      \end{bmatrix}
    \right\}.
  \end{equation}
\end{solution}

\begin{exercise}
  What does the main theorem for polyhedra say?
\end{exercise}

\begin{solution}
  The main theorem for polyhedra states that for each polyhedra $P
  \subseteq \R^n$ there are finite sets $V$ and $Z$ such that
  \begin{equation}
    P = \conv(V) + \cone(Z).
  \end{equation}
  For a pointed polyhedra, $V$ is the set of vertices, while $Z$ is
  the set of direction vectors for each extreme half-line.
\end{solution}
