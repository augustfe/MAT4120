\section{Network flows and combinatorial matrix theory}

\begin{exercise}
  Let $x$ be a flow in a a graph $D$. Show that $\sum_{v \in V} \diver_x(v) = 0$.
\end{exercise}

\begin{solution}
  We have the definition of $\diver_x : V \to \R$ as
  \begin{equation}
    \diver_x(v) = \sum_{e \in \delta^+(v)} x_e - \sum_{e \in \delta^-(v)} x_e,
  \end{equation}
  where $\delta^+(v)$ and $\delta^-(v)$ are the sets of outgoing and incoming edges of vertex $v$, respectively.
  Thus, for a graph $D = (V, E)$ we have
  \begin{align*}
    \sum_{v \in V} \diver_x(v)
    &= \sum_{v \in V} \left( \sum_{e \in \delta^+(v)} x_e - \sum_{e \in \delta^-(v)} x_e \right) \\
    &= \sum_{v \in V} \sum_{e \in \delta^+(v)} x_e - \sum_{v \in V} \sum_{e \in \delta^-(v)} x_e \\
    &= \sum_{e \in E} x_e - \sum_{e \in E} x_e \\
    &= 0,
  \end{align*}
  as each edge contributes exactly once to the sum of outgoing flows and exactly once to the sum of incoming flows.
  Therefore, we conclude that $\sum_{v \in V} \diver_x(v) = 0$.
\end{solution}

\begin{exercise}
  Let $x$ be a circulation in a graph $D = (V, E)$ and let $S \subseteq V$.
  Prove that $\sum_{v \in S} \delta^-(v) = \sum_{v \in S} \delta^+(v)$.
  (Hint:\@ sum the flow balance equations for vertices in S.)
\end{exercise}

\begin{solution}
  \textbf{\color{green} TODO:}
  What is $\delta^-(v)$ and $\delta^+(v)$ in this context?
  Are they the sets of incoming and outgoing edges, or the total incoming and outgoing flow?
  In what way are we supposed to sum them?
\end{solution}

\begin{manualtheorem}{1.1}[Hoffman's circulation theorem]\label{thm:hoffman}
  Let $l, u : E \to \R$ be edge functions satisfying $l \leq u$.
  Then there exists a circulation $x$ in $D$ such that
  \begin{equation}
    l \leq x \leq u
  \end{equation}
  if and only if
  \begin{equation}
    \sum_{e \in \delta^-(S)} l(e) \leq \sum_{e \in \delta^+(S)} u(e)
    \qquad (S \subseteq V).
  \end{equation}
  Moreover, if $l$ and $u$ are integral (the function values are integral), then $x$ can be taken to be integral.
\end{manualtheorem}

\begin{exercise}
  Consider the problem treated in Hoffmans's circulation theorem (\cref{thm:hoffman}):
  decide if a circulation $x$ satisfying $l \leq x \leq u$ exists, and if so, find one.
  Show that this problem may be transformed into a flow problem with zero lower bounds, but with a given divergence.
  Hint:\@ apply the transformation $x'(e) = x(e) - l(e)$ ($e \in E$).
\end{exercise}

\begin{solution}
  Let $\tilde{u}(e) = u(e) - l(e)$ for each edge $e \in E$, and define the divergence vector $b : V \to \R$ as
  \begin{equation}
    b(v) = l(\delta^-(v)) - l(\delta^+(v)), \quad (v \in V).
  \end{equation}
  Consider the change of variables $x'(e) = x(e) - l(e)$ for each edge $e \in E$.
  Then, the constraints $l(e) \leq x(e) \leq u(e)$ can be rewritten as
  \begin{equation}
    0 \leq x'(e) \leq u(e) - l(e) = \tilde{u}(e) \quad (e \in E).
  \end{equation}
  Furthermore, the circulation condition $\diver_x(v) = x(\delta^+(v)) - x(\delta^-(v)) = 0$ for each vertex $v \in V$.
  Substituting $x = x' + l$ into this condition gives
  \begin{align*}
    x'(\delta^+(v)) - x'(\delta^-(v)) &= l(\delta^-(v)) - l(\delta^+(v)) \\
    \diver_{x'}(v) &= b(v).
  \end{align*}
  Therefore, the original problem of finding a circulation $x$ satisfying $l \leq x \leq u$ is equivalent to finding a flow $x'$ with zero lower bounds and upper bounds $\tilde{u}$, such that the divergence of $x'$ equals the vector $b$.
  Clearly, we must therefore require that $\tilde{u} \geq 0$, i.e., $u \geq l$.
\end{solution}

\begin{exercise}
  Show that when $x$ is an $st$-flow, then $\val(x)$ is equal to the total inflow to the sink $t$.
\end{exercise}

\begin{solution}
  We have that the value of an $st$-flow $x$ is defined as
  \begin{equation}
    \val(x) = \sum_{e \in \delta^+(s)} x(e).
  \end{equation}
  Assuming without loss of generality that there are no edges entering the source $s$, we additionally have that $\val(x) = \diver_x(s)$.
  By the flow conservation property, we know that for any vertex $v \in V \setminus \{s, t\}$, the divergence is zero, i.e., $\diver_x(v) = 0$.
  Therefore, summing the divergences over all vertices in $V$ gives
  \begin{align*}
    \sum_{v \in V} \diver_x(v) &= \diver_x(s) + \diver_x(t) + \sum_{v \in V \setminus \{s, t\}} \diver_x(v) \\
    &= \diver_x(s) + \diver_x(t) + 0 \\
    &= \val(x) + \diver_x(t).
  \end{align*}
  This gives us that
  \begin{equation}
    \val(x) = -\diver_x(t) = \sum_{e \in \delta^-(t)} x(e),
  \end{equation}
  which shows that the value of the flow is equal to the total inflow to the sink $t$.
\end{solution}

\begin{exercise}
  Explain why the Extreme Value Theorem gives the existence of a maximum flow.
  Also explain why this following from linear programming theory.
\end{exercise}

\begin{solution}
  The Extreme Value Theorem states that a continuous function defined on a compact set attains its maximum and minimum values.
  In the context of the maximum flow problem, the set of all feasible flows in a network can be represented as a convex polytope defined by linear constraints (capacity constraints and flow conservation constraints).
  This polytope is closed and bounded, hence compact.
  The objective function, which is the value of the flow, is a linear function and therefore continuous.
  By the Extreme Value Theorem, there exists a flow within this feasible region that maximizes the value of the flow.

  From the perspective of linear programming theory, the maximum flow problem can be formulated as a linear program where the objective is to maximize the total flow from the source to the sink subject to capacity and conservation constraints.
  Linear programming theory guarantees that if there exists an optimal solution to this linear program, it will occur at a vertex of the feasible region.
  Since the feasible region is non-empty and bounded due to capacity constraints, an optimal solution must exist, thus ensuring the existence of a maximum flow.
\end{solution}

\begin{manualtheorem}{1.4}[Max-flow min cut theorem]\label{thm:max-flow-min-cut}
  For any directed graph $D$, edge capacity function $c$, and distinct vertices $s, t$, the value of a maximum $st$-flow equals the minimum $st$-cut capacity, i.e.,
  \begin{equation}
    \max \{ \val(x) : x \text{ is }st\text{-flow} \}
    = \min \{ \capc_c(K) : K \text{ is } st\text{-cut} \}.
  \end{equation}
\end{manualtheorem}

\begin{manualtheorem}{1.5}\label{thm:st-flow-max}
  Let $x$ be an $st$-flow.
  Then $x$ is a maximum flow if and only if $D_x$ contains no $x$-augmenting paths.
\end{manualtheorem}

\begin{exercise}
  Use \cref{thm:st-flow-max} to give a proof of the max-flow min-cut theorem (\cref{thm:max-flow-min-cut}).
  Hint:\@ consider a maximum flow.
\end{exercise}

\begin{solution}
  Recall that an auxillary graph $D_x = (V, E_x)$ is given by
  \begin{equation}
    E_x =
    \{ e \in E : x(e) < u(e) \}
    \cup
    \{ e^{-1} : e \in E, \ l(e) < x(e) \},
  \end{equation}
  where $e^{-1}$ is the edge $e$ with reversed direction.
  The intuitive meaning of $D_x$ is that it contains all edges that can still carry more flow (the first set) and all edges where flow can be reduced (the second set).

  Let $x$ be an $st$-flow.
  For any $S \subseteq V$ with $s \in S$ and $t \notin S$, we have that
  \begin{equation}
    \val(x) = x(\delta^+(S)) - x(\delta^-(S)) \leq c(\delta^+(S)),
  \end{equation}
  since $x \leq c$ on every edge and $x \geq 0$.
  Taking the minimum over all such cuts gives
  \begin{equation}\label{eq:leq_min}
    \max \{\val(x) : x \text{ is } st\text{-flow} \} \leq \min \{ \capc_c(K) : K \text{ is } st\text{-cut} \}.
  \end{equation}
  Now let $x^*$ be a maximum $st$-flow.
  By \cref{thm:st-flow-max}, its residual graph $D_{x^*}$ contains no $x^*$-augmenting paths from $s$ to $t$.
  Let
  \begin{equation}
    S = \{ v \in V : v \text{ is reachable from } s \text{ in } D_{x^*} \}.
  \end{equation}
  Then $s \in S$ and $t \notin S$.

  We now claim that:
  \begin{align*}
    \text{(i)} \quad x^*(e) &= c(e) \quad \text{for every }e \in \delta^+(S), \\
    \text{(ii)} \quad x^*(e) &= 0 \qquad \text{for every }e \in \delta^-(S).
  \end{align*}
  To see (i), let $e = (u, v)\in \delta^+(S)$.
  If $x^*(e) < c(e)$, then $e$ would be in $E_{x^*}$, and hence $v$ would be reachable from $s$ in $D_{x^*}$, a contradiction
  To see (ii), let $e = (u, v) \in \delta^-(S)$.
  If $x^*(e) > 0$, then $e^{-1}$ would be in $E_{x^*}$, and hence $u$ would be reachable from $s$ via $u$ in $D_{x^*}$, again contradicting our choice of $S$.
  Thus, our claim holds.

  Therefore,
  \begin{equation}
    x^*(\delta^+(S)) = c(\delta^+(S))
    \quad\text{and}\quad
    x^*(\delta^-(S)) = 0,
  \end{equation}
  and hence
  \begin{equation}
    \val(x^*) = x^*(\delta^+(S)) - x^*(\delta^-(S)) = c(\delta^+(S)) = \capc_c(K).
  \end{equation}
  Combining this with \cref{eq:leq_min} gives the desired
  \begin{equation}
    \max \{\val(x) : x \text{ is } st\text{-flow} \}
    = \min \{ \capc_c(K) : K \text{ is } st\text{-cut} \},
  \end{equation}
  proving \cref{thm:max-flow-min-cut}.
\end{solution}

\begin{exercise}
  Choose an example of $D, s, t, c$ and find a maximum flow and minimum cut using the Ford-Fulkerson algorithm.
\end{exercise}

\begin{solution}
  Consider a graph with vertices $V = \{s, a, b, t\}$ and capacities as shown in \cref{fig:ford-fulkerson-example}.

  \begin{figure}[htbp]
    \centering
    \begin{tikzpicture}[>=Stealth, every node/.style={font=\small, text=axis_text}]
  % Styles
  \tikzset{
    vertex/.style={
      circle, draw=axis_line, semithick, fill=white,
      minimum size=18pt, inner sep=0pt
    },
    source/.style={vertex, draw=my_blue, fill=my_blue!12},
    sink/.style={vertex, draw=my_red,  fill=my_red!12},
    edge/.style={->, semithick, draw=axis_line},
    capacity/.style={midway, inner sep=1.5pt, rounded corners=1pt,
    text=axis_text}
  }

  % Nodes
  \node[source] (s) at (0, 0) {$s$};
  \node[vertex] (a) at (3, 1) {$a$};
  \node[vertex] (b) at (3, -1) {$b$};
  \node[sink]   (t) at (6, 0) {$t$};

  % Edges + capacities
  \draw[edge] (s) -- node[capacity, above] {$2$} (a);
  \draw[edge] (s) -- node[capacity, below] {$1$} (b);
  \draw[edge] (a) to[bend right=15] node[capacity, left] {$1$} (b);
  \draw[edge] (a) -- node[capacity, above] {$2$} (t);
  \draw[edge] (b) -- node[capacity, below] {$2$} (t);
\end{tikzpicture}

    \caption{Graph for Ford-Fulkerson example\label{fig:ford-fulkerson-example}}
  \end{figure}

  We start with zero flow on all edges, $x = 0$.
  Initially, we then have that $D_x = D$ as
  \begin{equation}
    E_x =
    \{ e \in E : x(e) < u(e) \}
    \cup
    \{ e^{-1} : e \in E, \ l(e) < x(e) \}
    = E \cup \emptyset = E.
  \end{equation}
  We then find an $x$-augmenting path from $s$ to $t$ by applying BFS. % chktex 13
  As $V_0 = \{s\}$, we add $a$ and $b$ to get $V_1 = \{a, b\}$, and initially discover $t$ in $V_2$ via $a$ (assuming we are going alphabetically).
  This gives an initial augmenting path $P_1 = (s, a, t)$.

  We then compute the bottleneck by taking the minimum of $c(e) - x(e)$ over all edges $e$ in $P_1^+$ and $x(e)$ over all edges $e$ in $P_1^-$.
  As $P_1$ only contains forward edges, we have
  \begin{equation}
    \varepsilon = \min \{ c(s, a) - x(s, a), \ c(a, t) - x(a, t) \} = \min \{ 2 - 0, \ 2 - 0 \} = 2.
  \end{equation}
  This gives us the updated flow $x'$ with $x'(s, a) = x'(a, t) = 2$ and $x'(e) = x(e) = 0$ for all other edges.

  In the next iteration, we construct $D_{x'}$.
  The edges in $E_{x'}$ are given by
  \begin{align*}
    E_{x'} &=
    \{ e \in E : x'(e) < u(e) \}
    \cup
    \{ e^{-1} : e \in E, \ l(e) < x'(e) \} \\
    &= \{ (s, b), (a, b), (b, t) \} \cup \{ (a, s), (t, a) \} \\
    &= \{ (s, b), (a, b), (b, t), (a, s), (t, a) \},
  \end{align*}
  illustrated in \cref{fig:ford-fulkerson-iteration-2}.

  \begin{figure}[htbp]
    \centering
    \input{network_flows/second_iter.tex}
    \caption{
      Residual graph $D_{x'}$ after first iteration.
      The dashed red edges are backward edges, and the edges are labeled with the current flow/capacity.\label{fig:ford-fulkerson-iteration-2}
    }
  \end{figure}

  Applying BFS again, we start with $V_0 = \{s\}$, add $b$ to get $V_1 = \{b\}$, and then discover $t$ in $V_2$ via $b$.
  This gives us the augmenting path $P_2 = (s, b, t)$.
  Again, we only have forward edges, so the bottleneck is
  \begin{equation}
    \varepsilon = \min \{ c(s, b) - x'(s, b), \ c(b, t) - x'(b, t) \} = \min \{ 1 - 0, \ 2 - 0 \} = 1.
  \end{equation}
  This gives us the updated flow $x''$ with $x''(s, a) = x''(a, t) = 2$, $x''(s, b) = x''(b, t) = 1$, and $x''(e) = 0$ for all other edges.

  In the next iteration, we construct $D_{x''}$.
  The edges in $E_{x''}$ are given by
  \begin{align*}
    E_{x''}
    &= \{ (a, b), (b, t) \} \cup \{ (a, s), (t, a), (b, s), (t, b) \} \\
    &= \{ (a, b), (b, t), (a, s), (t, a), (b, s), (t, b) \},
  \end{align*}
  illustrated in \cref{fig:ford-fulkerson-final}.

  \begin{figure}[htbp]
    \centering
    \begin{tikzpicture}[>=Stealth, every node/.style={font=\small, text=axis_text}]
  % Styles
  \tikzset{
    vertex/.style={
      circle, draw=axis_line, semithick, fill=white,
      minimum size=18pt, inner sep=0pt
    },
    source/.style={vertex, draw=my_blue, fill=my_blue!12},
    sink/.style={vertex, draw=my_red,  fill=my_red!12},
    edgeF/.style={->, semithick, draw=axis_line},
    edgeB/.style={->, semithick, draw=my_red, dashed},
    flowlab/.style={midway, inner sep=1.5pt, text=axis_text},
  }

  % Nodes
  \node[source] (s) at (0, 0) {$s$};
  \node[vertex] (a) at (3, 1) {$a$};
  \node[vertex] (b) at (3, -1) {$b$};
  \node[sink]   (t) at (6, 0) {$t$};

  % Current flow x/c on the original arcs
  % Highlight forward arcs used in the augmenting path
  \draw[edgeF] (a) to[bend right=15] node[flowlab, right] {$0/1$} (b);
  \draw[edgeF] (b) -- node[flowlab, below] {$1/2$} (t);

  % Highlight backward arcs used in the augmenting path
  \draw[edgeB] (a) -- node[flowlab, above] {$2/2$} (s);
  \draw[edgeB] (t) -- node[flowlab, above] {$2/2$} (a);
  \draw[edgeB] (b) -- node[flowlab, below] {$1/1$} (s);
  \draw[edgeB] (t) -- (b);
\end{tikzpicture}
    \caption{
      Residual graph $D_{x''}$ after second iteration.
      Note that both $(b, t)$ and its reverse $(t, b)$ are present in the residual graph.\label{fig:ford-fulkerson-final}
    }
  \end{figure}

  Starting now with $V_0 = \{ s \}$, we see that there are no outgoing edges from $s$ in $D_{x''}$, and hence we cannot reach $t$.
  Therefore, there are no more $x''$-augmenting paths from $s$ to $t$, and the algorithm terminates.
  The final flow $x''$ is a maximum flow with value
  \begin{equation}
    \val(x'') = x''(\delta^+(s)) = x''(s, a) + x''(s, b) = 2 + 1 = 3.
  \end{equation}
\end{solution}

\begin{exercise}
  How can we find a flow $x$ with a given divergence, say $\diver_x = b$, and satisfying capacity constraints $0 \leq x \leq c$?
  This is an important problem and it can be solved by transforming it into a maximum flow problem (for which several extremely fast a;gorithms exist).
  The construction is very similar to the one we gave in the proof of Theorem~1.2.
  Let $D = (V, E)$ be the given graph.
  Define $V^- = \{ v \in V : b(v) < 0 \}$ and $V^+ = \{ v \in V : b(v) > 0 \}$.
  Construct a graph $D' = (V', E')$ with vertex set $V' = V \cup \{s, t\}$, so we add \emph{two} new vertices $s$ and $t$.
  Let the edge set of $D'$ consist of (i) each edge $e \in E$, (ii) an edge $(s, v)$ for each $v \in V^+$, and (iii) an edge $(v, t)$ for each $v \in V^-$.
  Define a capacity function $c' : E' \to \R$ by $c'(s, v) = b(v)$ for each $v \in V^+$, $c'(v, t) = -b(v)$ for each $v \in V^-$, and $c'(e) = c(e)$ for each $e \in E$.

  The questions are then:
  \begin{enumerate}[label = (\alph*)]
    \item Show that the maximum value of an $st$-flow is at most $M \coloneq \sum_{v \in V^+} b(v)$.
    \item Show that a flow $x$ in $D$ satisfying $\diver_x = b$ and $0 \leq x \leq c$ exists if and only if the value of a maximum $st$-flow in $D'$ equals $M$ (defined in (a)).
    \item How do you find the desired $x$ from this maximum flow?
  \end{enumerate}
\end{exercise}

\begin{solution}
  (a) Let $x'$ be any $st$-flow in $D'$.
  Then, by the flow conservation property, we have that
  \begin{equation}
    \val(x') = \sum_{e \in \delta^+(s)} x'(e) - \sum_{e \in \delta^-(s)} x'(e) = \sum_{e \in \delta^+(s)} x'(e),
  \end{equation}
  since there are no edges entering $s$.
  Furthermore, for each $v \in V^+$, we have that $x'(s, v) \leq c'(s, v) = b(v)$.
  Therefore,
  \begin{equation}
    \val(x') = \sum_{v \in V^+} x'(s, v) \leq \sum_{v \in V^+} b(v) = M.
  \end{equation}

  (b) Suppose first that there exists a flow $x$ in $D$ satisfying $\diver_x = b$ and $0 \leq x \leq c$.
  We can then construct an $st$-flow $x'$ in $D'$ by setting $x'(e) = x(e)$ for each edge $e \in E$, $x'(s, v) = b(v)$ for each $v \in V^+$, and $x'(v, t) = -b(v)$ for each $v \in V^-$.
  It is straightforward to verify that $x'$ satisfies the capacity constraints in $D'$ and that the flow conservation property holds at all vertices in $V$.
  The value of this flow is
  \begin{equation}
    \val(x') = \sum_{v \in V^+} x'(s, v) = \sum_{v \in V^+} b(v) = M.
  \end{equation}

  Conversely, suppose that the value of a maximum $st$-flow in $D'$ equals $M$.
  Let $x'$ be such a maximum flow.
  We can then define a flow $x$ in $D$ by setting $x(e) = x'(e)$ for each edge $e \in E$.
  Again, it is straightforward to verify that $x$ satisfies the capacity constraints in $D$.
  To show that $\diver_x = b$, we consider each vertex $v \in V$.
  If $v \in V^+$, then the flow conservation property at $v$ in $D'$ gives
  \begin{equation}
    \diver_x(v) = \sum_{e \in \delta^+(v)} x(e) - \sum_{e \in \delta^-(v)} x(e) = x'(s, v) = b(v).
  \end{equation}
  If $v \in V^-$, then similarly we have
  \begin{equation}
    \diver_x(v) = \sum_{e \in \delta^+(v)} x(e) - \sum_{e \in \delta^-(v)} x(e) = -x'(v, t) = b(v).
  \end{equation}
  Finally, if $v \in V \setminus (V^+ \cup V^-)$, then the flow conservation property at $v$ in $D'$ gives
  \begin{equation}
    \diver_x(v) = \sum_{e \in \delta^+(v)} x(e) - \sum_{e \in \delta^-(v)} x(e) = 0 = b(v).
  \end{equation}
  Thus, we have shown that $\diver_x = b$.
\end{solution}
