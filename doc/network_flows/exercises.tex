\section{Network flows and combinatorial matrix theory}

\begin{exercise}
  Let $x$ be a flow in a a graph $D$. Show that $\sum_{v \in V}
  \diver_x(v) = 0$.
\end{exercise}

\begin{solution}
  We have the definition of $\diver_x : V \to \R$ as
  \begin{equation}
    \diver_x(v) = \sum_{e \in \delta^+(v)} x_e - \sum_{e \in \delta^-(v)} x_e,
  \end{equation}
  where $\delta^+(v)$ and $\delta^-(v)$ are the sets of outgoing and
  incoming edges of vertex $v$, respectively.
  Thus, for a graph $D = (V, E)$ we have
  \begin{align*}
    \sum_{v \in V} \diver_x(v)
    &= \sum_{v \in V} \left( \sum_{e \in \delta^+(v)} x_e - \sum_{e
    \in \delta^-(v)} x_e \right) \\
    &= \sum_{v \in V} \sum_{e \in \delta^+(v)} x_e - \sum_{v \in V}
    \sum_{e \in \delta^-(v)} x_e \\
    &= \sum_{e \in E} x_e - \sum_{e \in E} x_e \\
    &= 0,
  \end{align*}
  as each edge contributes exactly once to the sum of outgoing flows
  and exactly once to the sum of incoming flows.
  Therefore, we conclude that $\sum_{v \in V} \diver_x(v) = 0$.
\end{solution}

\begin{exercise}
  Let $x$ be a circulation in a graph $D = (V, E)$ and let $S \subseteq V$.
  Prove that $\sum_{v \in S} \delta^-(v) = \sum_{v \in S} \delta^+(v)$.
  (Hint:\@ sum the flow balance equations for vertices in S.)
\end{exercise}

\begin{solution}
  \textbf{\color{green} TODO:}
  What is $\delta^-(v)$ and $\delta^+(v)$ in this context?
  Are they the sets of incoming and outgoing edges, or the total
  incoming and outgoing flow?
  In what way are we supposed to sum them?
\end{solution}

\begin{manualtheorem}{1.1}[Hoffman's circulation theorem]\label{thm:hoffman}
  Let $l, u : E \to \R$ be edge functions satisfying $l \leq u$.
  Then there exists a circulation $x$ in $D$ such that
  \begin{equation}
    l \leq x \leq u
  \end{equation}
  if and only if
  \begin{equation}
    \sum_{e \in \delta^-(S)} l(e) \leq \sum_{e \in \delta^+(S)} u(e)
    \qquad (S \subseteq V).
  \end{equation}
  Moreover, if $l$ and $u$ are integral (the function values are
  integral), then $x$ can be taken to be integral.
\end{manualtheorem}

\begin{exercise}
  Consider the problem treated in Hoffmans's circulation theorem
  (\cref{thm:hoffman}):
  decide if a circulation $x$ satisfying $l \leq x \leq u$ exists,
  and if so, find one.
  Show that this problem may be transformed into a flow problem with
  zero lower bounds, but with a given divergence.
  Hint:\@ apply the transformation $x'(e) = x(e) - l(e)$ ($e \in E$).
\end{exercise}

\begin{solution}
  Let $\tilde{u}(e) = u(e) - l(e)$ for each edge $e \in E$, and
  define the divergence vector $b : V \to \R$ as
  \begin{equation}
    b(v) = l(\delta^-(v)) - l(\delta^+(v)), \quad (v \in V).
  \end{equation}
  Consider the change of variables $x'(e) = x(e) - l(e)$ for each
  edge $e \in E$.
  Then, the constraints $l(e) \leq x(e) \leq u(e)$ can be rewritten as
  \begin{equation}
    0 \leq x'(e) \leq u(e) - l(e) = \tilde{u}(e) \quad (e \in E).
  \end{equation}
  Furthermore, the circulation condition $\diver_x(v) =
  x(\delta^+(v)) - x(\delta^-(v)) = 0$ for each vertex $v \in V$.
  Substituting $x = x' + l$ into this condition gives
  \begin{align*}
    x'(\delta^+(v)) - x'(\delta^-(v)) &= l(\delta^-(v)) - l(\delta^+(v)) \\
    \diver_{x'}(v) &= b(v).
  \end{align*}
  Therefore, the original problem of finding a circulation $x$
  satisfying $l \leq x \leq u$ is equivalent to finding a flow $x'$
  with zero lower bounds and upper bounds $\tilde{u}$, such that the
  divergence of $x'$ equals the vector $b$.
  Clearly, we must therefore require that $\tilde{u} \geq 0$, i.e., $u \geq l$.
\end{solution}

\begin{exercise}
  Show that when $x$ is an $st$-flow, then $\val(x)$ is equal to the
  total inflow to the sink $t$.
\end{exercise}

\begin{solution}
  We have that the value of an $st$-flow $x$ is defined as
  \begin{equation}
    \val(x) = \sum_{e \in \delta^+(s)} x(e).
  \end{equation}
  Assuming without loss of generality that there are no edges
  entering the source $s$, we additionally have that $\val(x) = \diver_x(s)$.
  By the flow conservation property, we know that for any vertex $v
  \in V \setminus \{s, t\}$, the divergence is zero, i.e., $\diver_x(v) = 0$.
  Therefore, summing the divergences over all vertices in $V$ gives
  \begin{align*}
    \sum_{v \in V} \diver_x(v) &= \diver_x(s) + \diver_x(t) + \sum_{v
    \in V \setminus \{s, t\}} \diver_x(v) \\
    &= \diver_x(s) + \diver_x(t) + 0 \\
    &= \val(x) + \diver_x(t).
  \end{align*}
  This gives us that
  \begin{equation}
    \val(x) = -\diver_x(t) = \sum_{e \in \delta^-(t)} x(e),
  \end{equation}
  which shows that the value of the flow is equal to the total inflow
  to the sink $t$.
\end{solution}

\begin{exercise}
  Explain why the Extreme Value Theorem gives the existence of a maximum flow.
  Also explain why this following from linear programming theory.
\end{exercise}

\begin{solution}
  The Extreme Value Theorem states that a continuous function defined
  on a compact set attains its maximum and minimum values.
  In the context of the maximum flow problem, the set of all feasible
  flows in a network can be represented as a convex polytope defined
  by linear constraints (capacity constraints and flow conservation
  constraints).
  This polytope is closed and bounded, hence compact.
  The objective function, which is the value of the flow, is a linear
  function and therefore continuous.
  By the Extreme Value Theorem, there exists a flow within this
  feasible region that maximizes the value of the flow.

  From the perspective of linear programming theory, the maximum flow
  problem can be formulated as a linear program where the objective
  is to maximize the total flow from the source to the sink subject
  to capacity and conservation constraints.
  Linear programming theory guarantees that if there exists an
  optimal solution to this linear program, it will occur at a vertex
  of the feasible region.
  Since the feasible region is non-empty and bounded due to capacity
  constraints, an optimal solution must exist, thus ensuring the
  existence of a maximum flow.
\end{solution}

\begin{manualtheorem}{1.4}[Max-flow min cut theorem]\label{thm:max-flow-min-cut}
  For any directed graph $D$, edge capacity function $c$, and
  distinct vertices $s, t$, the value of a maximum $st$-flow equals
  the minimum $st$-cut capacity, i.e.,
  \begin{equation}
    \max \{ \val(x) : x \text{ is }st\text{-flow} \}
    = \min \{ \capc_c(K) : K \text{ is } st\text{-cut} \}.
  \end{equation}
\end{manualtheorem}

\begin{manualtheorem}{1.5}\label{thm:st-flow-max}
  Let $x$ be an $st$-flow.
  Then $x$ is a maximum flow if and only if $D_x$ contains no
  $x$-augmenting paths.
\end{manualtheorem}

\begin{exercise}
  Use \cref{thm:st-flow-max} to give a proof of the max-flow min-cut
  theorem (\cref{thm:max-flow-min-cut}).
  Hint:\@ consider a maximum flow.
\end{exercise}

\begin{solution}
  Recall that an auxiliary graph $D_x = (V, E_x)$ is given by
  \begin{equation}
    E_x =
    \{ e \in E : x(e) < u(e) \}
    \cup
    \{ e^{-1} : e \in E, \ l(e) < x(e) \},
  \end{equation}
  where $e^{-1}$ is the edge $e$ with reversed direction.
  The intuitive meaning of $D_x$ is that it contains all edges that
  can still carry more flow (the first set) and all edges where flow
  can be reduced (the second set).

  Let $x$ be an $st$-flow.
  For any $S \subseteq V$ with $s \in S$ and $t \notin S$, we have that
  \begin{equation}
    \val(x) = x(\delta^+(S)) - x(\delta^-(S)) \leq c(\delta^+(S)),
  \end{equation}
  since $x \leq c$ on every edge and $x \geq 0$.
  Taking the minimum over all such cuts gives
  \begin{equation}\label{eq:leq_min}
    \max \{\val(x) : x \text{ is } st\text{-flow} \} \leq \min \{
    \capc_c(K) : K \text{ is } st\text{-cut} \}.
  \end{equation}
  Now let $x^*$ be a maximum $st$-flow.
  By \cref{thm:st-flow-max}, its residual graph $D_{x^*}$ contains no
  $x^*$-augmenting paths from $s$ to $t$.
  Let
  \begin{equation}
    S = \{ v \in V : v \text{ is reachable from } s \text{ in } D_{x^*} \}.
  \end{equation}
  Then $s \in S$ and $t \notin S$.

  We now claim that:
  \begin{align*}
    \text{(i)} \quad x^*(e) &= c(e) \quad \text{for every }e \in \delta^+(S), \\
    \text{(ii)} \quad x^*(e) &= 0 \qquad \text{for every }e \in \delta^-(S).
  \end{align*}
  To see (i), let $e = (u, v)\in \delta^+(S)$.
  If $x^*(e) < c(e)$, then $e$ would be in $E_{x^*}$, and hence $v$
  would be reachable from $s$ in $D_{x^*}$, a contradiction
  To see (ii), let $e = (u, v) \in \delta^-(S)$.
  If $x^*(e) > 0$, then $e^{-1}$ would be in $E_{x^*}$, and hence $u$
  would be reachable from $s$ via $u$ in $D_{x^*}$, again
  contradicting our choice of $S$.
  Thus, our claim holds.

  Therefore,
  \begin{equation}
    x^*(\delta^+(S)) = c(\delta^+(S))
    \quad\text{and}\quad
    x^*(\delta^-(S)) = 0,
  \end{equation}
  and hence
  \begin{equation}
    \val(x^*) = x^*(\delta^+(S)) - x^*(\delta^-(S)) = c(\delta^+(S))
    = \capc_c(K).
  \end{equation}
  Combining this with \cref{eq:leq_min} gives the desired
  \begin{equation}
    \max \{\val(x) : x \text{ is } st\text{-flow} \}
    = \min \{ \capc_c(K) : K \text{ is } st\text{-cut} \},
  \end{equation}
  proving \cref{thm:max-flow-min-cut}.
\end{solution}

\begin{exercise}
  Choose an example of $D, s, t, c$ and find a maximum flow and
  minimum cut using the Ford-Fulkerson algorithm.
\end{exercise}

\begin{solution}
  Consider a graph with vertices $V = \{s, a, b, t\}$ and capacities
  as shown in \cref{fig:ford-fulkerson-example}.

  \begin{figure}[htbp]
    \centering
    \begin{tikzpicture}[>=Stealth, every node/.style={font=\small, text=axis_text}]
  % Styles
  \tikzset{
    vertex/.style={
      circle, draw=axis_line, semithick, fill=white,
      minimum size=18pt, inner sep=0pt
    },
    source/.style={vertex, draw=my_blue, fill=my_blue!12},
    sink/.style={vertex, draw=my_red,  fill=my_red!12},
    edge/.style={->, semithick, draw=axis_line},
    capacity/.style={midway, inner sep=1.5pt, rounded corners=1pt,
    text=axis_text}
  }

  % Nodes
  \node[source] (s) at (0, 0) {$s$};
  \node[vertex] (a) at (3, 1) {$a$};
  \node[vertex] (b) at (3, -1) {$b$};
  \node[sink]   (t) at (6, 0) {$t$};

  % Edges + capacities
  \draw[edge] (s) -- node[capacity, above] {$2$} (a);
  \draw[edge] (s) -- node[capacity, below] {$1$} (b);
  \draw[edge] (a) to[bend right=15] node[capacity, left] {$1$} (b);
  \draw[edge] (a) -- node[capacity, above] {$2$} (t);
  \draw[edge] (b) -- node[capacity, below] {$2$} (t);
\end{tikzpicture}

    \caption{Graph for Ford-Fulkerson example\label{fig:ford-fulkerson-example}}
  \end{figure}

  We start with zero flow on all edges, $x = 0$.
  Initially, we then have that $D_x = D$ as
  \begin{equation}
    E_x =
    \{ e \in E : x(e) < u(e) \}
    \cup
    \{ e^{-1} : e \in E, \ l(e) < x(e) \}
    = E \cup \emptyset = E.
  \end{equation}
  We then find an $x$-augmenting path from $s$ to $t$ by applying
  BFS. % chktex 13
  As $V_0 = \{s\}$, we add $a$ and $b$ to get $V_1 = \{a, b\}$, and
  initially discover $t$ in $V_2$ via $a$ (assuming we are going
  alphabetically).
  This gives an initial augmenting path $P_1 = (s, a, t)$.

  We then compute the bottleneck by taking the minimum of $c(e) -
  x(e)$ over all edges $e$ in $P_1^+$ and $x(e)$ over all edges $e$ in $P_1^-$.
  As $P_1$ only contains forward edges, we have
  \begin{equation}
    \varepsilon = \min \{ c(s, a) - x(s, a), \ c(a, t) - x(a, t) \} =
    \min \{ 2 - 0, \ 2 - 0 \} = 2.
  \end{equation}
  This gives us the updated flow $x'$ with $x'(s, a) = x'(a, t) = 2$
  and $x'(e) = x(e) = 0$ for all other edges.

  In the next iteration, we construct $D_{x'}$.
  The edges in $E_{x'}$ are given by
  \begin{align*}
    E_{x'} &=
    \{ e \in E : x'(e) < u(e) \}
    \cup
    \{ e^{-1} : e \in E, \ l(e) < x'(e) \} \\
    &= \{ (s, b), (a, b), (b, t) \} \cup \{ (a, s), (t, a) \} \\
    &= \{ (s, b), (a, b), (b, t), (a, s), (t, a) \},
  \end{align*}
  illustrated in \cref{fig:ford-fulkerson-iteration-2}.

  \begin{figure}[htbp]
    \centering
    \input{network_flows/second_iter.tex}
    \caption{
      Residual graph $D_{x'}$ after first iteration.
      The dashed red edges are backward edges, and the edges are
      labelled with the current
      flow/capacity.\label{fig:ford-fulkerson-iteration-2}
    }
  \end{figure}

  Applying BFS again, we start with $V_0 = \{s\}$, add $b$ to get
  $V_1 = \{b\}$, and then discover $t$ in $V_2$ via $b$.
  This gives us the augmenting path $P_2 = (s, b, t)$.
  Again, we only have forward edges, so the bottleneck is
  \begin{equation}
    \varepsilon = \min \{ c(s, b) - x'(s, b), \ c(b, t) - x'(b, t) \}
    = \min \{ 1 - 0, \ 2 - 0 \} = 1.
  \end{equation}
  This gives us the updated flow $x''$ with $x''(s, a) = x''(a, t) =
  2$, $x''(s, b) = x''(b, t) = 1$, and $x''(e) = 0$ for all other edges.

  In the next iteration, we construct $D_{x''}$.
  The edges in $E_{x''}$ are given by
  \begin{align*}
    E_{x''}
    &= \{ (a, b), (b, t) \} \cup \{ (a, s), (t, a), (b, s), (t, b) \} \\
    &= \{ (a, b), (b, t), (a, s), (t, a), (b, s), (t, b) \},
  \end{align*}
  illustrated in \cref{fig:ford-fulkerson-final}.

  \begin{figure}[htbp]
    \centering
    \begin{tikzpicture}[>=Stealth, every node/.style={font=\small, text=axis_text}]
  % Styles
  \tikzset{
    vertex/.style={
      circle, draw=axis_line, semithick, fill=white,
      minimum size=18pt, inner sep=0pt
    },
    source/.style={vertex, draw=my_blue, fill=my_blue!12},
    sink/.style={vertex, draw=my_red,  fill=my_red!12},
    edgeF/.style={->, semithick, draw=axis_line},
    edgeB/.style={->, semithick, draw=my_red, dashed},
    flowlab/.style={midway, inner sep=1.5pt, text=axis_text},
  }

  % Nodes
  \node[source] (s) at (0, 0) {$s$};
  \node[vertex] (a) at (3, 1) {$a$};
  \node[vertex] (b) at (3, -1) {$b$};
  \node[sink]   (t) at (6, 0) {$t$};

  % Current flow x/c on the original arcs
  % Highlight forward arcs used in the augmenting path
  \draw[edgeF] (a) to[bend right=15] node[flowlab, right] {$0/1$} (b);
  \draw[edgeF] (b) -- node[flowlab, below] {$1/2$} (t);

  % Highlight backward arcs used in the augmenting path
  \draw[edgeB] (a) -- node[flowlab, above] {$2/2$} (s);
  \draw[edgeB] (t) -- node[flowlab, above] {$2/2$} (a);
  \draw[edgeB] (b) -- node[flowlab, below] {$1/1$} (s);
  \draw[edgeB] (t) -- (b);
\end{tikzpicture}
    \caption{
      Residual graph $D_{x''}$ after second iteration.
      Note that both $(b, t)$ and its reverse $(t, b)$ are present in
      the residual graph.\label{fig:ford-fulkerson-final}
    }
  \end{figure}

  Starting now with $V_0 = \{ s \}$, we see that there are no
  outgoing edges from $s$ in $D_{x''}$, and hence we cannot reach $t$.
  Therefore, there are no more $x''$-augmenting paths from $s$ to
  $t$, and the algorithm terminates.
  The final flow $x''$ is a maximum flow with value
  \begin{equation}
    \val(x'') = x''(\delta^+(s)) = x''(s, a) + x''(s, b) = 2 + 1 = 3.
  \end{equation}
\end{solution}

\begin{exercise}
  How can we find a flow $x$ with a given divergence, say $\diver_x =
  b$, and satisfying capacity constraints $0 \leq x \leq c$?
  This is an important problem and it can be solved by transforming
  it into a maximum flow problem (for which several extremely fast
  algorithms exist).
  The construction is very similar to the one we gave in the proof of
  Theorem~1.2.
  Let $D = (V, E)$ be the given graph.
  Define $V^- = \{ v \in V : b(v) < 0 \}$ and $V^+ = \{ v \in V : b(v) > 0 \}$.
  Construct a graph $D' = (V', E')$ with vertex set $V' = V \cup \{s,
  t\}$, so we add \emph{two} new vertices $s$ and $t$.
  Let the edge set of $D'$ consist of (i) each edge $e \in E$, (ii)
  an edge $(s, v)$ for each $v \in V^+$, and (iii) an edge $(v, t)$
  for each $v \in V^-$.
  Define a capacity function $c' : E' \to \R$ by $c'(s, v) = b(v)$
  for each $v \in V^+$, $c'(v, t) = -b(v)$ for each $v \in V^-$, and
  $c'(e) = c(e)$ for each $e \in E$.

  The questions are then:
  \begin{enumerate}[label = (\alph*)]
    \item Show that the maximum value of an $st$-flow is at most $M
      \coloneq \sum_{v \in V^+} b(v)$.
    \item Show that a flow $x$ in $D$ satisfying $\diver_x = b$ and
      $0 \leq x \leq c$ exists if and only if the value of a maximum
      $st$-flow in $D'$ equals $M$ (defined in (a)).
    \item How do you find the desired $x$ from this maximum flow?
  \end{enumerate}
\end{exercise}

\begin{solution}
  (a) Let $x'$ be any $st$-flow in $D'$.
  Then, by the flow conservation property, we have that
  \begin{equation}
    \val(x') = \sum_{e \in \delta^+(s)} x'(e) - \sum_{e \in
    \delta^-(s)} x'(e) = \sum_{e \in \delta^+(s)} x'(e),
  \end{equation}
  since there are no edges entering $s$.
  Furthermore, for each $v \in V^+$, we have that $x'(s, v) \leq
  c'(s, v) = b(v)$.
  Therefore,
  \begin{equation}
    \val(x') = \sum_{v \in V^+} x'(s, v) \leq \sum_{v \in V^+} b(v) = M.
  \end{equation}

  (b) We prove the statement by showing both directions of the equivalence.

  (Exists a flow $x$ in $D$ satisfying $\diver_x = b$ and $0 \leq x
  \leq c$  $\implies$ the value of a maximum $st$-flow in $D'$ equals $M$) \\
  Assume we have a flow $x : E \to \R$ in $D$ satisfying $\diver_x =
  b$ and $0 \leq x \leq c$.
  We construct a new flow $x' : E' \to \R$ in $D'$ by
  \begin{equation}
    x'(e) =
    \begin{cases}
      x(e) & \text{if } e \in E, \\
      b(v) & \text{if } e = (s, v) \text{ for } v \in V^+, \\
      -b(v) & \text{if } e = (v, t) \text{ for } v \in V^-.
    \end{cases}
  \end{equation}
  We then get that the capacity constraints are satisfied in $D'$ as
  \begin{align*}
    x'(e) = x(e) \leq c(e) &= c'(e) & \text{for } e &\in E, \\
    x'(s, v) = b(v) &= c'(s, v) & \text{for } v &\in V^+, \\
    x'(v, t) = -b(v) &= c'(v, t) & \text{for } v &\in V^-,
  \end{align*}
  such that $0 \leq x' \leq c'$.

  For all $v \in V \setminus (V^+ \cup V^-)$, we did not add any new
  edges incident to $v$ in $D'$, so the flow conservation property
  holds at these vertices.
  For $v \in V^+$, we have that
  \begin{align*}
    x'(\delta^-(v)) + x'(s, v) &= x(\delta^+(v)) \\
    x(\delta^+(v)) - x'(\delta^-(v)) &= b(v),
  \end{align*}
  where we used that $\diver_x(v) = b(v)$ and $x'(s, v) = b(v)$.
  Similarly, for $v \in V^-$, we have that
  \begin{align*}
    x'(\delta^+(v)) + x'(v, t) &= x(\delta^-(v)) \\
    x(\delta^+(v)) - x'(\delta^-(v)) &= b(v).
  \end{align*}
  Then, as the only edges leaving $s$ are $(s, v)$ for $v \in V^+$, we have that
  \begin{equation}
    \val(x') = \sum_{v \in V^+} x'(s, v) = \sum_{v \in V^+} b(v) = M.
  \end{equation}
  Thus, $x'$ is an $st$-flow in $D'$ with value $M$, which by part
  (a) must be a maximum flow.

  (Exists a flow $x$ in $D$ satisfying $\diver_x = b$ and $0 \leq x
  \leq c$  $\impliedby$ the value of a maximum $st$-flow in $D'$ equals $M$) \\
  Suppose now that $x'$ is a maximum $st$-flow in $D'$ with
  \begin{equation}
    \val(x') = M.
  \end{equation}
  Again, we have
  \begin{equation}
    \val(x') = \sum_{v \in V^+} x'(s, v) \leq \sum_{v \in V^+} c'(s,
    v) = \sum_{v \in V^+} b(v) = M,
  \end{equation}
  such that we must have equality throughout.
  As all terms are non-negative, this implies that
  \begin{equation}
    x'(s, v) = c'(s, v) = b(v) \quad \text{for all } v \in V^+,
  \end{equation}
  and similarly,
  \begin{equation}
    x'(v, t) = c'(v, t) = -b(v) \quad \text{for all } v \in V^-.
  \end{equation}

  Restricting $x'$ to the edges in $E$ gives a flow $x : E \to \R$ in
  $D$, satisfying $0 \leq x(e) = x'(e) \leq c'(e) = c(e)$ for all $e \in E$.
  To show that $\diver_x = b$, we consider each vertex $v \in V$.
  If $v \in V^+$, then the flow conservation property at $v$ in $D'$ gives
  \begin{align*}
    x'(\delta^-(v)) + x'(s, v) &= x'(\delta^+(v)) \\
    x(\delta^+(v)) - x(\delta^-(v)) &= b(v),
  \end{align*}
  where we used that $x'(s, v) = b(v)$ and $x'(e) = x(e)$ for all $e \in E$.
  We again get an equivalent result for $v \in V^-$.
  Finally, if $v \in V \setminus (V^+ \cup V^-)$, then the flow
  conservation property at $v$ in $D'$ gives
  \begin{equation}
    x'(\delta^+(v)) = x'(\delta^-(v)) \implies \diver_x(v) = 0 = b(v).
  \end{equation}
  Thus, we have shown that $\diver_x = b$.

  (c) Algorithmically, we find $x$ from a maximum flow by the following steps:
  \begin{enumerate}
    \item Construct $D' = (V', E')$ and $c'$ as described.
    \item Compute a maximum $st$-flow $x'$ in $D'$.
    \item Check its value:
      \begin{itemize}
        \item If $\val(x') < M$, then no flow $x$ in $D$ satisfying
          $\diver_x = b$ and $0 \leq x \leq c$ exists.
        \item If $\val(x') = M$, then restrict $x'$ to the edges in
          $E$ to obtain the desired flow $x$ in $D$.
      \end{itemize}
  \end{enumerate}
\end{solution}

\begin{manualtheorem}{1.6}[The integrality theorem]\label{thm:integrality}
  If the capacity function $c$ is integral (meaning each $c(e)$ is
  integral), then there is a maximum flow which is integral.
\end{manualtheorem}

\begin{exercise}
  Fill in the details of the proof of \cref{thm:integrality} (the
  integrality theorem).
\end{exercise}

\begin{solution}
  At each iteration of the Ford-Fulkerson algorithm, we find an
  $x$-augmenting path $P$ and compute the bottleneck $\varepsilon$.
  Initially we have $x = 0$, which is clearly integral.
  We then adjust the flow along $P$ by adding $\varepsilon$ to the
  flow on forward edges and subtracting $\varepsilon$ on backward
  edges, where $\varepsilon$ is computed as the minimum of $c(e) -
  x(e)$ over all forward edges $e$ in $P$ and $x(e)$ over all
  backward edges $e$ in $P$.
  Since $c$ is integral and $x$ is integral at the start of each
  iteration, both $c(e) - x(e)$ and $x(e)$ are integral for all edges $e$.
  Therefore, the minimum $\varepsilon$ is also integral.
  As a result, when we update the flow $x$ along the path $P$, the
  new flow values remain integral.
  This process continues until no more $x$-augmenting paths exist, at
  which point we have found a maximum flow.
  Since the flow remains integral throughout the algorithm, the final
  maximum flow is also integral.
\end{solution}

\begin{exercise}
  Consider the max-flow problem in the special case where $c(e) = 1$
  for each $e \in E$ (unit capacities).
  Choose an example of $D$, $s$ and $t$.
  Solve the max-flow problem.
  What can be said about the structure of your max-flow $x$?
  Prove, in general, that a max-flow may be represented by a set of
  \emph{edge-disjoint} $st$-paths (i.e., pairwise disjoint) and where
  there is unit flow on each of these paths.
\end{exercise}

\begin{solution}
  Intuitively, in the case of unit capacities, the maximum flow is
  equal to finding the number of distinct non-overlapping paths from $s$ to $t$.

  Along each path from $s$ to $t$, the flow can be at most 1 due to
  the unit capacity constraint on each edge.
  As the capacities are integral, by the integrality theorem
  (\cref{thm:integrality}), there exists a maximum flow that is integral.
  Let $x$ be such a maximum flow.
  Therefore, each edge $e$ carries either 0 or 1 unit of flow, and if
  an edge carries 1 unit of flow, it is fully utilized by a single
  path from $s$ to $t$.
  We can decompose the flow $x$ into a set of edge-disjoint
  $st$-paths by tracing the flow from $s$ to $t$.
  Starting from $s$, we follow edges with flow 1 until we reach $t$,
  marking these edges as used.
  We repeat this process until no more paths can be found.
  Since each edge can only be used once in this decomposition (due to
  the unit capacity), the resulting paths are edge-disjoint.
  Thus, we can represent the maximum flow $x$ as a collection of
  edge-disjoint $st$-paths, each carrying a unit flow.

  The number of such edge-disjoint paths corresponds to the value of
  the maximum flow.
\end{solution}

\begin{exercise}
  Prove Menger's theorem:
  \begin{manualtheorem}{}[Menger's theorem]
    The maximum number of edge-disjoint $st$-paths is equal to the
    minimum cardinality of an $st$-cut.
  \end{manualtheorem}
  \noindent
  This is a classical minmax theorem in graph theory.
  Hint:\@ Consider the previous exercise.
\end{exercise}

\begin{solution}
  For a graph $D = (V, E)$, assign each edge $e \in E$ a unit
  capacity, i.e., $c(e) = 1$.
  By the discussion in the previous exercise, the maximum number of
  edge-disjoint $st$-paths corresponds to the value of the maximum
  $st$-flow in this unit capacity network.
  According to the max-flow min-cut theorem
  (\cref{thm:max-flow-min-cut}), the value of the maximum $st$-flow
  is equal to the minimum capacity of an $st$-cut.
  In the case of unit capacities, the capacity of an $st$-cut is
  simply the number of edges in the cut.
  Therefore, the minimum cardinality of an $st$-cut is equal to the
  number of edge-disjoint $st$-paths.
\end{solution}

\begin{exercise}
  The \emph{matrix rounding problem} is the following problem.
  Given a real $m \times n$ matrix $A = [a_{ij}]$ with row sums $r_i
  = \sum_{j = 1}^{n} a_{ij}$ $(i \leq m)$ and column sums $s_j =
  \sum_{i = 1}^{m} a_{ij}$ $(j \leq n)$, round each entry $a_{ij}$ to
  either $\floor{a_{ij}}$ or $\ceil{a_{ij}}$ and also round each
  $r_i$ and $s_j$ either up or down, such that the new matrix
  $\bar{A}$ has row and columns sums equal to the corresponding rounded sums.
  For instance, consider
  \begin{equation}
    A =
    \begin{bmatrix}
      4.2 & 3.5 & 2.6 \\
      1.1 & 2.1 & 8.3 \\
      6.5 & 3.9 & 1.2
    \end{bmatrix}.
  \end{equation}
  Here, $(r_1, r_2, r_3) = (10.3, 11.5, 11.6)$ and $(s_1, s_2, s_3) =
  (11.8, 9.5, 12.1)$.
  Then we might start by rounding $a_{11}, a_{12}, a_{11}$ and $r_1$
  to 4, 4, 3 and 11, respectively.
  Go on, and try to solve the problem.
  The difficulty is that we have get the ``right'' column sums as well.
  The general matrix rounding problem may be represented as a flow
  problem in the following graph.

  Let $G = (V, E)$ be a graph with vertex set $v = \{ s, t, u_1,
  \ldots, u_m, v_1, \ldots, v_n \}$ and the following edges:
  \begin{equation}
    \begin{split}
      (s, u_i) & \text{ for } i \leq m, \\
      (v_j, t) & \text{ for } j \leq n, \\
      (u_i, v_j) & \text{ for } i \leq m, \ j \leq n, \\
      (t, s).
    \end{split}
  \end{equation}
  The idea is to represent the matrix entry $\bar{a}_{ij}$ by the
  flow $x(e)$ in the edge $e = (u_i, v_j)$.
  Moreover, the $i$th row sum corresponds to the flow in the edge
  $(s, u_i)$, and the $j$th column sum corresponds to the flow in the
  edge $(v_j, t)$.
  Draw the graph for the specific example above.
  Your task is to define a lower bound $l(e)$ and an upper bound
  $u(e)$ on the flow in each edge $e$ such that the matrix rounding
  problem becomes that of finding a circulation $x$ satisfying $l
  \leq x \leq u$.
\end{exercise}

\begin{solution}
  As an upper bound on the flow in the edge $e = (u_i, v_j)$, we set
  the upper bound to be $u(e) = \ceil{a_{ij}}$, and as a lower bound,
  we set $l(e) = \floor{a_{ij}}$.
  For the edge $e = (s, u_i)$, we set the upper bound to be $u(e) =
  \ceil{r_i}$ and the lower bound to be $l(e) = \floor{r_i}$.
  Similarly, for the edge $e = (v_j, t)$, we set the upper bound to
  be $u(e) = \ceil{s_j}$ and the lower bound to be $l(e) = \floor{s_j}$.
  Finally, we'd like to leave the final edge $e = (t, s)$ as
  unrestricted as possible, so we set $l(t, s) = 0$ and $u(t, s) =
  \sum_{i = 1}^{m} \ceil{r_i}$.
  With these definitions, finding a circulation $x$ satisfying $l
  \leq x \leq u$ corresponds to finding a rounded matrix $\bar{A}$
  where each entry $\bar{a}_{ij}$ is either $\floor{a_{ij}}$ or
  $\ceil{a_{ij}}$, and the row and column sums of $\bar{A}$ match the
  rounded sums of the original matrix $A$.
  The example graph is illustrated in \cref{fig:matrix-rounding}.

  \begin{figure}[htbp]
    \centering
    \usetikzlibrary{backgrounds}

\begin{tikzpicture}[>=Stealth, every node/.style={font=\small, text=axis_text}]
  \path[use as bounding box] (-0.5,-3.4) rectangle (8.5,2.4);

  % Styles
  \tikzset{
    vertex/.style={
      circle, draw=axis_line, semithick, fill=white,
      minimum size=18pt, inner sep=0pt
    },
    source/.style={vertex, draw=my_blue, fill=my_blue!12},
    sink/.style={vertex, draw=my_red,  fill=my_red!12},
    edge/.style={->, semithick, draw=axis_line},
    capacity/.style={
      midway,
      inner sep=2pt,            % more “air” around the text
      rounded corners=1pt,
      text=axis_text,
      fill=white,               % hide the line behind the label
      draw=none                 % no border; omit this to have a box border
    }
  }

  \def\ROWS{3}
  \def\COLS{3}
  \pgfkeys{/pgf/number format/.cd,fixed,fixed zerofill,precision=1}

  % Flattened matrix A in row-major order:
  % [ [4.2, 3.5, 2.6],
  %   [1.1, 2.1, 8.3],
  %   [6.5, 3.9, 1.2] ]
  \def\A{4.2,3.5,2.6,1.1,2.1,8.3,6.5,3.9,1.2}

  % Helper: A_{ij}
  \newcommand{\Aij}[2]{%
    \pgfmathtruncatemacro{\idx}{(#1-1)*\COLS + #2-1}%
    \pgfmathparse{{\A}[\idx]}%
    \pgfmathresult%
  }

  % Nodes
  \node[source] (s) at (0, 0) {$s$};
  \node[sink]   (t) at (8, 0) {$t$};

  % Row nodes
  \foreach \i in {1,...,\ROWS} {
    \node[vertex] (u\i) at (2, {(\ROWS - 1) - (\i - 1) * 2}) {$u_{\i}$};
  }

  % Column nodes
  \foreach \j in {1,...,\COLS} {
    \node[vertex] (v\j) at (6, {(\COLS - 1) - (\j - 1) * 2}) {$v_{\j}$};
  }

  % Edges from source to row nodes: row sums r_i = sum_j A_{ij}
  \foreach \i in {1,...,\ROWS} {
    \pgfmathsetmacro{\rowsum}{0}%
    \foreach \j in {1,...,\COLS} {
      \pgfmathtruncatemacro{\idx}{(\i-1)*\COLS + (\j-1)}%
      \pgfmathsetmacro{\rowsum}{\rowsum + {\A}[\idx]}%
      \xdef\rowsum{\rowsum}%
    }
    \draw[edge] (s) -- node[capacity, sloped] {\pgfmathprintnumber{\rowsum}} (u\i);
  }

  % Edges from column nodes to sink: column sums s_j = sum_i A_{ij}
  \foreach \j in {1,...,\COLS} {
    \pgfmathsetmacro{\colsum}{0}%
    \foreach \i in {1,...,\ROWS} {
      \pgfmathtruncatemacro{\idx}{(\i-1)*\COLS + (\j-1)}%
      \pgfmathsetmacro{\colsum}{\colsum + {\A}[\idx]}%
      \xdef\colsum{\colsum}%
    }
    \draw[edge] (v\j) -- node[capacity, sloped] {\pgfmathprintnumber{\colsum}} (t);
  }

  % Draw the edges in the center
  \foreach \i in {1,...,\ROWS} {
    \foreach \j in {1,...,\COLS} {
      \draw[edge] (u\i) -- (v\j);
    }
  }

  % Add the labels (avoid overlap)
  \foreach \i in {1,...,\ROWS} {
    \ifodd\i
    \def\labpos{0.3}
    \else
    \def\labpos{0.7}
    \fi
    \foreach \j in {1,...,\COLS} {
      \path (u\i) -- (v\j)
      node[capacity, pos=\labpos, sloped] {$\Aij{\i}{\j}$};
    }
  }

  % Final edge from sink to source, loop around in arc
  \draw[edge] (t) to[bend left=90, looseness=1.3] (s);
\end{tikzpicture}

    \caption{Graph representation of the example.\label{fig:matrix-rounding}}
  \end{figure}

  As each lower and upper bound is integral, by \cref{thm:hoffman},
  if a circulation exists, then there exists an integral circulation.
  This integral circulation then directly corresponds to a valid
  rounded matrix $\bar{A}$.
\end{solution}

\begin{exercise}
  Let $P$ be a polytope contained in the unit cube, so that each $x
  \in P$ satisfies $0 \leq x_i \leq 1$ ($i \leq n$).
  Prove that each $(0, 1)$-vector in $P$ is a vertex of $P$.
\end{exercise}

\begin{solution}
  Let $x \in P$ be a $(0, 1)$-vector, i.e.\ such that $x_i \in \{0,
  1\}$ for all $i \leq n$.
  In order to show that $x$ is a vertex of $P$, we will show that
  there is no way to express $x$ as a convex combination of two
  distinct points in $P$, i.e., there do not exist $y, z \in P$, $y
  \neq z$, and $\lambda \in (0, 1)$ such that
  \begin{equation}
    x = \lambda y + (1 - \lambda) z.
  \end{equation}

  Suppose, for the sake of contradiction, that such $y, z, \lambda$ exist.
  Then, for each coordinate $i \leq n$, we have
  \begin{equation}
    x_i = \lambda y_i + (1 - \lambda) z_i.
  \end{equation}
  For $x_i = 0$, we have that $\lambda y_i + (1 - \lambda) z_i = 0$.
  Since $\lambda \in (0, 1)$, and $y_i, z_i \geq 0$, this implies
  that $y_i = z_i = 0$, and hence equals $x_i$.
  Similarly, for $x_i = 1$, we have that $\lambda y_i + (1 - \lambda) z_i = 1$.
  Since $\lambda \in (0, 1)$, and $y_i, z_i \leq 1$, this implies
  that $y_i = z_i = 1$, and hence again equals $x_i$.
  Therefore, for all coordinates $i \leq n$, we have $y_i = z_i =
  x_i$, contradicting our assumption that $y \neq z$.
  Thus, $x$ cannot be expressed as a convex combination of two
  distinct points in $P$, and hence $x$ is a vertex of $P$.
\end{solution}

\begin{exercise}
  Show that every permutation matrix is a vertex of $\Omega_n$ and
  that these are precisely the integral matrices in $\Omega_n$.
  Hint:\@ Use the previous exercise.
\end{exercise}

\begin{solution}
  A permutation matrix is a matrix $P = [p_{ij}] \in \R^{n \times n}$
  where each row and each column has exactly one entry equal to 1 and
  all other entries are 0,
  i.e., $p_{ij} \in \{0, 1\}$ for all $i, j \leq n$.
  We then utilize the bijection between $\R^{n \times n}$ and
  $\R^{n^2}$ by mapping each matrix to a vector by stacking its rows,
  such that $P \mapsto x$ with
  \begin{equation}
    x_{(i - 1)n + j} = p_{ij} \quad \text{for } i, j \leq n.
  \end{equation}
  We have that each such $x$ is a $(0, 1)$-vector in $\R^{n^2}$,
  contained in the polytope defined by the constraints
  \begin{equation}
    \begin{split}
      \sum_{j = 1}^{n} x_{(i - 1)n + j} &= 1 \quad \text{for } i \leq n, \\
      \sum_{i = 1}^{n} x_{(i - 1)n + j} &= 1 \quad \text{for } j \leq n, \\
      0 \leq x_k &\leq 1 \quad \text{for } k \leq n^2,
    \end{split}
  \end{equation}
  which corresponds to a polytope contained in the unit cube.
  By the previous exercise, each such $(0, 1)$-vector $x$ is a vertex
  of this polytope, and hence each permutation matrix $P$ is a vertex
  of $\Omega_n$.
\end{solution}
