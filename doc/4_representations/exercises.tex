\section{Representation of convex sets}

\begin{exercise}
  Consider the polytope \(P \subset \R^2\) being the convex hull of the
  point \((0, 0)\), \((1, 0)\) and \((0, 1)\) (so \(P\) is a simplex
  in \(\R^2\)).
  \begin{enumerate}[label = (\emph{\roman*})]
    \item Find the unique face of \(P\) that contains the point \((1/3, 1/2)\).
    \item Find all faces of \(P\) that contain the point \((1/3, 2/3)\).
    \item Determine all the faces of \(P\).
  \end{enumerate}
\end{exercise}

\begin{solution}
  Beginning with (\emph{iii}), we note the the dimension \(0\) faces of
  \(P\) are the vertices.
  The dimension \(1\) faces are the edges between the vertices, and the
  dimension \(2\) face is \(P\) itself.
  (\emph{i}): The point \((1/3, 1/2)\) clearly does not lie on any of
  the edges or vertices, so the only face containing it is \(P\) itself.
  (\emph{ii}): The point \((1/3, 2/3)\) lies on the edge between
  \((0, 1)\) and \((1, 0)\) (\(\lambda = 1/3\)), so the faces containing it are
  this edge and \(P\) itself.
\end{solution}

\begin{exercise}
  Explain why an equivalent definition of face is obtained using the
  condition: if whenever \(x_1, x_2 \in C\) and \((1/2)(x_1 + x_2)
  \in F\), then \(x_1, x_2 \in F\).
\end{exercise}

\begin{solution}
  Let \(F\) be a face of \(C\).
  Then, by definition, if \(x_1, x_2 \in C\) and \(\lambda x_1 + (1 -
  \lambda) x_2 \in F\) for some \(\lambda \in (0, 1)\), then \(x_1, x_2 \in F\).
  In particular, this holds for \(\lambda = 1/2\), so the condition
  is satisfied.

  For the converse, assume that \(F\) satisfies the midpoint condition:
  whenever \(x_1, x_2 \in C\) and \((2)(x_1 + x_2) \in F\), then
  \(x_1, x_2 \in F\).
  We can essentially perform a binary search to find each point on
  the line segment between \(x_1\) and \(x_2\).
  More precisely, for a point \(y = \lambda x_1 + (1 - \lambda) x_2\)
  where \(\lambda \in (0, 1)\), we can express \(\lambda\) in its binary
  representation.
  This means that there is a sequence \(\{ \sigma_i (1/2)^i
  \}_{i=1}^\infty\) where \(\sigma_i \in \{ -1, 1 \}\), \(\sigma_1 =
  1/2\), such that
  \begin{equation}
    \lambda_n = \sum_{i=1}^n \sigma_i (1/2)^i,
  \end{equation}
  and \(\lambda_n \to \lambda\) as \(n \to \infty\).
  We can then define a sequence of points \(\{ y_n \}_{n=1}^\infty\)
  where \(y_n = \lambda_n x_1 + (1 - \lambda_n) x_2\).
  At each step \(n\), we are only considering a subinterval \([x_l^n,
  x_r^n] \subseteq [x_1, x_2]\) where \(y_n = \frac{1}{2}(x_l^n + x_r^n)\).
  By the midpoint condition, if \(y_n \in F\), then \(x_l^n, x_r^n \in F\).
  If \(\lambda > \lambda_n\), we set \(x_l^{n+1} = y_n\) and
  \(x_r^{n+1} = x_r^n\).
  If \(\lambda < \lambda_n\), we set \(x_l^{n+1} = x_l^n\) and
  \(x_r^{n+1} = y_n\).
  In either case, we have that \(y_{n+1} = \frac{1}{2}(x_l^{n+1} + x_r^{n+1})\).
  Since \(y_1 \in F\), we have that \(x_l^1, x_r^1 \in F\).
  By induction, we have that \(x_l^n, x_r^n \in F\) for all \(n\).
  Since \(y_n\) is the midpoint of \(x_l^n\) and \(x_r^n\), we have that
  \(y_n \in F\) for all \(n\).
  Finally, since \(y_n \to y\) as \(n \to \infty\) and \(F\) is closed, we
  have that \(y \in F\).
  Thus, \(F\) is a face of \(C\).
\end{solution}

\begin{manualprop}{4.1.2}[Face of face]
  Let \(C \subseteq \R^n\) be a non-empty convex set.
  Let \(F_1\) be a face of \(C\) and \(F_2\) a face of \(F_1\).
  Then \(F_2\) is also a face of \(C\).
\end{manualprop}

\begin{exercise}
  Prove this proposition!
\end{exercise}

\begin{solution}
  We have that \(F_2 \subseteq C\) and that \(F_2\) is convex.
  As \(F_2\) is a face of \(F_1\), we have that for \(a, b \in F_1\) such
  that \(\tfrac{1}{2} (a + b) = x \in F_2\), then \(a, b \in F_1\).
  However, \(a \in F_1\) gives us that for \(a_1, a_2 \in C\) such that
  \(\tfrac{1}{2} (a_1 + a_2) = a\) implies that \(a_1, a_2 \in F_1\).
  As \(a \in F_2\), we then have that \(a_1, a_2 \in F_2\), as \(F_2\) is a
  face of \(F_1\).
  Therefore, \(F_2\) is also a face of \(C\).
\end{solution}

\begin{exercise}
  Define
  \begin{equation}
    C = \{
      (x_1, x_2) \in \R^2 :
      x_1 \geq 0,
      x_2 \geq 0,
      x_1 + x_2 \leq 1
    \}.
  \end{equation}
  Why does \(C\) not have any extreme half-line?
  Find all the extreme points of \(C\).
\end{exercise}

\begin{solution}
  We can formulate this as a polyhedra, which gives us that the
  extreme points of \(C\) are \((0, 0)\), \((0, 1)\) and \((1, 0)\).
  All the faces of \(C\) with dimension \(1\) are bounded, and are
  therefore edges, and not half-lines.
\end{solution}

\begin{exercise}
  Consider a polytope \(P \subset \R^n\), say \(P = \conv(\{x_1,
  \ldots, x_n\})\).
  Show that if \(x\) is an extreme point of \(P\), then $x \in \{ x_1,
  \ldots, x_t \}$.
  Is every \(x_j\) necessarily an extreme point?
\end{exercise}

\begin{solution}
  Any point \(x \in P\) can be written as
  \begin{equation}
    x = \sum_{i = 0}^{t} \lambda_i x_i
    \quad\text{with}\quad
    \sum_{i = 0}^{t} \lambda_i = 1,
    \lambda_i \geq 0.
  \end{equation}
  Therefore, the only possible candidates which can only be written
  as the convex combination of itself, are necessarily the points
  \(\{ x_1, \ldots, x_t \}\).
  Not every \(x_i\) are extreme, say e.g.\ that \(x_2\) lies on the line
  between \(x_1\) and \(x_2\).
\end{solution}

\begin{exercise}
  Show that \(\rec(C, x)\) is a closed convex cone.
  First, verify that \(z \in \rec(C, x)\) implies that \(\mu z \in
  \rec(C, x)\) for each \(\mu \geq 0\).
  Next, in order to verify convexity, you may show that
  \begin{equation}
    \rec(C, x) = \bigcap_{\lambda > 0} \frac{1}{\lambda} (C - x),
  \end{equation}
  where \(\tfrac{1}{\lambda} (C - x)\) is the set of all vectors of the
  form \(\tfrac{1}{\lambda}(y - x)\) where \(y \in C\).
\end{exercise}

\begin{solution}
  \(z \in \rec(C, x)\) implies that \(\mu z \in \rec(C, x)\) when
  \(\mu \geq 0\), as we have that \(x + \lambda z \in C\) for all
  \(\lambda \geq 0\), and particularly for \(\tfrac{\lambda}{\mu} \geq 0\).
  (The case with \(\mu = 0\) follows trivially).

  Note next that \(C - x\) is convex, as for any \(y - x \in (C -
  x)\), we can write
  \begin{equation}
    y - x = \lambda (y_1 - x) + (1 - \lambda) (y_2 - x),
  \end{equation}
  with \(y_1, y_2 \in C\).
  Next, note that this also holds for \(\tfrac{1}{\lambda} (C - x)\)
  for any \(\lambda > 0\) by similar logic.
  \(z \in \tfrac{1}{\lambda} (C - x)\) is then equivalent with
  \begin{equation}
    x + \lambda z \in C,
  \end{equation}
  and if \(z\) is in the intersection of all such sets, then this holds
  for all \(\lambda > 0\).
  This is equivalent with the original definition of the recession cone.
\end{solution}

\begin{exercise}
  Show that a closed convex set of \(C\) is bounded if and only if
  \(\rec(C) = \{ 0 \}\).
\end{exercise}

\begin{solution}
  Suppose that \(\rec(C) \neq \{ 0 \}\).
  There is then some non-zero \(z \in \rec(C)\) such that \(x +
  \lambda z \in C\) for all \(\lambda \geq 0\).
  Then, by the reverse triangle inequality for sufficiently large \(\lambda\),
  \begin{equation}
    \norm{x + \lambda z} \geq \abs{\lambda \norm{z} - \norm{x}} =
    \lambda \norm{z} - \norm{x},
  \end{equation}
  meaning there is no \(M\) such that \(\norm{x} \leq M\) for all \(x
  \in C\), so \(C\) is unbounded.

  Suppose now that \(C\) is unbounded, and let \(x_0\) be some point in \(C\).
  As \(C\) is unbounded, we can find a sequence \(x_k \in C\) with
  \(\norm{x_k} \to \infty\).
  Define next \(y_k \coloneq x_k - x_0\) and \(u_k \coloneq
  \frac{y_k}{\norm{y_k}}\).
  Then \(\norm{u_k} = 1\) for all \(k > 0\), and as the unit sphere is
  compact, we can choose a subsequence, still denoted \(u_k\), and some
  \(u\) with \(\norm{u} = 1\) such that \(u_k \to u\).

  Fix any \(\lambda \geq 0\).
  For \(k\) sufficiently large, \(\lambda \leq \norm{y_k}\), so
  \begin{equation}
    t_k \coloneq \frac{\lambda}{\norm{y_k}} \in [0, 1].
  \end{equation}
  As \(C\) is convex,
  \begin{equation}
    x_0 + t_k y_k = (1 - t_k) x_0 + t_k x_k \in C.
  \end{equation}
  However, we have
  \begin{equation}
    x_0 + t_k y_k = x_0 + \frac{\lambda}{\norm{y_k}} y_k = x_0 + \lambda u_k.
  \end{equation}
  As \(u_k \to u\) and \(C\) is closed, we have that \(x_0 + \lambda
  u \in C\), and as \(\lambda \geq 0\) was arbitrary, this proves
  that \(u \in \rec(C)\).
  Additionally, as \(\norm{u} = 1\), we have that \(u \neq 0\), so
  \(\rec(C) \neq \{ 0 \}\).
\end{solution}

\begin{exercise}
  Consider a hyperplane \(H\).
  Determine its recession cone and lineality space.
\end{exercise}

\begin{solution}
  A hyperplane is given by \( H = \{ x \in \R^n : a^T x = \alpha \}
  \), so an element of \( z \in \rec(H) \) satisfies
  \begin{equation}
    x + \lambda z \in H
  \end{equation}
  for all \(\lambda \geq 0\).
  We must therefore have
  \begin{equation}
    a^T (x + \lambda z) = \alpha,
  \end{equation}
  such that \(a^T z = 0\).
  Therefore, the recession cone of \(H\) contains the vectors
  orthogonal to \( a \).
  Note that if \(z\) satisfies \(a^T z = 0\), then so does \(-z\).
  Therefore, \( \rec(H) = -\rec(H) \), and we have \( \lin(H) = \rec(H) \).
\end{solution}

\begin{exercise}
  What is \(\rec(P)\) and \(\lin(P)\) when \( P \) is a polytope?
\end{exercise}

\begin{solution}
  Polytopes are compact, so they are closed and of interest here bounded.
  Therefore, \(\rec(P) = \{ 0 \} = -\rec(P) = \lin(P)\).
\end{solution}

\begin{exercise}
  Let \(C\) be a closed convex cone in \(\R^n\).
  Show that \(\rec(C) = C\).
\end{exercise}

\begin{solution}
  A convex cone \(C\) has the property that \(\lambda_1 x_1 +
  \lambda_2 x_2 \in C\) whenever \( x_1, x_2 \in C \) and
  \(\lambda_1, \lambda_2 \geq 0\).
  Now, simply let \(\lambda_1 = 1\).
  This leaves us with the definition of the recession cone, so the
  sets are equal.
\end{solution}

\begin{exercise}
  Prove that \(\lin(C)\) is a linear subspace of \(\R^n\).
\end{exercise}

\begin{solution}
  We have that \(\lin(C)\) is closed under scalar multiplication, as
  \(x \in \rec(C)\) implies that \(-x \in -\rec(C)\).
  As they are both closed under addition, so is the intersection, and
  \(\lin(C)\) is a linear subspace.
\end{solution}

\begin{exercise}
  Show that \(\rec(\{ x : Ax \leq b\}) = \{ x : Ax \leq 0\} \).
\end{exercise}

\begin{solution}
  Suppose that \(z\) satisfies \(Az \leq 0\).
  Then,
  \begin{equation}
    A(x + \lambda z) = Ax + \lambda Az \leq b,
  \end{equation}
  so \(\{z : Az \leq 0\} \subseteq \rec(\{ x : Ax \leq b \})\).

  Let \( C = \{ x : Ax \leq b \} \).
  Next, suppose that \( x \in C \) and \( z \in \rec(C) \).
  We then have both \(Ax \leq b\) and \(A (x + \lambda z) \leq b \)
  for all \(\lambda \geq 0\).
  For this to hold for all \(\lambda\), we must clearly have \(Az
  \leq 0\), showing that the sets are equal.
\end{solution}

\begin{exercise}
  Let \( C \) be a line-free closed convex set and let \(F\) be an
  extreme halfline of \(C\).
  Show that then there is an \( x \in C \) and a \( z \in \rec(C) \)
  such that \( F = x + \cone(\{z\}) \).
\end{exercise}

\begin{solution}
  An extreme halfline is of the form \( \{ x_{0} + \lambda z :
  \lambda \geq 0 \} \).
  As \(C\) is closed, we have that \( F \subseteq C \), which then
  coincides with the definition given.
\end{solution}

\begin{exercise}
  Decide if the following statement is true:
  if \(z \in \rec(C) \) then \( x + \cone(\{z\} )\) is an extreme
  halfline of \(C\).
\end{exercise}

\begin{solution}
  The statement is false.
  The halfline need not be extreme.
  Consider \( C \subset \R^n \) given by
  \begin{equation}
    C = \{ (x_1, x_2) : x_1 \geq 0, x_2 \geq 0 \}.
  \end{equation}
  Then \( z = (1, 1) \in \rec(C) \), however the only extreme
  halflines are \( (0, 1) \) and \( (1, 0) \).
\end{solution}

\begin{exercise}
  Consider again the set \( C = \{ (x_1, x_2, 0) \in \R^3 : x_1^2 +
  x_2^2 \leq 1 \} \) from \cref{ex:referenced_again}.
  Convince yourself that \(C\) equals the convex hull of its relative boundary.
  Note that we here have \(\bd(C) = C\) so the fact that \(C\) is the
  convex hull of its boundary is not very impressive!
\end{exercise}

\begin{solution}
  We can convince ourselves of this by considering the triangle
  inequality of convex combinations of points at the relative
  boundary, which must then be less than 1.
\end{solution}

\begin{exercise}
  Let \(H\) be a hyperplane in \(\R^n\).
  Prove that \( H \neq \conv(\rbd(H)) \).
\end{exercise}

\begin{solution}
  We've previously shown that \( H = \rint(H) \), so \( \rbd(H) =
  \emptyset \) and the inequality follows.
\end{solution}

\begin{exercise}
  Consider a polyhedral cone \(C = \{ x \in \R^n : Ax \leq 0 \} \)
  (where, as usual, \(A\) is a real \(m \times n\)-matrix).
  Show that \(0\) is the unique vertex of \(C\).
\end{exercise}

\begin{solution}
  Assume that \(\rank(A) < n\).
  Then, there exists a non-zero vector \(z\) such that \(Az = 0\).
  As such, we have that for any \(x \in C\) and any \(\lambda \in \mathbb{R}\),
  \begin{align*}
    A(x + \lambda z) = Ax + \lambda Az = Ax \leq 0,
  \end{align*}
  showing that \(x + \lambda z \in C\).
  Therefore, if \(\rank(A) < n\), then \(C\) has no vertices.
  As such, the exercise is missing information, as it is not
  necessarily true that \(0\) is the unique vertex of \(C\).

  As a simple counterexample, consider
  \begin{equation}
    C = \{ (x_0, x_1) \in \mathbb{R}^2 : x_0 \leq 0 \},
  \end{equation}
  which contains the line \(\{ (0, x_1) : x_1 \in \mathbb{R} \}\), and
  thus has no vertices.

  If on the other hand \(\rank(A) = n\), then there exists an \(n
  \times n\) submatrix \(A'\) of \(A\) such that \(\det(A') \neq 0\).
  Then, the only solution to \(A'x = 0\) is \(x = 0\), regardless of
  which such submatrix we choose.
  Therefore, \(0\) is the only point that can be a vertex of \(C\).
\end{solution}

\begin{exercise}
  Let \(F\) be a face of a convex set \(C\) in \(\R^n\).
  Show that every extreme point of \(F\) is also an extreme point of \(C\).
\end{exercise}

\begin{solution}
  An extreme point of \(F\) is a face of \(F\) of dimension 0, and is
  hence also a face of dimension 0 of \(C\) as \(F\) is a face of \(C\).
\end{solution}

\begin{exercise}
  Find all the faces of the unit ball in \(\R^2\).
  What about the unit ball in \(\R^n\)?
\end{exercise}

\begin{solution}
  The faces of dimension \(0\) are the boundary points, while the
  face of dimension \(n\) is the ball itself.
\end{solution}

\begin{exercise}
  Let \(F\) be a face of a convex set \(C\) in \(\R^n\).
  Show that \(F \subseteq \bd(C) \) (recall that \(\bd(C) \) is the
  boundary of \(C\)).
  Is the stronger statement \(F \subseteq \rbd(C)\) also true?
  Find an example where \( F = \bd(C) \).
\end{exercise}

\begin{solution}
  \todo[inline]{TODO}
\end{solution}

\begin{exercise}
  Consider the convex set \(C = B + ([0, 1] \times \{0\} )\) where
  \(B\) is the unit ball (of the Euclidean norm) in \(\R^2\).
  Find a point on the boundary of \(C\) which is a face of \(C\), but
  not an exposed face.
\end{exercise}

\begin{solution}
  Consider the boundary point \(p = (0, 1)\).
  The unique supporting half-plane \(H\) of \(B\) at that point is the line
  \begin{equation}
    H = \{(x_1, x_2) : x_2 = 1\}.
  \end{equation}
  However, \( H \cap C \neq \{ p \} \), so \(\{p\}\) is a face, but
  not an exposed face.
\end{solution}