\section{Representation of convex sets}

\begin{exercise}
  Consider the polytope $P \subset \R^2$ being the convex hull of the
  point $(0, 0)$, $(1, 0)$ and $(0, 1)$ (so $P$ is a simplex in $\R^2$).
  \begin{enumerate}[label = (\emph{\roman*})]
    \item Find the unique face of $P$ that contains the point $(1/3, 1/2)$.
    \item Find all faces of $P$ that contain the point $(1/3, 2/3)$.
    \item Determine all the faces of $P$.
  \end{enumerate}
\end{exercise}

\begin{solution}
  Beginning with (\emph{iii}), we note the the dimension $0$ faces of
  $P$ are the vertices.
  The dimension $1$ faces are the edges between the vertices, and the
  dimension $2$ face is $P$ itself.
  (\emph{i}): The point $(1/3, 1/2)$ clearly does not lie on any of
  the edges or vertices, so the only face containing it is $P$ itself.
  (\emph{ii}): The point $(1/3, 2/3)$ lies on the edge between $(0,
  1)$ and $(1, 0)$ ($\lambda = 1/3$), so the faces containing it are
  this edge and $P$ itself.
\end{solution}

\begin{exercise}
  Explain why an equivalent definition of face is obtained using the
  condition: if whenever $x_1, x_2 \in C$ and $(1/2)(x_1 + x_2) \in
  F$, then $x_1, x_2 \in F$.
\end{exercise}

\begin{solution}
  Let $F$ be a face of $C$.
  Then, by definition, if $x_1, x_2 \in C$ and $\lambda x_1 + (1 -
  \lambda) x_2 \in F$ for some $\lambda \in (0, 1)$, then $x_1, x_2 \in F$.
  In particular, this holds for $\lambda = 1/2$, so the condition is satisfied.

  For the converse, assume that $F$ satisfies the midpoint condition:
  whenever $x_1, x_2 \in C$ and $(1/2)(x_1 + x_2) \in F$, then $x_1, x_2 \in F$.
  We can essentially perform a binary search to find each point on
  the line segment between $x_1$ and $x_2$.
  More precisely, for a point $y = \lambda x_1 + (1 - \lambda) x_2$
  where $\lambda \in (0, 1)$, we can express $\lambda$ in its binary
  representation.
  This means that there is a sequence $\{ \sigma_i (1/2)^i
  \}_{i=1}^\infty$ where $\sigma_i \in \{ -1, 1 \}$, $\sigma_1 = 1/2$, such that
  \begin{equation}
    \lambda_n = \sum_{i=1}^n \sigma_i (1/2)^i,
  \end{equation}
  and $\lambda_n \to \lambda$ as $n \to \infty$.
  We can then define a sequence of points $\{ y_n \}_{n=1}^\infty$
  where $y_n = \lambda_n x_1 + (1 - \lambda_n) x_2$.
  At each step $n$, we are only considering a subinterval $[x_l^n,
  x_r^n] \subseteq [x_1, x_2]$ where $y_n = \frac{1}{2}(x_l^n + x_r^n)$.
  By the midpoint condition, if $y_n \in F$, then $x_l^n, x_r^n \in F$.
  If $\lambda > \lambda_n$, we set $x_l^{n+1} = y_n$ and $x_r^{n+1} = x_r^n$.
  If $\lambda < \lambda_n$, we set $x_l^{n+1} = x_l^n$ and $x_r^{n+1} = y_n$.
  In either case, we have that $y_{n+1} = \frac{1}{2}(x_l^{n+1} + x_r^{n+1})$.
  Since $y_1 \in F$, we have that $x_l^1, x_r^1 \in F$.
  By induction, we have that $x_l^n, x_r^n \in F$ for all $n$.
  Since $y_n$ is the midpoint of $x_l^n$ and $x_r^n$, we have that
  $y_n \in F$ for all $n$.
  Finally, since $y_n \to y$ as $n \to \infty$ and $F$ is closed, we
  have that $y \in F$.
  Thus, $F$ is a face of $C$.
\end{solution}

\begin{manualprop}{4.1.2}[Face of face]
  Let $C \subseteq \R^n$ be a non-empty convex set.
  Let $F_1$ be a face of $C$ and $F_2$ a face of $F_1$.
  Then $F_2$ is also a face of $C$.
\end{manualprop}

\begin{exercise}
  Prove this proposition!
\end{exercise}

\begin{solution}
  We have that $F_2 \subseteq C$ and that $F_2$ is convex.
  As $F_2$ is a face of $F_1$, we have that for $a, b \in F_1$ such
  that $\tfrac{1}{2} (a + b) = x \in F_2$, then $a, b \in F_1$.
  However, $a \in F_1$ gives us that for $a_1, a_2 \in C$ such that
  $\tfrac{1}{2} (a_1 + a_2) = a$ implies that $a_1, a_2 \in F_1$.
  As $a \in F_2$, we then have that $a_1, a_2 \in F_2$, as $F_2$ is a
  face of $F_1$.
  Therefore, $F_2$ is also a face of $C$.
\end{solution}

\begin{exercise}
  Define
  \begin{equation}
    C = \{
      (x_1, x_2) \in \R^2 :
      x_1 \geq 0,
      x_2 \geq 0,
      x_1 + x_2 \leq 1
    \}.
  \end{equation}
  Why does $C$ not have any extreme half-line?
  Find all the extreme points of $C$.
\end{exercise}

\begin{solution}
  We can formulate this as a polyhedra, which gives us that the
  extreme points of $C$ are $(0, 0)$, $(0, 1)$ and $(1, 0)$.
  All the faces of $C$ with dimension $1$ are bounded, and are
  therefore edges, and not half-lines.
\end{solution}

\begin{exercise}
  Consider a polytope $P \subset \R^n$, say $P = \conv(\{x_1, \ldots, x_n\})$.
  Show that if $x$ is an extreme point of $P$, then $x \in \{ x_1,
  \ldots, x_t \}$.
  Is every $x_j$ necessarily an extreme point?
\end{exercise}

\begin{solution}
  Any point $x \in P$ can be written as
  \begin{equation}
    x = \sum_{i = 0}^{t} \lambda_i x_i
    \quad\text{with}\quad
    \sum_{i = 0}^{t} \lambda_i = 1,
    \lambda_i \geq 0.
  \end{equation}
  Therefore, the only possible candidates which can only be written
  as the convex combination of itself, are necessarily the points
  $\{ x_1, \ldots, x_t \}$.
  Not every $x_i$ are extreme, say e.g.\ that $x_2$ lies on the line
  between $x_1$ and $x_2$.
\end{solution}

\begin{exercise}
  Show that $\rec(C, x)$ is a closed convex cone.
  First, verify that $z \in \rec(C, x)$ implies that $\mu z \in
  \rec(C, x)$ for each $\mu \geq 0$.
  Next, in order to verify convexity, you may show that
  \begin{equation}
    \rec(C, x) = \bigcap_{\lambda > 0} \frac{1}{\lambda} (C - x),
  \end{equation}
  where $\tfrac{1}{\lambda} (C - x)$ is the set of all vectors of the
  form $\tfrac{1}{\lambda}(y - x)$ where $y \in C$.
\end{exercise}

\begin{solution}
  $z \in \rec(C, x)$ implies that $\mu z \in \rec(C, x)$ when $\mu
  \geq 0$, as we have that $x + \lambda z \in C$ for all $\lambda
  \geq 0$, and particularly for $\tfrac{\lambda}{\mu} \geq 0$.
  (The case with $\mu = 0$ follows trivially).

  Note next that $C - x$ is convex, as for any $y - x \in (C - x)$, we can write
  \begin{equation}
    y - x = \lambda (y_1 - x) + (1 - \lambda) (y_2 - x),
  \end{equation}
  with $y_1, y_2 \in C$.
  Next, note that this also holds for $\tfrac{1}{\lambda} (C - x)$
  for any $\lambda > 0$ by similar logic.
  $z \in \tfrac{1}{\lambda} (C - x)$ is then equivalent with
  \begin{equation}
    x + \lambda z \in C,
  \end{equation}
  and if $z$ is in the intersection of all such sets, then this holds
  for all $\lambda > 0$.
  This is equivalent with the original definition of the recession cone.
\end{solution}

\begin{exercise}
  Show that a closed convex set of $C$ is bounded if and only if
  $\rec(C) = \{ 0 \}$.
\end{exercise}

\begin{solution}

\end{solution}