\section{Convex hulls and Carathéodory's theorem}

\begin{exercise}
  Illustrate some combinations (linear, convex, non-negative) of two vectors in $\mathbb{R}^2$.
\end{exercise}

\begin{solution}
  Let $\mathbf{u} =
  \begin{pmatrix} 1 \\ 0
  \end{pmatrix}$ and $\mathbf{v} =
  \begin{pmatrix} 0 \\ 1
  \end{pmatrix}$.
  \begin{itemize}
    \item Linear combinations: $\alpha \mathbf{u} + \beta \mathbf{v}$ for $\alpha, \beta \in \mathbb{R}$.
    \item Convex combinations: $\alpha \mathbf{u} + \beta \mathbf{v}$ for $\alpha, \beta \geq 0$ and $\alpha + \beta = 1$.
    \item Non-negative combinations: $\alpha \mathbf{u} + \beta \mathbf{v}$ for $\alpha, \beta \geq 0$.
  \end{itemize}
  The linear combinations fill the entire $\mathbb{R}^2$ plane, the convex combinations fill the line segment between $\mathbf{u}$ and $\mathbf{v}$, and the non-negative combinations fill the first quadrant of the plane.
\end{solution}

\begin{exercise}
  Choose your favourite three points $x_1$, $x_2$, $x_3$ in $\mathbb{R}^2$, but make sure that they do not all lie on the same line.
  Thus, the three points form the corners of a triangle $C$.
  Describe those points that are convex combinations of two of the three points.
  What about the interior of the triangle $C$, i.e., those points that lie in $C$ but not on the boundary (the three sides): can these points be written as convex combinations of $x_1$, $x_2$ and $x_3$? If so, how?
\end{exercise}

\begin{solution}
  Let's choose the points $x_1 = (0, 0)$, $x_2 = (1, 0)$, and $x_3 = (0, 1)$, i.e., the corners of a right triangle.
  The points that are convex combinations of two of the three points are those that lie on the edges of the triangle.
  The interior points, which do not lie on the boundary, can be expressed as
  \begin{equation}
    c = \lambda_1 x_1 + \lambda_2 x_2 + \lambda_3 x_3,
  \end{equation}
  with $0 < \lambda_i < 1$ for $i = 1, 2, 3$ and $\lambda_1 + \lambda_2 + \lambda_3 = 1$.
  \begin{figure}[htbp]
    \centering

    \resizebox{0.3\textwidth}{!}{
      \begin{tikzpicture}
  \coordinate (A) at (0, 0);
  \coordinate (B) at (0, 1);
  \coordinate (C) at (1, 0);

  \fill[my_red, opacity=0.18] (A) -- (B) -- (C) -- cycle;
  \draw[thick, my_red] (A) -- (B) -- (C) -- cycle;

  \draw[gray, dashed] (-0.25, 0) -- (1.25, 0);
  \draw[gray, dashed] (0, -0.25) -- (0, 1.25);

  \node at (A) [below left] {\tiny$0$};
  \node at (B) [left] {\tiny$1$};
  \node at (C) [below] {\tiny$1$};
\end{tikzpicture}
    }
    \caption{
      The unit triangle with points $x_1$, $x_2$, and $x_3$.\label{fig:unit_triangle}
    }
  \end{figure}
\end{solution}

\begin{exercise}
  Show that $\conv(S)$ is convex for all $S \subseteq \mathbb{R}^n$.
  (Hint: look at two convex combinations $\sum_j \lambda_j x_j$ and $\sum_j \mu_j y_j$, and note that both these points may be written as a convex combination of the same set of vectors.)
\end{exercise}

\begin{solution}
  Let $u, v \in \conv(S)$.
  Then there exists points $x_1, x_2, \ldots, x_n \in S$ and $y_1, y_2, \ldots, y_m \in S$ and coefficients $\lambda_i, \mu_j \geq 0$ with $\sum_i \lambda_i = 1$ and $\sum_j \mu_j = 1$ such that
  \begin{equation}
    u = \sum_{i=1}^n \lambda_i x_i, \quad v = \sum_{j=1}^m \mu_j y_j.
  \end{equation}
  For any $0 \leq \theta \leq 1$, we have that
  \begin{equation}
    \begin{split}
      \theta u + (1 - \theta) v
      &= \theta \sum_{i=1}^n \lambda_i x_i + (1 - \theta) \sum_{j=1}^m \mu_j y_j \\
      &= \sum_{i=1}^n (\theta \lambda_i) x_i + \sum_{j=1}^m ((1 - \theta) \mu_j) y_j.
    \end{split}
  \end{equation}
  The new coefficients are non-negative, and furthermore we have
  \begin{equation}
    \sum_{i=1}^n (\theta \lambda_i) + \sum_{j=1}^m ((1 - \theta) \mu_j)
    = \theta \cdot 1 + (1 - \theta) \cdot 1
    = 1,
  \end{equation}
  showing that $\theta u + (1 - \theta) v$ is a convex combination of points in $S$, and therefore lies in $\conv(S)$.
  $\conv(S)$ is thus convex.
\end{solution}

\begin{exercise}
  Give an example of two distinct sets $S$ and $T$ having the same convex hull.
  It makes sense to look for a smallest possible subset $S_0$ of a set $S$ such that $S = \conv(S_0)$.
  We study this question later.
\end{exercise}

\begin{solution}
  Let $S = \{(0, 0), (1, 0), (0, 1)\}$ and $T = \{(0, 0), (1, 0), (0, 1), (0.5, 0.5)\}$.
  Both sets have the same convex hull, which is the triangle formed by the points $(0, 0)$, $(1, 0)$, and $(0, 1)$.
  The point $(0.5, 0.5)$ in set $T$ lies within this triangle and does not change the convex hull.
\end{solution}

\begin{exercise}
  Prove that if $S \subseteq T$, then $\conv(S) \subseteq \conv(T)$.
\end{exercise}

\begin{solution}
  Let $u \in \conv(S)$.
  Then there exist points $x_1, x_2, \ldots, x_n \in S$ and coefficients $\lambda_i \geq 0$ with $\sum_i \lambda_i = 1$ such that
  \begin{equation}
    u = \sum_{i=1}^n \lambda_i x_i.
  \end{equation}
  Since $S \subseteq T$, we have that $x_i \in T$ for all $i$.
  Therefore, $u$ is also a convex combination of points in $T$, and thus lies in $\conv(T)$.
  Hence, $\conv(S) \subseteq \conv(T)$.
\end{solution}

\begin{exercise}
  If $S$ is convex, then $\conv(S) = S$.
  Show this!
\end{exercise}

\begin{solution}
  Since $S$ is convex, for any $u, v \in S$ and any $0 \leq \lambda \leq 1$, we have $\lambda u + (1 - \lambda) v \in S$.
  By the definition of convex hull, $\conv(S)$ is the smallest convex set containing $S$.
  Since $S$ is already convex and contains itself, it follows that $\conv(S) = S$.
\end{solution}

\begin{exercise}
  Let $S = \{ x \in \mathbb{R}^2 : \norm{x}_2 = 1 \}$, this is the unit circle in $\mathbb{R}^2$.
  Determine $\conv(S)$ and $\cone(S)$.
\end{exercise}

\begin{solution}
  The convex hull $\conv(S)$ of the unit circle is the unit disk, i.e., the set $\{ x \in \mathbb{R}^2 : \norm{x}_2 \leq 1 \}$.
  This is because any point inside the unit circle can be expressed as a convex combination of points on the unit circle.

  The conical hull $\cone(S)$ of the unit circle is the entire $\mathbb{R}^2$ plane.
  This is because any point in $\mathbb{R}^2$ can be expressed as a non-negative combination of points on the unit circle, scaled appropriately.
  This can be seen simply by looking at the problem in polar coordinates.
\end{solution}

\begin{exercise}
  Does affine independence imply linear independence?
  Does linear independence imply affine independence?
  Prove or disprove!
\end{exercise}

\begin{solution}
  Affine independence does not imply linear independence.
  Consider a set of linearly independent vectors $x_2, \ldots, x_t \in \mathbb{R}^n$ and let $x_1 = 0$.
  Then clearly the set $\{x_1, x_2, \ldots, x_t\}$ is affinely independent, but not linearly independent since $x_1$ is the zero vector.

  Linear independence does imply affine independence.
  Suppose $\{x_1, x_2, \ldots, x_t\}$ is linearly independent.
  To show affine independence, we need to show that the only solution to
  \begin{equation}
    \sum_{i=1}^t \lambda_i x_i = 0 \quad \text{and} \quad \sum_{i=1}^t \lambda_i = 0
  \end{equation}
  is $\lambda_i = 0$ for all $i$.
  Since the vectors are linearly independent, the first equation implies that all $\lambda_i$ must be zero.
  Thus, the set is affinely independent.
\end{solution}

\begin{exercise}
  Let $x_1, \ldots, x_t \in \mathbb{R}^n$ be affinely independent and let $w \in \mathbb{R}^n$.
  Show that $x_1 + w, \ldots, x_t + w$ are also affinely independent.
\end{exercise}

\begin{solution}
  As $x_1, \ldots, x_t$ are affinely independent, we have that
  \begin{equation}
    \{ x_2 - x_1, x_3 - x_1, \ldots, x_t - x_1 \}
  \end{equation}
  is linearly independent.
  We then have that
  \begin{equation}
    \{ (x_2 + w) - (x_1 + w), \ldots, (x_t + w) - (x_1 + w) \}
    = \{ x_2 - x_1, x_3 - x_1, \ldots, x_t - x_1 \},
  \end{equation}
  which is linearly independent.
  Thus, $x_1 + w, \ldots, x_t + w$ are affinely independent.
\end{solution}

\paragraph{Dimension of a set.}
The \emph{dimension} of a set $S \subseteq \mathbb{R}^n$, denoted by $\dim(S)$, is the maximal number of affinely independent points in $S$ minus 1.
So, for example in $\mathbb{R}^3$, the dimension of a point and a lines is $0$ and $1$ respectively, and the dimension of the plane $x_3 = 0$ is $2$.

\begin{exercise}
  Let $L$ be a linear subspace of dimension (in the usual linear algebra sense) $t$.
  Check this this coincides with our new definition of dimension above.
  (Hint: add $O$ to a ``suitable'' set of vectors).
\end{exercise}

\begin{solution}
  Let $\{v_1, v_2, \ldots, v_t\}$ be a basis for the linear subspace $L$.
  Then the set $\{0, v_1, v_2, \ldots, v_t\}$ contains $t + 1$ affinely independent points in $L$.
  To see this, note that the vectors $v_1, v_2, \ldots, v_t$ are linearly independent by definition of a basis.
  Therefore, the maximal number of affinely independent points in $L$ is $t + 1$, and thus $\dim(L) = t + 1 - 1 = t$, which coincides with the usual definition of dimension in linear algebra.
\end{solution}

\begin{exercise}
  Consider a convex set $C$ of dimension $d$.
  Then there are (and no more than) $d + 1$ affinely independent points in $C$.
  Let $S = \{ x_1, \ldots, x_{d+1} \}$ denote a set of such points.
  Then the set of all convex combinations of these vectors, i.e., $\conv(S)$ is a polytope contained in $C$ and $\dim(S) = \dim(C)$.
  Moreover, let $A$ be the set of all vectors of the form $\sum_{j = 1}^{t}\lambda_j x_j$ where $\sum_{j = 1}^{t} \lambda_j = 1$ (no sign restrictions of the $\lambda$'s).
  Then $A$ is an affine set containing $C$, and it is the smallest affine set with this property.
  $A$ is called the \emph{affine hull} of $C$.

  Prove the last statements in the previous paragraph.
\end{exercise}

\begin{solution}
  Consider two points $u, v \in A$ where
  \begin{equation}
    u = \sum_{j = 1}^{t} \lambda_j x_j
    \quad\text{and}\quad
    v = \sum_{j = 1}^{t} \mu_j x_j,
  \end{equation}
  where $\sum_{j = 1}^{t} \lambda_j = 1$ and $\sum_{j = 1}^{t} \mu_j = 1$.
  Then, we have that
  \begin{equation}
    (1 - \theta)u + \theta v
    = \sum_{j = 1}^{t} ((1 - \theta) \lambda_j + \theta \mu_j) x_j \in A
  \end{equation}
  for any $\theta \in \mathbb{R}$ as the coefficients still sum to $1$.
  $A$ is therefore affine.
  Choosing $\lambda_j = \delta_{ij}$ (the Kronecker delta) shows that $x_i \in A$ for all $i$, and thus $\conv(C) = C \subseteq A$.
\end{solution}

\begin{exercise}
  Construct a set which is neither open nor closed.
\end{exercise}

\begin{solution}
  Consider the interval $S = (0, 1]$ in $\mathbb{R}$.
  This set is not open because it contains the point $1$, which is a limit point of the set.
  It is not closed because it does not contain the point $0$, which is also a limit point of the set.
  Therefore, $S$ is neither open nor closed.
\end{solution}

\begin{exercise}
  Show that $x^k \to x$ if and only if $x_j^k \to x_j$ for $j = 1, \ldots, n$.
  Thus convergence of a point sequence simply means that all the component sequences are convergent.
\end{exercise}

\begin{solution}
  ($\Rightarrow$) Suppose $x^k \to x$.
  Then, as
  \begin{equation}
    \abs{x_j^k - x_j} \leq \norm{x^k - x},
  \end{equation}
  we have that $x_j^k \to x_j$ for each component $j$.

  ($\Leftarrow$) Conversely, suppose $x_j^k \to x_j$ for each component $j$.
  Then,
  \begin{equation}
    \norm{x^k - x}^2 = \sum_{j = 1}^{n} (x_j^k - x_j)^2 \to 0 = 0,
  \end{equation}
  showing that $x^k \to x$.
  This is all rather informal, but the details are easy to fill in.
\end{solution}

\begin{exercise}
  Show that every simplex cone is closed.
\end{exercise}

\begin{solution}
  Let $x_1, \ldots, x_t \in \mathbb{R}^n$ be linearly independent vectors, and consider the simplex cone spanned by these vectors.
  Let $X$ be the matrix with columns $x_1, \ldots, x_t$.
  Then any point in the simplex cone can be written as $X \boldsymbol{\lambda}$ for some $\boldsymbol{\lambda} \geq 0$.
  $X$ then has full column rank, and thus $X^T X$ is invertible, so we define the pseudo-inverse $X^\dagger = (X^T X)^{-1} X^T$.
  We then obtain $\boldsymbol{\lambda} = X^\dagger x$, such that the simplex cone can be written as the inverse image of the closed set $\{ \boldsymbol{\lambda} : \boldsymbol{\lambda} \geq 0 \}$ under the continuous map $x \mapsto X^\dagger x$.
  The simplex cone is therefore closed.
\end{solution}

\paragraph{Boundary of a set.}
The \emph{boundary} $\bd(S)$ of $S$ is defined by $\bd(S) = \cl(S) \setminus \intr(S)$.
For instance, we have that $\bd(B(a, r)) = \{ x \in \mathbb{R}^n : \norm{x - a} = r \}$.

\begin{exercise}
  Prove that $x \in \bd(S)$ if and only if each ball with center $x$ intersects both $S$ and the complement of $S$.
\end{exercise}

\begin{solution}
  For $x \in \bd(S)$, we have that $x \in \cl(S)$ and $x \notin \intr(S)$.
  Since $x \in \cl(S)$, every ball centred at $x$ intersects $S$.
  Since $x \notin \intr(S)$, every ball centred at $x$ also intersects the complement of $S$.
\end{solution}

\paragraph{Affine hull.}
The \emph{affine hull} of a set $S$, denoted by $\aff(S)$, is the smallest affine set containing $S$.

\paragraph{Relative interior and relative boundary.}
We say that $x$ is a \emph{relative interior point} of $S$ if there is an $r > 0$ such that
\begin{equation}
  B^\circ(x, r) \cap \aff(S) \subseteq S.
\end{equation}
This means that $x$ is the center of some open ball whose intersection with $\aff(S)$ is contained in $S$.
We let the \emph{relative interior} of $S$, denoted by $\intr(S)$, be the set of all such relative interior points of $S$.
Similarly, we define the \emph{relative boundary} of $S$ by $\rbd(S) = \cl(S) \setminus \rint(S)$.

\begin{exercise}
  Consider again the set $C = \{ (x_1, x_2, 0) \in \mathbb{R}^3 : x_1^2 + x_2^2 \leq 1 \}$.
  Verify that:
  \begin{enumerate}[label = (\textit{\roman*})]
    \item $C$ is closed,
    \item $\dim(C) = 2$,
    \item $\intr(C) = \emptyset$,
    \item $\bd(C) = C$,
    \item $\rint(C) = \{ (x_1, x_2, 0) \in \mathbb{R}^3 : x_1^2 + x_2^2 < 1 \}$,
    \item $\rbd(C) = \{ (x_1, x_2, 0) \in \mathbb{R}^3 : x_1^2 + x_2^2 = 1 \}$.
  \end{enumerate}
\end{exercise}

\begin{solution}
  (\textit{i}): Assume that $x^k \to x$ where $x^k = (x_k, y_k, 0) \in C$ for all $k$.
  As $x_k^2 + y_k^2 \leq 1$ for all $k$, we have that $x^2 + y^2 \leq 1$ by taking limits, showing that $x \in C$.
  Similarly, as $z_k = 0$ for all $k$, we have that $z = 0$, and thus $x \in C$.
  $C$ is therefore closed.

  (\textit{ii}): We find only two linearly independent vectors in $C$, for instance $(1, 0, 0)$ and $(0, 1, 0)$.
  Thus, $\dim(C) = 2$.
  A similar argument is possible through the affinely independent points $(0, 0, 0)$, $(1, 0, 0)$, and $(0, 1, 0)$.

  (\textit{iii}): Any ball centred around a point $x = (x_1, x_2, 0) \in C$ will contain points of the form $(x_1, x_2, \varepsilon)$ for $\varepsilon > 0$, which are not in $C$.
  Thus, $\intr(C) = \emptyset$.

  (\textit{iv}): As $\intr(C) = \emptyset$, we have that $\bd(C) = \cl(C) \setminus \intr(C) = C \setminus \emptyset = C$.

  (\textit{v}): We have that $\aff(C)$ is the plane $z = 0$.
  For a point $x = (x_1, x_2, 0) \in C$ with $x_1^2 + x_2^2 < 1$, we can find a ball $B^\circ(x, r) \subset B^\circ(0, 1)$.
  Then
  \begin{equation}
    \begin{split}
      B^\circ(x, r) \cap \aff(C)
      &= \{ (y_1, y_2, 0) : (y_1 - x_1)^2 + (y_2 - x_2)^2 < r^2 \} \\
      &= (B^\circ((x_1, x_2), r), 0) \subseteq C,
    \end{split}
  \end{equation}
  so $\{ (x_1, x_2, 0) \in \mathbb{R}^3 : x_1^2 + x_2^2 < 1 \} \subseteq \rint(C)$.
  If we have a point $x = (x_1, x_2, 0)$ with $x_1^2 + x_2^2 = 1$, any ball centred at $x$ will contain points of the form $(x_1, x_2, \varepsilon)$ for $\varepsilon > 0$, which are not in $C$.
  Thus, $\rint(C) = \{ (x_1, x_2, 0) \in \mathbb{R}^3 : x_1^2 + x_2^2 < 1 \}$.

  (\textit{v}): We have simply
  \begin{equation}
    \begin{split}
      \rbd(C) &= \cl(C) \setminus \rint(C) = C \setminus \{ (x_1, x_2, 0) \in \mathbb{R}^3 : x_1^2 + x_2^2 < 1 \} \\
      &= \{ (x_1, x_2, 0) \in \mathbb{R}^3 : x_1^2 + x_2^2 = 1 \}.
    \end{split}
  \end{equation}
\end{solution}

\begin{exercise}
  Show that every polytope in $\mathbb{R}^n$ is bounded.
  (Hint: use the properties of the norm: $\norm{x + y} \leq \norm{x} + \norm{y}$ and $\norm{\lambda x} = \lambda \norm{x}$ for $\lambda \geq 0$.)
\end{exercise}

\begin{solution}
  For a polytope $P = \conv(\{x_1, \ldots, x_t\})$, we can express any point $x \in P$ as
  \begin{equation}
    x = \sum_{i=1}^t \lambda_i x_i,
  \end{equation}
  where $\lambda_i \geq 0$ and $\sum_{i=1}^t \lambda_i = 1$.
  We then have that
  \begin{equation}
    \begin{split}
      \norm{x}
      &= \norm*{\sum_{i=1}^t \lambda_i x_i} \leq \sum_{i=1}^t \norm{\lambda_i x_i} = \sum_{i=1}^t \lambda_i \norm{x_i} \\
      &\leq \max_{i=1, \ldots, t} \norm{x_i} \sum_{i=1}^t \lambda_i = \max_{i=1, \ldots, t} \norm{x_i}.
    \end{split}
  \end{equation}
  Thus, $P$ is bounded.
\end{solution}

\begin{exercise}
  Consider the standard simplex $S_t$.
  Show that it is compact, i.e., closed and bounded.
\end{exercise}

\begin{solution}
  The standard simplex $S_t$ is given by
  \begin{equation}
    S_t = \{ x \in \mathbb{R}^t : x_i \geq 0, \sum_{i=1}^t x_i = 1 \},
  \end{equation}
  or equivalently, $S_t = \conv(\{e_1, e_2, \ldots, e_t\})$ where $e_i$ are the standard basis vectors in $\mathbb{R}^t$.
  As $S_t$ is a polytope, it is bounded by the previous exercise.
  It is closed as it is the inverse of the closed set $\{1\}$ under the continuous map $x \mapsto \sum_{i=1}^t x_i$ intersected with the closed set $\{ x : x_i \geq 0 \text{ for all } i \}$.
  Thus, $S_t$ is compact.
\end{solution}

\begin{exercise}
  Give an example of a convex cone which is not closed.
\end{exercise}

\begin{solution}
  Consider the set $C = \{ (x_1, x_2) \in \mathbb{R}^2 : x_2 > 0 \}$.
  This set is a convex cone because for any $x, y \in C$ and any $\alpha, \beta \geq 0$, we have $\alpha x + \beta y \in C$.
  However, $C$ is not closed because it does not contain the boundary line $x_2 = 0$.
  Thus, $C$ is a convex cone that is not closed.
\end{solution}

\begin{exercise}
  Let $S \subseteq \mathbb{R}^2$ and let $W$ be the set of all convex combinations of points in $S$.
  Prove that $W$ is convex.
\end{exercise}

\begin{solution}
  Let $u, v \in W$ be two convex combinations of points in $S$, i.e.,
  \begin{equation}
    u = \sum_{i=1}^m \lambda_i x_i
    \quad\text{and}\quad
    v = \sum_{j=1}^n \mu_j y_j.
  \end{equation}
  where $x_i, y_j \in S$, $\lambda_i, \mu_j \geq 0$, and $\sum_{i=1}^m \lambda_i = 1$, $\sum_{j=1}^n \mu_j = 1$.
  For any $0 \leq \theta \leq 1$, we have that
  \begin{equation}
    \begin{split}
      \theta u + (1 - \theta) v
      &= \theta \sum_{i=1}^m \lambda_i x_i + (1 - \theta) \sum_{j=1}^n \mu_j y_j \\
      &= \sum_{i=1}^m (\theta \lambda_i) x_i + \sum_{j=1}^n ((1 - \theta) \mu_j) y_j.
    \end{split}
  \end{equation}
  The new coefficients are non-negative, and furthermore we have
  \begin{equation}
    \sum_{i=1}^m (\theta \lambda_i) + \sum_{j=1}^n ((1 - \theta) \mu_j)
    = \theta \cdot 1 + (1 - \theta) \cdot 1
    = 1,
  \end{equation}
  showing that $\theta u + (1 - \theta) v$ is a convex combination of points in $S$, and therefore lies in $W$.
  $W$ is thus convex.
\end{solution}