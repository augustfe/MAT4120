\section{The basic concepts}

\begin{exercise}
  Let \( x_1, x_2, y_1, y_2 \in \R^n \) and assume that $x_1 \leq
  x_2$ and $y_1 \leq y_2$.
  Verify that the inequality $x_1 + y_1 \leq x_2 + y_2$ holds.
  Let now $\lambda$ be a non-negative real number.
  Explain why $\lambda x_1 \leq \lambda x_2$ holds.
  What happens if $\lambda$ is negative?
\end{exercise}

\begin{solution}
  With $x_1 \leq x_2$, we have that
  \begin{equation}
    (x_1)_i \leq (x_2)_i \qquad \forall i = 1, \ldots, n.
  \end{equation}
  Component-wise, we then have
  \begin{equation}
    (x_1)_i + (y_1)_i \leq (x_2)_i + (y_2)_i \qquad \forall i = 1, \ldots, n,
  \end{equation}
  and thus $x_1 + y_1 \leq x_2 + y_2$.
  Similarly, if $\lambda \geq 0$, we have
  \begin{equation}
    \lambda (x_1)_i \leq \lambda (x_2)_i \qquad \forall i = 1, \ldots, n,
  \end{equation}
  and therefore $\lambda x_1 \leq \lambda x_2$.
  Finally, for $\lambda < 0$, the inequality reverses:
  \begin{equation}
    \lambda (x_1)_i \geq \lambda (x_2)_i \qquad \forall i = 1, \ldots, n,
  \end{equation}
  giving $\lambda x_1 \geq \lambda x_2$.
\end{solution}

\paragraph{Example 1.2.1}
(\emph{The non-negative real vectors})
The sum of two non-negative numbers is again a non-negative number.
Similarly, we see that the sum of two non-negative vectors is a
non-negative vector.
Moreover, if we multiply a non-negative vector by a non-negative
number, we get another non-negative vector.
These two properties may be summarized by saying that $\R^n_+$ is
closed under addition and multiplication by non-negative scalars.
We shall see that this means that $\R^n_+$ is a convex cone, a
special type of convex set.

\begin{exercise}
  Think about the question in Exercise 1.1 again, now in light of the
  properties explained in Example 1.2.1.
\end{exercise}

\begin{solution}
  We can now rewrite $x_1 \leq x_2$ as $x_2 - x_1 \in \R^n_+$.
  We can now easily consider the first question as
  \begin{equation}
    (x_2 + y_2) - (x_1 + y_1) = (x_2 - x_1) + (y_2 - y_1) \in \R^n_+,
  \end{equation}
  as $\R^n_+$ is closed under addition.
  Similarly, we can use the fact that $\R^n_+$ is closed under
  multiplication by non-negative scalars to see that
  \begin{equation}
    (\lambda x_2 - \lambda x_1) = \lambda (x_2 - x_1) \in \R^n_+,
  \end{equation}
  for $\lambda \geq 0$.
  As $\R^n_+$ is not closed under multiplication by negative scalars,
  we cannot conclude that $\lambda x_1 \leq \lambda x_2$ for $\lambda < 0$.
\end{solution}

\begin{exercise}
  Let $a \in \R^n_+$ and assume that $x \leq y$.
  Show that $a^T x \leq a^T y$.
  What happens if we do not require $a$ to be non-negative here?
\end{exercise}

\begin{solution}
  With $a \in \R^n_+$ and $x \leq y$, we have that
  \begin{equation}
    x_i \leq y_i \qquad \forall i = 1, \ldots, n,
  \end{equation}
  and consequently
  \begin{equation}
    a_i x_i \leq a_i y_i \qquad \forall i = 1, \ldots, n,
  \end{equation}
  as shown previously.
  Written in vector notation, we therefore have
  \begin{equation}
    a^T x \leq a^T y.
  \end{equation}
  With $a$ not necessarily non-negative, we may have neither $a^T x
  \geq a^T y$ nor $a^T x \leq a^T y$, as we could have $a_i x_i > a_i
  y_i$ for some $i$.
\end{solution}

\begin{exercise}
  Show that every ball $B(a, r): = \{ x \in \R^n : \norm{x - a} \leq
  r \}$ is convex (where $a \in \R^n$ and $r \geq 0$).
\end{exercise}

\begin{solution}
  Let $x, y \in B(a, r)$ for some $a \in \R^n$ and $r \geq 0$.
  Then, let $0 \leq \lambda \leq 1$ and consider $z = \lambda x + (1
  - \lambda) y$.
  We then have
  \begin{equation}
    \begin{split}
      \norm{z - a} &= \norm{\lambda (x - a) + (1 - \lambda) (y - a)} \\
      &\leq \lambda \norm{x - a} + (1 - \lambda) \norm{y - a} \\
      &\leq \lambda r + (1 - \lambda) r = r,
    \end{split}
  \end{equation}
  showing that $z \in B(a, r)$.
  $B(a, r)$ is therefore convex.
\end{solution}

\begin{exercise}
  Explain how you can write the LP problem $\max\{ c^T x : Ax \leq b
  \}$ in the form $\max\{ c^T x : Ax = b, x \geq O \}$
\end{exercise}

\begin{solution}
  We introduce new slack variables $w \in \R^m_+$, where $m$ is the
  number of rows/inequalities in $A$, defined by
  \begin{equation}
    w_j = b_j - (Ax)_j \quad \forall j = 1, \ldots, m.
  \end{equation}
  We can then rewrite our system of equations by setting $\tilde{A} =
  \begin{bmatrix} A & I
  \end{bmatrix}$, $\tilde{x} =
  \begin{bmatrix} x \\ w
  \end{bmatrix}$, and $\tilde{c} =
  \begin{bmatrix} c \\ 0
  \end{bmatrix}$.
  We then have
  \begin{equation}
    \tilde{A} \tilde{x} =
    \begin{bmatrix} A & I
    \end{bmatrix}
    \begin{bmatrix} x \\ w
    \end{bmatrix} = Ax + w = b,
  \end{equation}
  and
  \begin{equation}
    \tilde{c}^T \tilde{x} =
    \begin{bmatrix} c \\ 0
    \end{bmatrix}^T
    \begin{bmatrix} x \\ w
    \end{bmatrix} = c^T x.
  \end{equation}
  Again, as we require $w \geq 0$, we then have $Ax \leq b$.
\end{solution}

\begin{exercise}
  Make a drawing of the standard simplices $S_1$, $S_2$, and $S_3$.
  Verify that each unit vector $e_j$ lies in $S_n$ ($e_j$ has a one
  in position $j$, all other components are zero).
  Each $x \in S_n$ may be written as a linear combination $x =
  \sum_{j=1}^{n} \lambda_j e_j$ where each $\lambda_j$ is
  non-negative and $\sum_{j = 1}^{n} \lambda_j = 1$.
  How?
  Can this be done in several ways?
\end{exercise}

\begin{solution}

  \cref{fig:simplices} shows the standard simplices $S_1$, $S_2$, and $S_3$.
  Clearly each unit vector $e_j$ lies in $S_n$.
  Each $x \in S_n$ may be written as $\sum_{j=1}^{n} \lambda_j e_j$
  where $\lambda_j = x_j$, i.e.\ the coordinate components of $x$.

  \begin{figure}[htbp]
    \centering

    \resizebox{0.9\textwidth}{!}{
      \begin{tikzpicture}[line cap=round, line join=round, >=latex]
  \tikzset{
    axis/.style={gray, dashed},
    tick/.style={gray, thick},
    vlabel/.style={font=\small},
  }

  \begin{scope}
    \coordinate (A1) at (0,0);
    \coordinate (B1) at (1,0);

    \draw[axis] ($(A1)+(-0.4,0)$) -- ($(B1)+(0.4,0)$);

    \def\yt{0.15}
    \foreach \P in {A1,B1}{
      \draw[tick] ($(\P)+(0,\yt)$) -- ($(\P)+(0,-\yt)$);
    }
    \node[vlabel, below left] at (A1) {$0$};
    \node[vlabel, below right] at (B1) {$e_1$};

    \draw[very thick, my_red] (A1)--(B1);
  \end{scope}

  \begin{scope}[xshift=2.5cm]
    \coordinate (O2) at (0,0);
    \coordinate (X2) at (1,0);
    \coordinate (Y2) at (0,1);

    \draw[axis] ($(O2)+(-0.4,0)$)--($(X2)+(0.4,0)$);
    \draw[axis] ($(O2)+(0,-0.4)$)--($(Y2)+(0,0.4)$);

    \coordinate (A2) at (0,0);
    \coordinate (B2) at (1,0);
    \coordinate (C2) at (0,1);

    \fill[my_red, opacity=0.18] (A2)--(B2)--(C2)--cycle;
    \draw[very thick,my_red] (A2)--(B2)--(C2)--cycle;

    \node[vlabel, below left]  at (A2) {$0$};
    \node[vlabel, below right] at (B2) {$e_1$};
    \node[vlabel, above left]  at (C2) {$e_2$};
  \end{scope}

  \begin{scope}[
      xshift=5cm,
      x={(1,0)}, y={(0,1)}, z={(0.8, 0.6)}
    ]

    % Axes 0..1 with padding
    \draw[axis] (-0.4,0,0) -- (1.4,0,0);
    \draw[axis] (0,-0.4,0) -- (0,1.4,0);
    \draw[axis] (0,0,-0.4) -- (0,0,1.4);

    % Ticks at 1
    \def\tt{0.06}
    \draw[tick] (1,0,0)++(\tt,0,0)--+(-2*\tt,0,0);
    \draw[tick] (0,1,0)++(0,\tt,0)--+(0,-2*\tt,0);
    \draw[tick] (0,0,1)++(\tt,0,0)--+(-2*\tt,0,0);

    % Vertices
    \coordinate (O3) at (0,0,0);
    \coordinate (X3) at (1,0,0);
    \coordinate (Y3) at (0,1,0);
    \coordinate (Z3) at (0,0,1);

    \begin{scope}[canvas is xz plane at y=0] % triangle O-X-Z
      \fill[my_red, opacity=0.18] (0,0) -- (1,0) -- (0,1) -- cycle;
    \end{scope}

    \begin{scope}[canvas is yz plane at x=0] % triangle O-Y-Z
      \fill[my_red, opacity=0.18] (0,0) -- (1,0) -- (0,1) -- cycle;
    \end{scope}

    \begin{scope}[canvas is xy plane at z=0] % triangle O-X-Y
      \fill[my_red, opacity=0.18] (0,0) -- (1,0) -- (0,1) -- cycle;
    \end{scope}

    \draw[very thick, my_red, dashed] (O3) -- (Z3);
    \draw[very thick, my_red] (O3) -- (X3) -- (Z3) -- (Y3);
    \draw[very thick, my_red] (O3) -- (Y3) -- (X3);

    % Labels
    \node[vlabel, below left]  at (O3) {$0$};
    \node[vlabel, below right] at (X3) {$e_1$};
    \node[vlabel, above left]  at (Y3) {$e_2$};
    \node[vlabel, above right] at (Z3) {$e_3$};
  \end{scope}
\end{tikzpicture}

    }
    \caption{
      The simplices $S_1$, $S_2$, and $S_3$.
      For $S_1$, the standard simplex is the point $e_1$,
      for $S_2$, the standard simplex is the line segment between
      $e_1$ and $e_2$,
      and for $S_3$, the standard simplex is the triangle with
      vertices $e_1$, $e_2$, and $e_3$.\label{fig:simplices}
    }
  \end{figure}
\end{solution}

\begin{exercise}
  Show that each convex cone is indeed a convex set.
\end{exercise}

\begin{solution}
  To see that a convex cone is a convex set, let first $x_1,x_2 \in C$.
  Then let $0 \leq \lambda_1 \leq 1$ and $\lambda_2 = 1 - \lambda_1 \geq 0$.
  We then have by definition of the convex cone that
  \begin{equation}
    \lambda_1 x_1 + (1 - \lambda_1) x_1 = \lambda_1 x_1 + \lambda_2 x_2 \in C,
  \end{equation}
  showing that the set is convex.
\end{solution}

\begin{exercise}
  Let $A \in \R^{m \times n}$ and consider the set $C = \{ x \in \R^n
  : Ax \leq O \}$.
  Prove that $C$ is a convex cone.
\end{exercise}

\begin{solution}
  Let $x_1, x_2 \in C$ and $\lambda_1, \lambda_2 \in \R_+$.
  We then have
  \begin{equation}
    A(\lambda_1 x_1 + \lambda_2 x_2) = \lambda_1 Ax_1 + \lambda_2
    Ax_2 \leq \lambda_1 O + \lambda_2 O = O,
  \end{equation}
  showing that the set is a convex cone.
\end{solution}

\paragraph{Polyhedral cone}
A convex cone of the form $\{x \in \R^n : Ax \leq O \}$ where $A \in
\R^{m \times n}$ is called a \emph{polyhedral cone}.
Let $x_1, \ldots, x_t \in \R^n$ and let $C(x_1, \ldots, x_t)$ be the
set of vectors of the form
\begin{equation}
  u = \sum_{j = 1}^{t} \lambda_j x_j,
\end{equation}
where $\lambda_j \geq 0$ for each $j = 1, \ldots, t$.

\begin{exercise}
  Prove that $C(x_1, \ldots, x_t)$ is a convex cone.
\end{exercise}

\begin{solution}
  Let $C = C(x_1, \ldots, x_t)$ here for convenience.
  Let $u, v \in C$ with respective coefficients $\lambda_j, \mu_j
  \geq 0$ for $j = 1, \ldots, t$.
  Then, for arbitrary coefficients $\alpha, \beta \geq 0$, we have
  \begin{equation}
    \begin{split}
      A(\alpha u + \beta v) &= A\left(\alpha \sum_{j=1}^{t} \lambda_j
      x_j + \beta \sum_{j=1}^{t} \mu_j x_j\right) \\
      &= \alpha A \sum_{j=1}^{t} \lambda_j x_j + \beta A
      \sum_{j=1}^{t} \mu_j x_j \\
      &\leq \alpha O + \beta O = O,
    \end{split}
  \end{equation}
  showing that $\alpha u + \beta v \in C$, and that $C$ is a convex cone.
\end{solution}

\begin{exercise}
  Let $S = \{ (x, y, z) : z \geq x^2 + y^2 \} \subset \R^3$.
  Sketch the set and verify that it is a convex set.
  Is $S$ a finitely generated cone?
\end{exercise}

\begin{solution}
  Let $u = (x_1, y_1, z_1)$ and $v = (x_2, y_2, z_2)$ be two points
  in $S$, and $0 \leq \lambda \leq 1$.
  We then seek to show that $\lambda u + (1 - \lambda) v \in S$.
  Note that $f(x) = x^2$ is convex, i.e.\ that $f(\lambda x_1 + (1 -
  \lambda) x_2) \leq \lambda f(x_1) + (1 - \lambda) f(x_2)$.

  Considering the right hand side of the inequality first, we have
  component-wise that
  \begin{equation}
    \begin{split}
      (\lambda x_1 + (1 - \lambda) x_2)^2 &\leq \lambda x_1^2 + (1 -
      \lambda) x_2^2 \\
      (\lambda y_1 + (1 - \lambda) y_2)^2 &\leq \lambda y_1^2 + (1 -
      \lambda) y_2^2,
    \end{split}
  \end{equation}
  and therefore
  \begin{equation}
    \begin{split}
      (\lambda x_1 + (1 - \lambda) x_2)^2 + (\lambda y_1 + (1 -
      \lambda) y_2)^2 &\leq \lambda (x_1^2 + y_1^2) + (1 - \lambda)
      (x_2^2 + y_2^2) \\
      &\leq \lambda z_1 + (1 - \lambda) z_2,
    \end{split}
  \end{equation}
  and so $\lambda u + (1 - \lambda) v \in S$, showing that $S$ is convex.

  $S$ is not a finitely generated cone, as it should then be closed
  under scaling.
  Consider e.g.\ $u = (1,0,1)$, which satisfies $1 \geq 1^2 + 0^2$
  and so $u \in S$.
  Letting however $\lambda_1 = 2$ (and $\lambda_2 = 0$), we have $2u
  = (2,0,2) \notin S$ as $2 \not\geq 2^2 + 0^2$.
\end{solution}

\begin{exercise}
  Consider the linear system $0 \leq x_i \leq 1$ for $i = 1, \ldots,
  n$, and let $P$ denote the solution set.
  Explain how to solve a linear programming problem
  \begin{equation}
    \max\{ c^T x : x \in P \}.
  \end{equation}
  What if the linear system was $a_i \leq x_i \leq b_i$ for $i = 1, \ldots, n$?
  Here we assume $a_i \leq b_i$ for each $i$.
\end{exercise}

\begin{solution}
  We choose the solution component-wise, by maximizing each component $c_i x_i$.
  If $c_i > 0$, simply let $x_i = 1$.
  If $c_i < 0$, increasing $x_i$ decreases the objective function, so
  we let $x_i = 0$.
  The same argument holds in the alternate case, just choose $x_i =
  b_i$ or $x_i = a_i$ respectively based on the sign of $c_i$.
\end{solution}

\begin{exercise}
  Is the union of two convex sets again convex?
\end{exercise}

\begin{solution}
  No.
  Let $A = [-2, -1]$ and $B = [1, 2]$.
  Then $A \cup B = [-2, -1] \cup [1, 2]$ is not convex, as
  e.g.\ $\tfrac{1}{2} 1 + (1 - \tfrac{1}{2}) (-1) = 0 \notin A \cup B$.
\end{solution}

\begin{exercise}
  Determine the sum $A + B$ in each of the following cases:
  \begin{align*}
    (i) && A &= \{ (x, y) : x^2 + y^2 \leq 1 \}, & B &= \{ (3, 4) \}; \\
    (ii) && A &= \{ (x, y) : x^2 + y^2 \leq 1 \}, & B &= [0, 1] \times \{0\}; \\
    (iii) && A &= \{ (x, y) : x + 2y = 5 \}, & B &= \{ (x, y) : x =
    y, 0 \leq x \leq 1 \}; \\
    (iv) && A &= [0, 1] \times [1, 2], & B &= [0, 2] \times [0, 2].
  \end{align*}
\end{exercise}

\begin{solution}
  Taking the cases in turn:

  (\emph{i}) The sum $A + B$ is given by the set of points $\{ u +
  (3, 4) : u \in A \}$, that is, the unit disk centred around $(3, 4)$.

  (\emph{ii}) The sum $A + B$ is given by those points that are
  either in the rectangle with corners at $(0,-1)$, $(1, -1)$, $(1,
  1)$ and $(0, 1)$, or in the unit disks centred about $(0,0)$ or $(1, 0)$.

  (\emph{iii}) Let $B = \{ (\lambda, \lambda) : 0 \leq \lambda \leq 1 \}$.
  Then we can write a point $(x, y) \in A + B$ as
  \begin{equation}
    (x, y) = (x_0 + \lambda, y_0 + \lambda).
  \end{equation}
  Then
  \begin{equation}
    x + 2y = (x_0 + 2y_0) + 3 \lambda = 5 + 3\lambda,
  \end{equation}
  which gives us that
  \begin{equation}
    A + B = \{ (x, y) : 5 \leq x + 2y \leq 8 \}.
  \end{equation}

  (\emph{iv}) The sum $A + B$ is simply given by $[0, 3] \times [1, 4]$.
\end{solution}

\begin{exercise}
  More enumerated exercises\dots

  \begin{enumerate}[label = (\emph{\roman*})]
    \item Prove that, for every $\lambda \in \R$ and $A, B \subseteq
      \R^n$, it holds that $\lambda(A + B) = \lambda A + \lambda B$.

    \item Is it true that $(\lambda + \mu) A = \lambda A + \mu A$ for
      every $\lambda, \mu \in \R$ and $A \subseteq \R^n$?
      If not, find a counterexample.

    \item Show that, if $\lambda, \mu \geq 0$ and $A \subseteq \R^n$
      is convex, then $(\lambda + \mu) A = \lambda A + \mu A$.
  \end{enumerate}
\end{exercise}

\begin{solution}
  Taking the exercises in turn again\dots

  (\emph{i}) We have that
  \begin{align*}
    \lambda (A + B) &= \{ \lambda (a + b) : a \in A, b \in B \} \\
    &= \{ \lambda a + \lambda b : a \in A, b \in B \} \\
    &= \{ a + b : a \in \lambda A, b \in \lambda B \} \\
    &= \lambda A + \lambda B.
  \end{align*}

  (\emph{ii}) No, it is not true.
  Consider $A = [1, 2]$, $\lambda = 1$ and $\mu = -1$.
  Then we have
  \begin{equation}
    (\lambda + \mu) A = 0A = \{ 0 \}
    \quad\text{and}\quad
    \lambda A + \mu A = [1, 2] + [-2, -1] = [-1, 1].
  \end{equation}

  (\emph{iii}) Let $\lambda, \mu \geq 0$.
  For any $a \in A$, we have that
  \begin{equation}
    (\lambda + \mu) a = \lambda a + \mu a \in \lambda A + \mu A,
  \end{equation}
  so $(\lambda + \mu) A \subseteq \lambda A + \mu A$.
  For the reverse inclusion, let $u \in \lambda A + \mu A$.
  Then $u = \lambda a + \mu b$ for some $a, b \in A$.
  Scaling the factors, we have that
  \begin{equation}
    u = (\lambda + \mu) \left(
      \frac{\lambda}{\lambda + \mu} a + \frac{\mu}{\lambda + \mu} b
    \right).
  \end{equation}
  As both the inner coefficients are non-negative and sum to one, and
  $a, b \in A$, we have that
  \begin{equation}
    \frac{\lambda}{\lambda + \mu} a + \frac{\mu}{\lambda + \mu} b \in A,
  \end{equation}
  and $u \in (\lambda + \mu) A$.
  Therefore, $(\lambda + \mu) A = \lambda A + \mu A$ with the given assumptions.
\end{solution}

\begin{exercise}
  Show that if $C_1, \dots, C_t \subseteq \R^n$ are all convex sets,
  then $C_1 \cap \dots \cap C_t$ is convex.
  Do the same when all sets are affine (or linear subspaces, or convex cones).
  In fact, a similar result for the intersection of any family of convex sets.
  Explain this.
\end{exercise}

\begin{solution}
  Let $x, y \in C_1 \cap \dots \cap C_t$ and $0 \leq \lambda \leq 1$.
  Then $x, y \in C_i$ for all $i = 1, \dots, t$. Since each $C_i$ is
  convex, we have that
  \begin{equation}
    \lambda x + (1 - \lambda) y \in C_i
  \end{equation}
  for all $i = 1, \dots, t$. Thus,
  \begin{equation}
    \lambda x + (1 - \lambda) y \in C_1 \cap \dots \cap C_t,
  \end{equation}
  proving that the intersection is convex.

  Suppose there is a matrix $A \in \R^{m \times n}$ and a vector $b \in \R^m$.
  Then the set
  \begin{equation}
    C = \{ x \in \R^n : Ax = b \}
  \end{equation}
  is an affine set.
  Then let $x \in \bigcap_{i = 1}^t C_i$, meaning that $A_i x = b_i$
  for all $i = 1, \dots, t$.
  Thus, $x$ satisfies
  \begin{equation}
    (A_1 + A_2 + \cdots + A_t) x = b_1 + b_2 + \cdots + b_t,
  \end{equation}
  and $\bigcap_{i = 1}^t C_i$ is itself affine.

  A similar argument shows that the closure property of convex sets
  is preserved under finite intersections.
\end{solution}

\begin{exercise}
  Consider a family (possibly infinite) of linear inequalities $a_i^T
  x \leq b, i \in I$, and $C$ be its solution set, i.e., $C$ is the
  set of points satisfying all the inequalities.
  Prove that $C$ is a convex set.
\end{exercise}

\begin{solution}
  Let $x, y \in C$, and $0 \leq \lambda \leq 1$.
  Then, for each $i \in I$, we have
  \begin{equation}
    a_i^T (\lambda x + (1 - \lambda) y) = \lambda a_i^T x + (1 -
    \lambda) a_i^T y \leq \lambda b_i + (1 - \lambda) b_i = b_i.
  \end{equation}
  Therefore, $\lambda x + (1 - \lambda) y \in C$, and $C$ is convex.
\end{solution}

\begin{exercise}
  Consider the unit disc $S = \{ (x_1, x_2) \in \R^2 : x_1^2 + x_2^2
  \leq 1 \}$ in $\R^2$.
  Find a family of linear inequalities as in the previous problem
  with solution set $S$.
\end{exercise}

\begin{solution}
  Let $U = \{ (\cos\theta, \sin\theta) : 0 \leq \theta < 2\pi \}$ be
  the unit circle in $\R^2$.
  Then, for each $(u, v) \in U$, we have the linear inequalities
  \begin{equation}
    u x_1 + v x_2 \leq 1.
  \end{equation}
  This feels like a circular argument, but alright.
\end{solution}

\begin{exercise}
  Is the unit ball $B = \{ x \in \R^n : \norm{x}_2 \leq 1 \}$ a polyhedron?
\end{exercise}

\begin{solution}
  I don't believe the unit ball is a polyhedron.
  A polyhedron requires a \emph{finite} number of linear
  inequalities, i.e.\ constraints which can be written as $Ax \leq b$.
  The unit ball however is inherently smooth, without edges, and
  therefore requires an infinite number of linear constraints, as
  those applied in the previous exercise.
\end{solution}

\begin{exercise}
  Show that the unit ball $B_\infty = \{ x \in \R^n : \norm{x}_\infty
  \leq 1 \}$ is convex.
  Here $\norm{x}_\infty = \max_j \abs{x_j}$ is the max norm of $x$.
  Show that $B_\infty$ is a polyhedron.
  Illustrate when $n = 2$.
\end{exercise}

\begin{solution}
  Let $x, y \in B_\infty$ and $0 \leq \lambda \leq 1$.
  We then have
  \begin{equation}
    \norm{\lambda x + (1 - \lambda) y}_\infty \leq \lambda
    \norm{x}_\infty + (1 - \lambda) \norm{y}_\infty \leq \lambda + (1
    - \lambda) = 1,
  \end{equation}
  so $\lambda x + (1 - \lambda) y \in B_\infty$, and $B_\infty$ is convex.
  $B_\infty$ is a polyhedron, as it is described by the constraints
  \begin{equation}
    x_j \leq 1
    \quad\text{and}\quad
    -x_j \leq 1
    \qquad
    j = 1, \ldots, n.
  \end{equation}
  When $n = 2$, $B_\infty$ is the unit square, with corners $(\pm 1,
  \pm 1)$ and $(\pm 1, \mp 1)$, as seen in \cref{fig:unit_square}.

  \begin{figure}[htbp]
    \centering

    \resizebox{0.4\textwidth}{!}{
      \begin{tikzpicture}
  \coordinate (A) at (1, 1);
  \coordinate (B) at (1, -1);
  \coordinate (C) at (-1, -1);
  \coordinate (D) at (-1, 1);

  \fill[my_red, opacity=0.18] (A) -- (B) -- (C) -- (D) -- cycle;
  \draw[very thick, my_red] (A) -- (B) -- (C) -- (D) -- cycle;

  \draw[gray, dashed] (-1.5, 0) -- (1.5, 0);
  \draw[gray, dashed] (0, -1.5) -- (0, 1.5);

  \node at (0, 1) [above left] {\tiny$1$};
  \node at (1, 0) [below right] {\tiny$1$};
  \node at (0, -1) [below left] {\tiny$-1$};
  \node at (-1, 0) [below left] {\tiny$-1$};
\end{tikzpicture}
    }
    \caption{$B_\infty$ when $n = 2$.\label{fig:unit_square}}
  \end{figure}
\end{solution}

\begin{exercise}
  Show that the unit ball $B_1 = \{ x \in \R^n : \norm{x}_1 \leq 1
  \}$ is convex.
  Here $\norm{x}_1 = \sum_{j=1}^n \abs{x_j}$ is the absolute norm of $x$.
  Show that $B_1$ is a polyhedron.
  Illustrate when $n = 2$.
\end{exercise}

\begin{solution}
  Let $x, y \in B_1$ and $0 \leq \lambda \leq 1$.
  Then we have
  \begin{equation}
    \begin{split}
      \norm{\lambda x + (1 - \lambda) y}_1
      &= \sum_{j=1}^n \abs{\lambda x_j + (1 - \lambda) y_j} \\
      &\leq \sum_{j=1}^n \lambda \abs{x_j} + (1 - \lambda) \abs{y_j}
      = \lambda \sum_{j=1}^n \abs{x_j} + (1 - \lambda)\sum_{j=1}^n \abs{y_j} \\
      &\leq \lambda + (1 - \lambda) = 1.
    \end{split}
  \end{equation}
  Therefore, $\lambda x + (1 - \lambda) y \in B_1$, and $B_1$ is convex.

  $B_1$ is a polyhedron, as it is described by the constraints
  \begin{equation}
    \sum_{j = 1}^{n} \sigma_j x_j \leq 1, \quad \forall (\sigma_1,
    \ldots, \sigma_n) \in \{-1, 1\}^n.
  \end{equation}
  When $n = 2$, $B_1$ is the unit diamond, with corners $(\pm 1, 0)$
  and $(0, \pm 1)$, as seen in \cref{fig:unit_diamond}.

  \begin{figure}[htbp]
    \centering
    \resizebox{0.4\textwidth}{!}{
      \input{1_basic_concepts/unit_diamond.tex}
    }
    \caption{$B_1$ when $n = 2$.\label{fig:unit_diamond}}
  \end{figure}
\end{solution}

\paragraph{Proposition 1.5.1} (\emph{Affine sets}).
Let $C$ be a non-empty subset of $\R^n$.
Then $C$ is an affine set if an only if there is a matrix $A \in
\R^{m \times n}$ and a vector $b \in \R^m$ for some $m$ such that
\begin{equation}
  C = \{ x \in \R^n : Ax = b \}.
\end{equation}
Moreover, $C$ may be written as $C = L + x_0 = \{ x + x_0 : x \in L
\}$ for some linear subspace $L$ of $\R^n$.
The subspace $L$ is unique.

\begin{exercise}
  Prove Proposition 1.5.1.
\end{exercise}

\begin{solution}
  Let $x_0 \in C$, where $C$ is affine, and $L = \{ x - x_0 : x \in C \}$.
  For an arbitrary $x \in C$, we have that
  \begin{equation}
    \lambda (x - x_0) = \lambda x + (1 - \lambda) x_0 - x_0 \in L.
  \end{equation}
  Finish later, this is getting long.
\end{solution}

\begin{exercise}
  Let $C$ be a non-empty affine set in $\R^n$.
  Define $L = C - C$.
  Show that $L$ is a subspace and that $C = L + x_0$ for some vector $x_0$.
\end{exercise}

\begin{solution}
  We have that
  \begin{equation}
    L = C - C = \{ x - y : x, y \in C \}.
  \end{equation}
  We have from the previous exercise that $C = L + x_0$ for a subspace $L$.
  Letting $x,y \in C$, we can express these as
  \begin{equation}
    x = \alpha + x_0
    \quad\text{and}\quad
    y = \beta + x_0
  \end{equation}
  for some $\alpha, \beta \in L$.
  Then $x - y = \alpha - \beta \in L$, showing that $C - C \subseteq L$.
  For the reverse inclusion, let $\alpha \in L$.
  Then $\alpha = x - x_0$ for some $x \in C$, showing that $L
  \subseteq C - C$, proving that $L = C - C$.
\end{solution}
