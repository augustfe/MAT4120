\section{Notes on Combinatorial Optimization}

\begin{exercise}
  Consider the knapsack problem
  \begin{equation}
    \max \left\{
      \sum_{j = 1}^{n} c_j x_j :
      \sum_{j = 1}^{n} a_j x_j \leq b,
      0 \leq x \leq 1
    \right\}
  \end{equation}
  where all the data are positive integers.
  Define the knapsack relaxation polytope by
  \begin{equation}
    P = \{
      x \in \R^n :
      \sum_{j = 1}^{n} a_j x_j \leq b,
      0 \leq x \leq 1
    \}.
  \end{equation}
  Assume that the variables have been ordered such that $c_1 / a_1 \geq c_2 / a_2 \geq \cdots \geq c_n / a_n$.
  Try to guess an optimal solution and prove the optimality by considering the dual problem.
  Use this result to characterize all the vertices of $P$.
  What about the cover inequalities in relation to these vertices?
\end{exercise}

\begin{solution}
  We guess a solution through applying a greedy strategy.
  Let $A_k = \sum_{j=1}^{k} a_j$.
  If $A_n \leq b$, then the optimal solution is clearly $x^* = 1$.
  Otherwise, let $k$ be the smallest index such that $A_k > b$.
  Then we set
  \begin{equation}
    x_j^* =
    \begin{cases}
      1, & j = 1, \ldots, k - 1, \\
      \frac{b - A_{k - 1}}{a_{k}}, & j = k, \\
      0, & j = k + 1, \ldots, n,
    \end{cases}
  \end{equation}
  i.e., we fill the knapsack with items in order of their value-to-weight ratio until we can no longer fit a full item, at which point we take a fraction of the next item to fill the knapsack exactly.

  To prove optimality, we consider the dual problem.
  The primal problem is given by
  \begin{equation}
    \begin{array}{rrcl}
      \text{maximize} & c^T x \\
      \text{subject to} & a^T x & \leq & b, \\
      & x & \leq & 1, \\
      & x & \geq & 0.
    \end{array}
  \end{equation}
  The dual is then given by
  \begin{equation}
    \begin{array}{rrcl}
      \text{minimize} & b y + 1^T z \\
      \text{subject to} & a_j y + z_j & \geq & c_j, \quad j = 1, \ldots, n, \\
      & y, z & \geq & 0.
    \end{array}
  \end{equation}
  We propose the dual solution
  \begin{equation}
    y^* = \frac{c_{k}}{a_{k}}, \quad
    z_j^* = \max\{ c_j - a_j y^*, 0 \}
  \end{equation}
  As the ratios are non-increasing, we have $c_j/a_j \geq c_k/a_k$ for $j < k$, so
  \begin{equation}
    a_j y^* + z_j^*
    = a_j \frac{c_k}{a_k} + \max\{ c_j - a_j \frac{c_k}{a_k}, 0 \}
    \geq c_j,
  \end{equation}
  so the dual constraints are satisfied.
  The dual objective value is
  \begin{equation}
    b y^* + 1^T z^*
    = b \frac{c_k}{a_k} + \sum_{j=1}^{k-1} (c_j - a_j \frac{c_k}{a_k}) + \sum_{j=k+1}^{n} c_j 0
    = \sum_{j=1}^{k-1} c_j + \frac{b - A_{k-1}}{a_k} c_k,
  \end{equation}
  which matches the primal objective value at $x^*$.
  By strong duality, $x^*$ is optimal.

  At any vertex of $P$, we must have $n$ linearly independent constraints active.
  If the knapsack constraint is not active, we must then have $x \in \{0, 1\}^n$.
  If the knapsack constraint is active, we can have at most $n - 1$ variables at their bounds, meaning that exactly one variable is fractional.
  Thus, the vertices of $P$ are exactly those points where either all variables are binary, or exactly one variable is fractional as in the solution $x^*$ above.

  Regarding cover inequalities, they are valid for the integer knapsack polytope but may cut off fractional vertices of $P$.
  Specifically, a cover inequality is derived from a subset of items whose total weight exceeds the knapsack capacity.
  Such inequalities can eliminate fractional solutions where a fractional variable corresponds to an item in the cover set, thus tightening the relaxation towards the integer hull.
\end{solution}

\begin{exercise}
  Consider the branch-and-bound algorithm.
  Consider a node $u$ in the enumeration tree with $v(R(u)) > v(Q)$ where $Q$ is the integer program and $R(u)$ is the LP relaxation in node $u$.
  Can we prune node $u$?
\end{exercise}

\begin{solution}
  No, we cannot prune node $u$.
  The condition $v(R(u)) > v(Q)$ indicates that the optimal value of the LP relaxation at node $u$ is greater than the optimal value of the integer program $Q$.
  This means that there is potential for finding a better integer solution in the subtree rooted at node $u$.
  Therefore, we must continue exploring this node and its children to ensure that we do not miss any optimal integer solutions.
\end{solution}

\begin{exercise}
  Prove, in detail, that (4.12) is a valid integer linear programming formulation of the TSP problem.
  Then do the same for the model obtained by replacing the cut inequalities by the subtour inequalities.
\end{exercise}

\begin{solution}
  The first constraint ensures that each node has exactly degree two, which is necessary for forming a tour.
  The second constraint ensures that for each cut in the graph, there is at least two edges crossing the cut, preventing the formation of subtours.
  Lastly, a binary variable is used to indicate whether an edge is included in the tour or not.

  These constraints together ensure that any feasible solution corresponds to a Hamiltonian cycle.
  The travel salesman problem is then simply to minimize the total length subject to these constraints.

  The equivalent formulation replaces the cut inequalities with subtour elimination constraints, i.e.,
  \begin{equation}
    x(E[S]) \leq \abs{S} - 1 \quad \text{for all } S \subset V, S \neq \emptyset, S \neq V,
  \end{equation}
  where $E[S]$ is the set of edges with both endpoints in $S$.
  A tour contains the same number of edges as nodes, so any subset of nodes $S$ in a tour can contain at most $\abs{S} - 1$ edges to avoid forming a subtour.
\end{solution}

\begin{exercise}
  Try to figure out what the odd cover inequalities might be based on in the examples given for the set covering problem.
\end{exercise}

\begin{solution}
  The examples given are:
  \begin{enumerate}
    \item Airline crew scheduling: Each flight must be covered by at least one crew.
    \item Allocating student classes to rooms: Each class must be assigned to a room.
    \item Network design: Certain capacity bottlenecks (like cuts) must be covered to ensure network connectivity.
  \end{enumerate}

  {\large I am very unsure about this one.}
  In each of these examples, the odd cover inequalities can be interpreted as follows:
  \begin{enumerate}
    \item In airline crew scheduling, an odd cover inequality could ensure that at any given time, the number of crews assigned to flights is sufficient to cover all flights.
    \item In allocating student classes to rooms, an odd cover inequality could ensure that for any subset of classes occurring at the same time, the number of rooms assigned is enough to accommodate all classes in that subset.
    \item In network design, an odd cover inequality could ensure that for any odd number of critical network components, there are enough resources allocated to cover them, preventing any component from being left unprotected.
  \end{enumerate}
\end{solution}

\begin{exercise}
  Consider the degree-constrained spanning tree problem.
  Find a valid integer linear programming formulation of this problem.
\end{exercise}
