% chktex-file 15
% chktex-file 1
% chktex-file 8

\usepackage{xparse}

\newtheoremstyle{exerciseStyle}
{ } % Space above
{ } % Space below
{\normalfont} % Body font
{ } % Indent amount
{\bfseries} % Theorem head font
{.} % Punctuation after theorem head
{ } % Space after theorem head
{\thmname{#1}\thmnumber{ #2}} % Theorem head spec (can be left empty, meaning `normal`)
\theoremstyle{exerciseStyle}
\newtheorem{exercise}{Exercise}[section]

\newtheoremstyle{solutionStyle}
{ } % Space above
{ } % Space below
{\normalfont} % Body font
{ } % Indent amount
{\bfseries} % Theorem head font
{.} % Punctuation after theorem head
{ } % Space after theorem head
{\thmname{#1}\thmnumber{ #2}} % Theorem head spec (can be left empty, meaning `normal`)
\theoremstyle{solutionStyle}
\newtheorem{solution}{Solution}[section]

\newcommand{\setexsol}[1]{%
  \setcounter{exercise}{\numexpr#1-1\relax}%
  \setcounter{solution}{\numexpr#1-1\relax}%
}

\newtheoremstyle{manualStyle}
{ }{ }{\normalfont}{ }{\bfseries}{}{ }% no punctuation here
{\thmname{#1}\thmnumber{ #2}{\normalfont\thmnote{ (#3)}}.}

\theoremstyle{manualStyle}

\makeatletter
% \DeclareManualTheorem{<shortname>}{<Printed Heading>}
% -> defines:
%    - innermanual<shortname> (the actual theorem env)
%    - manual<shortname>  (wrapper with manual number + optional note)
% tex-fmt: off
\NewDocumentCommand{\DeclareManualTheorem}{mm}{%
  \newtheorem{innermanual#1}{#2}%
  \NewDocumentEnvironment{manual#1}{m o}
  {%
    \@namedef{theinnermanual#1}{##1}%
    \IfNoValueTF{##2}%
    {%
      \begin{innermanual#1}%
    }{%
      \begin{innermanual#1}[##2]%
    }%
  }{%
    \end{innermanual#1}%
  }%
}
% tex-fmt: on
\makeatother

% Instantiate your three:
\DeclareManualTheorem{prop}{Proposition}
\DeclareManualTheorem{lemma}{Lemma}
\DeclareManualTheorem{theorem}{Theorem}
\DeclareManualTheorem{corollary}{Corollary}
